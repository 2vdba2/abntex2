%~~~~~~~~~~~~~~~~~~~~~~~~~~~~~~~~~~~~~~~~~~~~~~~~~~~~~~~~~~~~~~~~~~~~~
%
%    File      : introdução
%    Type      : TeX
%    Date      : terça-feira, março 19, 2012 at 10:12
%
%    Content   : Coloque aqui a introdução em de sua tese/dissertação.
%~~~~~~~~~~~~~~~~~~~~~~~~~~~~~~~~~~~~~~~~~~~~~~~~~~~~~~~~~~~~~~~~~~~~~


\label{cap_introducao}

Esta  pesquisa  pretende  mostrar  que  \ldots   através  de  \ldots
    conforme   concepçõees   apresentadas   por  [...].    Para   isso,
    articulamos o conceito de [...]  com o conceito de [...].  Fizemos
    pesquisas de recepção conforme  [...]. Articulamos os resultados a
    partir de  ideias de [...].  Neste primeiro parágrafo  você deve
    deixar  completamente claro  o  que pretende  com  o trabalho.  

    A introdução é redigida depois de escrito todo o trabalho porque,
    no decorrer da pesquisa, algumas coisas podem ser modificadas em
    relação ao projeto original.

    Depois,  de alguns parágrafos,   você  deve  falar   sobre  a
    problematização,   a   contextualização   histórica,   a   revisão
    bibliográfica,  os objetivos, a  justificativa, a  metodologia. As
    conclusões, evidentemente,  devem ficar no  capítulo Considerações
    Finais,  para  que  o  leitor  não  perca  o  interesse  pelo  seu
    trabalho.  Toda a  introdução  é feita  sem  subtítulos, em  texto
    normal.

    Tipicamente a introdução deve conter uma abordagem geral ao tema
    que justifique a razao para o trabalho ter sido desenvolvido. Deve
    resumir o estado da arte, se possível referindo outros trabalhos
    semelhantes. Os objetivos do trabalho devem estar apresentados de
    forma inequívoca e direta.
    
