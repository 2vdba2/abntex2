%
% Documento de teste do abnTeX2
%
\documentclass[12pt,twoside,openright,a4paper]{abntex2}

	% PACOTES FUNDAMENTAIS
 	\usepackage[T1]{fontenc}		% seleção de códigos de fonte.
	\usepackage[utf8]{inputenc}		% determina a codificação utiizada (conversão automática dos acentos)
	\usepackage{makeidx}            % cria o indice
	\usepackage[font=footnotesize,skip=4pt]{caption}[2011/11/10]	% captions
    \usepackage{bookmark}
    \usepackage{hyperref}  % controla a formação do índice

    
    % PACOTES ADICIONAIS, APENAS PARA TESTE DE FUNCIONALIDADES
    \usepackage{lipsum}				% para geração de dummy text
    \usepackage{mychemistry}	    % Desenho de estruturas químicas
    

	% CONFIGURACOES DE PACOTES
     \hypersetup{
     	backref=true,
     	%pagebackref=true,
		%bookmarks=true,         		% show bookmarks bar?
		pdftitle={TITULO DO PDF}, 
		pdfauthor={AUTORES},
		pdfkeywords={PALAVRAS CHAVES EM PORTUGUES},
    		pdfsubject={EPIGRAFE},
	    pdfproducer={Latex}, 	% producer of the document
    		colorlinks=true,       		% false: boxed links; true: colored links
    		linkcolor=blue,          	% color of internal links
    		citecolor=blue,        		% color of links to bibliography
    		filecolor=magenta,      		% color of file links
		urlcolor=blue,
		bookmarksdepth=4
		}
    

% compila o indice
\makeindex

\begin{document}

%%%%%%%%%%%%%%%%%%%%%%%%%%%%%%%%%%%%%%%%%%%%%%%%%%%%%%%%%%%%
% ELEMENTOS PRÉ-TEXTUAIS
%%%%%%%%%%%%%%%%%%%%%%%%%%%%%%%%%%%%%%%%%%%%%%%%%%%%%%%%%%%%

% inserir a capa
% \capa

% inserir a folha de rosto
% \folhaderosto

% inserir a ficha bibliografica
% \fichabibliografica

% inserir folha de aprovação
% \folhaaprovacao

% dedicatória
% \dedicatoria

% agradecimentos
% \agradecimentos

% epígrafe
% \epigrafe

% inserir o resumo em português  {título do resumo}{palavras chaves}
% \begin{resumo}{Reumo}{latex, abntex, editoração de texto}
%  Este é o resumo em português
% \end{resumo}

% inserir o resumo em inglês {título do resumo}{palavras chaves}
% \begin{resumo}{Abstract}{latex, abntex, text editing}
%  This is the english abstract.
% \end{resumo}

% inserir o resumo em francês {título do resumo}{palavras chaves}
% \begin{resumo}{Résumé}{latex, abntex, l'édition de texte}
%  Ceci est le résumé français.
% \end{resumo}

% inserir o resumo em espanhol {título do resumo}{palavras chaves}
% \begin{resumo}{Resumen}{latex, abntex, edición de texto}
%  Este es el resumen de español.
% \end{resumo}

% inserir lista de ilustrações
\listoffigures

% inserir lista de tabelas
\listoftables

% inserir lista de Abreviaturas e Siglas
%\listofabreviaturasSiglas

% inserir lista de símbolos
%\listofsimbols

% inserir o sumario
\tableofcontents



%%%%%%%%%%%%%%%%%%%%%%%%%%%%%%%%%%%%%%%%%%%%%%%%%%%%%%%%%%%%
% ELEMENTOS TEXTUAIS
%%%%%%%%%%%%%%%%%%%%%%%%%%%%%%%%%%%%%%%%%%%%%%%%%%%%%%%%%%%%

% ----------------------------------------------------------
% Introdução
% ----------------------------------------------------------
\chapter{Introdução}

\lipsum[1-2]


% ----------------------------------------------------------
% Capitulo 1
% ----------------------------------------------------------
\chapter{Lorem ipsum dolor sit amet}

\lipsum[3]

\begin{itemize}
  \item Item 1
  \item Item 2\footnote{As notas devem ser digitadas ou datilografadas dentro
  das margens, ficando separadas do texto por um espaço simples de entre as
  linhas e por filete de 5 cm, a partir da margem esquerda. Devem ser
  alinhadas, a partir da segunda linha da mesma nota, abaixo da primeira letra
  da primeira palavra, de forma a destacar o expoente, sem espaço entre elas e
  com fonte menor. NBR 14724/2011 - 5.2.1}
  \item Item 3\footnote{Item 3 footnote.}
\end{itemize}

% \begin{resumo}
% Este é o resumo que deve funcionar.
% \end{resumo}

\lipsum[4]

\begin{table}[htb]
\footnotesize
\caption[Níveis de investigação]{\footnotesize{Níveis de investigação.
\cite{van86}}}
\label{tab-nivinv}
\begin{tabular}{p{2.6cm}|p{6.0cm}|p{2.25cm}|p{3.40cm}}
  %\hline
   \textbf{Nível de Investigação} & \textbf{Insumos}  & \textbf{Sistemas de Investigação}  & \textbf{Produtos}  \\
    \hline
    Meta-nível & Filosofia\index{Filosofia} da Ciência  & Epistemologia &
    Paradigma  \\
    \hline
    Nível do objeto & Paradigmas do metanível e evidências do nível inferior &
    Ciência  & Teorias e modelos \\
    \hline
    Nível inferior & Modelos e métodos do nível do objeto e problemas do nível inferior & Prática & Solução de problemas  \\
   % \hline
\end{tabular}
\end{table}

\lipsum[5]

% figura química 

\begin{figure}[htb]
	\begin{center}
		\makevisible
		\colorlet{mCgreen}{green!50!gray}
		\colorlet{mCblue}{cyan!50!gray}
		\colorlet{mCred}{magenta!50!gray}
		\colorlet{mCyellow}{yellow!50!gray}
		\begin{rxn}
			\tikzset{reactant/.style={draw=#1,fill=#1!10,inner sep=1em,minimum height=10em,minimum width=12em,rounded corners}}
			\reactant[,cytosine,reactant=mCred]{\chemfig{H-[:30]N*6(-(=O)-N=(-NH _2)-=-)}}
			\anywhere{cytosine.-90,,yshift=-2mm}{Citosina}
			\reactant[,thymine,reactant=mCyellow]{\chemfig{H-[:30]N*6(-(=O)-N(-H)
			-(=O)-(-CH_3)=-)}}
			\anywhere{thymine.-90,,yshift=-2mm}{Timina}
			\reactant[cytosine.-90,adenine,yshift=-2em,reactant=mCblue]{\chemfig
			{[:-36]*5(-N(-H)-*6(-N=-N=(-NH_2)--)--N=)}}
			\anywhere{adenine.-90,Guanin,yshift=-2mm}{Adenina}
			\reactant[,guanine,reactant=mCgreen]{\chemfig{[:-36]*5(-N(-H)-*6(-N=(-
			NH_2)-N(-H)-(=O)--)--N=)}}
			\anywhere{guanine.-90,,yshift=-2mm}{Guanina}
		\end{rxn}
	
	    \caption{\label{fig_aromaticos}Configurações de moléculas químicas}
	    \caption*{Fonte: autores}
	\end{center}
\end{figure}

\lipsum[6]


\section{Seção 1}

\lipsum[7]

\begin{citacao}
As citações diretas, no texto, com mais de três linhas, devem ser
destacadas com recuo de 4 cm da margem esquerda, com letra menor que a do texto
utilizado e sem as aspas. No caso de documentos datilografados, deve-se
observar apenas o recuo. (ABNT NBR 10520/2002 - 5.3) 5.1 Formato
(...)
Recomenda-se, quando digitado, a fonte tamanho 12 para todo o trabalho,
inclusive capa, excetuando-se citaçõess com mais de três linhas, notas de
rodapé, paginação, dados internacionais de catalogação-na-publicação,
legendas e fontes das ilustrações e das tabelas, que devem ser em tamanho menor
e uniforme.
\end{citacao}

\subsection{Subseção 1}

\lipsum[8-10]

\subsection{Subseção 2}

Plain text.

\lipsum[11-15]

\subsection{Subseção 3}

More plain text.

\lipsum[16-20]

More plain text\index{plain text}.

\subsubsection{Subsubseção 3.1}

More plain text\index{plain text}.

\lipsum[21-25]



% ----------------------------------------------------------
% capitulo Segundo assunto
% ----------------------------------------------------------

\chapter{Etiam eget ligula eu lectus lobortis condimentum}

\lipsum[1-3]

\section{Seção 1}

\lipsum[4-5]

\subsection{Subseção 1}

Plain text.

\lipsum[6-10]

\subsection{Subseção 2}

Plain text\index{plain text}.

\lipsum[11-15]

\subsection{Subseção 3}

More plain text.

\lipsum[16-17]

\subsubsection{Subsubseção 3.1}

More plain text.

\lipsum[18-20]

%%%%%%%%%%%%%%%%%%%%%%%%%%%%%%%%%%%%%%%%%%%%%%%%%%%%%%%%%%%%
% ELEMENTOS PÓS-TEXTUAIS
%%%%%%%%%%%%%%%%%%%%%%%%%%%%%%%%%%%%%%%%%%%%%%%%%%%%%%%%%%%%

% ----------------------------------------------------------
% Conclusão
% ----------------------------------------------------------

\chapter*{Conclusão}

\lipsum[21-23]

% ----------------------------------------------------------
% Referências bibliográficas
% ----------------------------------------------------------
\bibliographystyle{plain}	% (uses file "plain.bst")
\bibliography{references}


% ----------------------------------------------------------
% Glossário
% ----------------------------------------------------------

%
% Há diversas soluções prontas para glossário. Não é necessário nos preocuparmos
% com isso agora.
%
%\glossary

% ----------------------------------------------------------
% Apêndices
% ----------------------------------------------------------
\appendix

% ----------------------------------------------------------
\chapter{Primeiro apêndice}
% ----------------------------------------------------------

\lipsum[50-52]

% ----------------------------------------------------------
\chapter{Segundo apêndice}
% ----------------------------------------------------------
\lipsum[55-57]

% ----------------------------------------------------------
% Anexos
% ----------------------------------------------------------

% \annex
% 
% % ----------------------------------------------------------
% \chapter{Primeiro anexo}
% % ----------------------------------------------------------
% 
% \lipsum[200-202]
% 
% % ----------------------------------------------------------
% \chapter{Segundo anexo}
% % ----------------------------------------------------------
% \lipsum[210-212]


%---------------------------------------------------------------------
% INDICE REMISSIVO
%---------------------------------------------------------------------

\cleardoublepage
\phantomsection 
\addcontentsline{toc}{chapter}{\indexname}
% \ProximoForaDoSumario
\printindex

\end{document}
