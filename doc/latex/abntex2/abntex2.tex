%% abtex2.tex, v<VERSION> laurocesar
%% Copyright 2012-2013 by abnTeX2 group at http://code.google.com/p/abntex2/ 
%%
%% This work may be distributed and/or modified under the
%% conditions of the LaTeX Project Public License, either version 1.3
%% of this license or (at your option) any later version.
%% The latest version of this license is in
%%   http://www.latex-project.org/lppl.txt
%% and version 1.3 or later is part of all distributions of LaTeX
%% version 2005/12/01 or later.
%%
%% This work has the LPPL maintenance status `maintained'.
%% 
%% The Current Maintainer of this work is the abnTeX2 team, led
%% by Lauro César Araujo. Further information are available on 
%% http://code.google.com/p/abntex2/
%%
%% This work consists of the file abntex2.tex
%%
%%
%% 2013.5.24 10h29	laurocesar
%%  Adiciona seção referente ao espaçamento entre o capítulo e o início do
%% texto.
%%  Documenta a nova opção language do ambiente citacao.
%%
%% 2013.5.18 09h51	sggranja / laurocesar
%%  Adiciona seção referente às opções chapter=TITLE, section=TITLE,
%% subsection=TITLE e subsubsection=TITLE.
%%
%% 2013.4.5 08h04	laurocesar
%%  Revisa alterações de gilsonolegario
%%
%% 2013.4.4 09h05	wagner@admwagner.com.br
%%  Revisão.
%%
%% 2013.4.01 21:00	gilsonolegario
%% Melhorias visuais da documentação
%%
%% 2013.3.19 09h32	laurocesar
%%  Adiciona orientações sobre a produção de documentos em outros idiomas.
%%
%% 2013.3.18 10h42	laurocesar
%%  Adiciona traduções à seção do babel
%%
%% 2013.3.17 11h42	laurocesar
%%  Adiciona comentários sobre as macros \and e \\ na descrição da macro \autor.
%%  Adiciona informação de que \titulo, \autor e \data são traduções de \title,
%% \autor e \date.
%%  Revisa a seção referente ao \hypersetup
%%
%% 2013.3.14 10h53	laurocesar
%%  Revisa a seção Referências bibliográficas
%%
%% 2013.3.12 08h04	laurocesar
%%  Altera a posição da instituição na forma de rosto
%%
%% 2013.3.11 10h42	laurocesar
%%  Inclui referências ao novo exemplo de glossários
%%
%% 2013.3.10 10h36	laurocesar
%%  Adiciona informações sobre pacotes inputenc e fontenc incorporados por
%% padrão quando xelatex não é usado.
%%
%% 2013.2.23 13h48	laurocesar
%%  Adiciona lista de siglas e abreviaturas e lista de símbolos
%%
%% 2013.2.17 11h09	laurocesar
%%  Inclui informações sobre hifenização em diferentes idiomas na seção sobre
%%  o pacote babel e na seção referente aos resumos.
%%
%% 2013.2.6 22h35	laurocesar
%%  Inclui informação de compatibilidade com a versão 2002 da ABNT NBR 6023 e
%%  informação de que os manuais do cite estão desatualizados.
%%  Corrige erro na instrução referente \partanexos* (estava \partname*)
%%  Remove \vspace*{1cm} na parte superior dos exemplos de capa, folha de
%%  rosto e folha de aprovação
%%
%% 2013.2.2 15h0	laurocesar
%%  Adiciona subseções referentes às macros \partanexos e \partapendices
%%
%% 2013.2.2 15h0	laurocesar
%%  Adiciona subseções referentes às opções de documentclass.
%%  Adiciona seção referente à codificação UTF8 do documento. 
%%
%% 2013.1.22 08h31	laurocesar
%%  Adiciona referências a enumitem em Alíneas e Sunalineas
%%
%% 2013.1.19 08h03	laurocesar
%%  Revisão do documento
%%
%% 2013.1.17 19h18	laurocesar
%%  Adiciona seção explicativa sobre as fontes.
%%  Adiciona \bfseries nos exemplos de capa, folha de rosto e folha de aprovacao
%%  Altera a definicao de \ABNTEXchapterfont
%%  Adiciona explicação da alteração de \listtablename do pacote babel 
%%  Revisões pontuais do texto
%%  Adiciona informações no histórico
%%  Inclui revisão de Youssef Cherem
%%
%% 2013.1.14 21h03	laurocesar
%%  Adiciona referência à norma ABNT NBR 6022:2003
%%  Adiciona informação de cabeçalhos com o comando \chapter*
%%  Revisa conteúdo e relacionamento entre as seções
%%  Altera o nome do documento de ``A classe abntex2: Modelo canônico de
%%   trabalhos acadêmicos brasileiros compatível com as normas ABNT NBR
%%   14724:2011, ABNT NBR 6024:2012 e outras'' para ``A classe
%%   abntex2: Documentos técnicos e científicos brasileiros compatíveis com as
%%   normas ABNT''
%%
%% 2013.1.13 09h46 laurocesar
%%  Adiciona explicação do recuo do ambiente citacao
%%  Adiciona explicação sobre espaçamentos entre parágrafos e linhas
%%
%% 2013.1.9 21h44 laurocesar
%%  Inclusão da norma ABNT NBR ABNT NBR 10719-2011 - Informação e
%% documentação - Relatório técnico e-ou científico - Apresentação
%%
%% 2013.1.6 11h43 laurocesar
%%  Inclusão da norma ABNT NBR 15287:2011: Informação e documentação - Projeto
%% de pesquisa - Apresentação
%%
%% 2013.1.3 09h08 laurocesar
%%  Documentação de \title{} \author{} e \date{} nos equivalentes em portugues.
%%
%% 2012.12.2 09h38 laurocesar
%%  Criação do arquivo 
%%

\documentclass[a4paper]{ltxdoc}
\usepackage{lmodern}			% Usa a fonte Latin Modern			
\usepackage[T1]{fontenc}		% seleção de códigos de fonte.
\usepackage[utf8]{inputenc}		% determina a codificação utiizada (conversão automática dos acentos)
\usepackage{hyperref}  			% controla a formação do índice
\usepackage{parskip}			% espaçamento entre os parágrafos


% Babel e ajustes
\usepackage[brazil]{babel}		% idiomas
\addto\captionsbrazil{
    %% ajusta nomes padroes do babel
    \renewcommand{\bibname}{Refer\^encias}
    \renewcommand{\indexname}{\'Indice}
    \renewcommand{\listfigurename}{Lista de ilustra\c{c}\~{o}es}
    \renewcommand{\listtablename}{Lista de tabelas}
    %% ajusta nomes usados com a macro \autoref
    \renewcommand{\pageautorefname}{p\'agina}
    \renewcommand{\sectionautorefname}{se{\c c}\~ao}
    \renewcommand{\subsectionautorefname}{subse{\c c}\~ao}
    \renewcommand{\paragraphautorefname}{par\'agrafo}
    \renewcommand{\subsubsectionautorefname}{subse{\c c}\~ao}
}  

\usepackage{color}
\definecolor{thered}{rgb}{0.65,0.04,0.07}
\definecolor{thegreen}{rgb}{0.06,0.44,0.08}
\definecolor{thegrey}{gray}{0.5}
\definecolor{theshade}{rgb}{1,1,0.97}
\definecolor{theframe}{gray}{0.6}
\definecolor{blue}{RGB}{41,5,195}

\IfFileExists{listings.sty}{
  \usepackage{listings}
\lstset{%
	language=[LaTeX]TeX,
	columns=flexible,
	basicstyle=\ttfamily\small,
	backgroundcolor=\color{theshade},
	frame=single,
	tabsize=2,
	rulecolor=\color{theframe},
	title=\lstname,
	escapeinside={\%*}{*)},
	breaklines=true,
	commentstyle=\color{thegrey},
	keywords=[0]{\fichacatalografica,\errata,\folhadeaprovacao,\dedicatoria,\agradecimentos,\epigrafe,\resumo,\siglas,\simbolos,\citacao,\alineas,\subalineas,\incisos},
	keywordstyle=[0]\color{thered},
	keywords=[1]{},
	keywordstyle=[1]\color{thegreen},
	breakatwhitespace=true,
	alsoother={0123456789_},
	inputencoding=utf8,
	extendedchars=true,
	literate={á}{{\'a}}1 {ã}{{\~a}}1 {é}{{\'e}}1 {è}{{\`{e}}}1 {ê}{{\^{e}}}1 {ë}{{\¨{e}}}1 {É}{{\'{E}}}1 {Ê}{{\^{E}}}1 {û}{{\^{u}}}1 {ú}{{\'{u}}}1 {â}{{\^{a}}}1 {à}{{\`{a}}}1 {á}{{\'{a}}}1 {ã}{{\~{a}}}1 {Á}{{\'{A}}}1 {Â}{{\^{A}}}1 {Ã}{{\~{A}}}1 {ç}{{\c{c}}}1 {Ç}{{\c{C}}}1 {õ}{{\~{o}}}1 {ó}{{\'{o}}}1 {ô}{{\^{o}}}1 {Õ}{{\~{O}}}1 {Ó}{{\'{O}}}1 {Ô}{{\^{O}}}1 {î}{{\^{i}}}1 {Î}{{\^{I}}}1 {í}{{\'{i}}}1 {Í}{{\~{Í}}}1,
}
\let\verbatim\relax
 	\lstnewenvironment{verbatim}[1][]{\lstset{##1}}{}
}


\usepackage[alf]{abntex2cite}	% citacoes


% COMANDOS PROPRIOS
\newcommand{\abnTeX}{abn\TeX}
\newcommand{\abnTeXForum}{\url{http://groups.google.com/group/abntex2}}
\newcommand{\abnTeXSite}{\url{http://abntex2.googlecode.com/}}

\title{\textbf{A classe \textsf{abntex2}}: \\ \Large{Documentos
técnicos e científicos brasileiros \\compatíveis com as normas ABNT}}

%   \thanks{Este documento
%   se referete ao \textsf{abntex2} versão \fileversion,
%   de \filedate.}
  
  
\author{Equipe \abnTeX2\\\abnTeXSite 
\and 
Lauro César Araujo\\\url{laurocesar@laurocesar.com}}

\date{\today, v<VERSION>}

\hypersetup{
		pdftitle={A classe abntex2},
		pdfauthor={Equipe abnTeX2},
    	pdfsubject={Documentos técnicos e científicos brasileiros compatíveis com
    	as normas ABNT}, 
    	pdfkeywords={ABNT}{abntex}{abntex2}{trabalho
		acadêmico}{dissertação}{tese}{ABNT 14724}{ABNT 6024}, 
		pdfproducer={Lauro César Araujo -- laurocesar@laurocesar.com}, 	% producer of the document
	    pdfcreator={LaTeX with abnTeX2},
    	colorlinks=true,
    	linkcolor=blue,
    	citecolor=blue,
		urlcolor=blue
}

\EnableCrossrefs
\CodelineIndex
\RecordChanges

\changes{v1.0}{2013/02/01}{Versão inicial}

\begin{document}


\maketitle

\begin{abstract}
Este é o manual de uso da classe \textsf{abntex2}. Trata-se de um conjunto de
customizações da classe \textsf{memoir} para elaboração de documentos técnicos e
científicos condizentes com as normas da Associação Brasileira de Normas
Técnicas, especialmente a ABNT NBR 6022:2003, ABNT NBR 10719:2011, ABNT NBR
14724:2011 e a ABNT NBR 6024:2012, além de outras referentes a ``Informação e
documentação''.
\end{abstract}

\tableofcontents

\listoftables

% ------
\section{Escopo}
% ------

O objetivo deste manual é descrever a classe \textsf{abntex2}, responsável pelo
\textit{layout} dos elementos de estruturação de documentos técnicos e
científicos --- como trabalhos acadêmicos, artigos e relatórios técnicos ---,
especialmente aqueles definidos pela ABNT NBR 6022:2003, ABNT NBR 10719:2011,
ABNT NBR 14724:2011 e pela ABNT NBR 6024:2012. A \autoref{sec-introducao} contém
a lista completa de normas observadas pela classe.

Pelo fato de a norma ABNT NBR 14724:2011 e a ABNT NBR 6024:2012 englobarem os
requisitos das demais normas, estas são utilizadas como base de referência para
este manual. De todo modo, a \autoref{sec-estrutura} descreve as diferentes e
similaridades entre elas e em que medida a classe \textsf{abntex2} atende a todas as normas
listadas no parágrafo anterior.

Este manual faz parte da suíte \abnTeX2. Para referências ao pacote de citações
\textsf{abntex2cite}, consulte \citeonline{abntex2cite} e
\citeonline{abntex2cite-alf}.

Exemplos de uso da classe descrita neste manual podem ser consultados em
\citeonline{abntex2modelo,abntex2modelo-artigo,abntex2modelo-relatorio,abntex2modelo-glossario}.
Os documentos indicados tratam-se de ``Modelos Canônicos'', ou seja, de modelos
que não são específicos a nenhuma universidade ou instituição, mas que
implementam exclusivamente os requisitos das normas da ABNT, Associação
Brasileira de Normas Técnicas.
 
% ------
\section{Considerações iniciais}
% ------

\subsection{Introdução}\label{sec-introducao}

Dentre as características de qualidade de trabalhos acadêmicos (teses,
dissertações e outros do gênero), de artigos científicos e de relatórios
técnicos, ao lado da pertinência do tema e dos aspectos relativos ao conteúdo
abordado no trabalho, consta também o resultado da editoração final e as
características de forma e de estruturação dos documentos.
Desse modo, a existência de um modelo e de ferramentas que atendam às normas
brasileiras de elaboração de trabalhos acadêmicos, artigos científicos e
relatórios técnicos propostas pela Associação Brasileira de Normas Técnicas
(ABNT) são recursos básicos para o aprimoramento da qualidade geral dos
trabalhos acadêmicos nacionais.

É com esse intuito que o \abnTeX2 é apresentado à comunidade acadêmica
brasileira: o de ser um instrumento de aperfeiçoamento da qualidade dos
textos produzidos. O \abnTeX2 surge para se somar ao já vasto universo de
ferramentas \LaTeX, porém que é escasso em utilitários específicos para
trabalhos brasileiros. Nesse sentido, \abnTeX2 é proposto como uma evolução do
\abnTeX1 original\footnote{Ver \autoref{sec-historico},
\nameref{sec-historico}.} e como facilitador na elaboração de documentos
técnicos e científicos condizentes com as normas ABNT.

O \abnTeX2 foi desenvolvido com base nos requisitos das seguintes normas ABNT:

\begin{description}
  \item[ABNT NBR 6022:2003] Informação e documentação - Artigo em publicação
  periódica científica impressa - Apresentação
  \item[ABNT NBR 6023:2002] Informação e documentação - Referência -
  Elaboração\footnote{ O abnTeX2 é compatível com a versão corrente da norma
  ABNT NBR 6023:2002. Porém, até este momento, os manuais
  \citeonline{abntex2cite} e \citeonline{abntex2cite-alf} estão atualizados com
  informações referentes à versão anterior da referida norma, ou seja, com
  informações sobre a versão 2000. A atualização do manual é uma das atividades
  em execução do projeto. Consulte
  \url{https://code.google.com/p/abntex2/issues/list} para outras informações.}
  \item[ABNT NBR 6024:2012] Informação e documentação - Numeração
  progressiva das seções de um documento - Apresentação
  \item[ABNT NBR 6027:2012] Informação e documentação - Sumário -
  Apresentação
  \item[ABNT NBR 6028:2003] Informação e documentação - Resumo -
  Apresentação
  \item[ABNT NBR 6034:2004] Informação e documentação - Índice -
  Apresentação
  \item[ABNT NBR 10520:2002] Informação e documentação - Citações
  \item[ABNT NBR 10719-2011] Informação e documentação - Relatório técnico
  e-ou científico - Apresentação
  \item[ABNT NBR 14724:2011] Informação e documentação - Trabalhos
  acadêmicos - Apresentação
  \item[ABNT NBR 15287:2011] Informação e documentação - Projeto de pesquisa -
  Apresentação
\end{description}

Este manual de uso não foi escrito sob o modelo do \abnTeX2, nem segue os
padrões da ABNT.

Este texto deve ser utilizado como complemento do manual do
\textsf{memoir}\footnote{\url{http://www.ctan.org/tex-archive/macros/latex/contrib/memoir/}}
\cite{memoir}.

\subsection{Histórico do projeto}\label{sec-historico}

Entre 2001 e 2002 nascia o projeto \abnTeX~original, cujo objetivo era o de
``prover um conjunto de macros para \LaTeX\ para formação de trabalhos
acadêmicos condizentes com as normas ABNT''. A última versão estável publicada pelos
integrantes originais\footnote{Os integrantes originais do projeto são Miguel
Frasson, Gerald Weber, Leslie H. Watter, Bruno Parente Lima, Flávio de
Vasconcellos Corrêa, Otavio Real Salvador e Renato Machnievscz.} é a versão
0.8.2 e data de 3.11.2004 (hospedada em
\url{http://abntex.codigolivre.org.br/}). Em 2006 uma versão não estável foi
publicada para testes, mas nunca foi evoluída.

Em maio de 2009 Leandro Salvador fez uma chamada no fórum Tex-BR\footnote{A
mensagem pode ser lida neste link:
\url{http://marc.info/?l=tex-br\&m=124110883528962}} clamando por voluntários
para a retomada do projeto. Embora tenha sido criado um novo repositório para o
projeto\footnote{O projeto de Salvador está hospedado em
\url{https://sourceforge.net/projects/abntex/}.}, nenhuma nova contribuição ao
código do \abnTeX~foi realizada desde 2006 até dezembro de 2012. No novo
endereço há discussões um pouco mais recentes sobre o projeto, porém datam de
2009. Há, inclusive, uma sugestão de Gerald Weber, um dos criadores originais:

\begin{quote}
``A sugestão que eu daria seria substituir a classe que o Miguel Frasson
desenvolveu por pacotes padrão do \LaTeX. Há pacotes que implementam
praticamente tudo que a ABNT requer, basta mexer nas opções. Assim teria algo
muito mais simples de manter e atualizar no longo prazo.''
(\url{https://sourceforge.net/projects/abntex/forums/forum/947854/topic/3265973})
\end{quote}

No início de 2012 Lauro César Araujo havia concluído sua dissertação de mestrado
escrita em \LaTeX\ com o \abnTeX\ original. Nessa época ele foi convidado pelo
diretor do Centro de Pesquisa em Arquitetura da Informação
(CPAI/UnB)\footnote{\url{http://www.cpai.unb.br}.}, Mamede Lima-Marques, para
realizar um \textit{workshop} de \LaTeX\ no âmbito do Terceiro Colóquio de
Arquitetura da Informação promovido pelo CPAI. Durante as apresentações, os
problemas com instalação e distribuição do \abnTeX\ original e as falhas
normativas em relação às normas vigentes da ABNT -- uma vez que tinham se
passado quase 10 anos desde a criação do \abnTeX -- foram motivadores adicionais
para que o novo projeto \abnTeX2 se tornasse urgente. Então, coordenado por
Lauro César, o novo projeto, foi iniciado oficialmente em maio de 2012, com o
apoio dos integrantes do grupo \abnTeXForum. A ideia de Gerald e o
software já construído do \abnTeX\ original serviriam de ponto de partida para
o \abnTeX2.

A primeira versão do \abnTeX2 foi concluída em dezembro de 2012 e
disponibilizada ao público em janeiro de 2013. O portal do projeto atual é
\abnTeXSite. O \abnTeX2 foi disponibilizado ao CTAN, \emph{The Comprehensive TEX
Archive Network}, em 2 de janeiro de 2013 e pode ser consultado em
\url{http://www.ctan.org/pkg/abntex2}. As principais distribuições
\LaTeX\footnote{Como TeX Live, MiKTeX, MacTeX e proTeXt} são construídas à
partir de pacotes e classes do CTAN.

O software é mantido desde então pela comunidade de indivíduos e de
organizações que adotam e/ou investem em software livre.

% O novo projeto utiliza o menor conjunto possível de classes nativas do \LaTeX~de
% modo a implementar as exigências das normas ABNT. Para isso, escolheu-se como
% base a classe \textsf{memoir}, por ser uma classe flexível, com excelente
% documentação e que constantemente recebe atualizações de novas funcionalidades e
% correção de eventuais problemas. Este documento descreve o uso desta nova
% classe.

\subsection{O que o \abnTeX2 traz de novo em relação ao \abnTeX1?}

A suíte \abnTeX2 é composta por quatro elementos principais\footnote{Somam-se a
esses elementos a documentação da classe e do pacote citação.}:

\begin{itemize}
  \item a classe de formação de documentos técnicos e científicos
  \textsf{abntex2}, descrita neste manual;
  \item o pacote de citações bibliográficas \textsf{abntex2cite}; e
  \item as especificações de formatação de referências bibliográficas
  |abntex2-num.bst| e |abntex2-alf.sty|;
  \item os modelos canônicos de uso do \abnTeX2;
\cite{abntex2modelo,abntex2modelo-artigo,abntex2modelo-relatorio,abntex2modelo-projeto-pesquisa}
\end{itemize}

Esta versão do \abnTeX2 fornece uma classe para produção de um vasto número de
documentos técnicos e científicos, como trabalhos acadêmicos, artigos científicos,
relatórios técnicos, projetos de pesquisa e outros do gênero. A classe utilizada
por este projeto é completamente nova em relação à utilizada pelo projeto
anterior, que tinha a intenção de produzir um número menor de documentos.
A classe anterior, chamada \textsf{abnt}, não está mais disponível e não é
distribuída nesta versão.

Já o pacote de citações bibliográficas e as especificações de formatação de
referências bibliográficas são uma evolução da versão anterior. Novas
funcionalidades foram incluídas, como a possibilidade de uso do pacote
\textsf{backref} (\autoref{sec-referencias}), compatibilidade com o
\textsf{beamer}\footnote{\url{http://www.ctan.org/pkg/beamer}}, entre outros
aperfeiçoamentos e ajustes de conformidade com a versão 2002 da ABNT NBR 6023.
Para referência ao pacote de citações \textsf{abntex2cite}, inclusive sobre
conformidade com as normas, consulte \citeonline{abntex2cite} e
\citeonline{abntex2cite-alf}. O abnTeX2 também traz suporte nativo para produção
de documentos em diferentes idiomas, como inglês por exemplo (\autoref{sec-internacional}).

Os modelos canônicos não estavam presentes no abnTeX original. Eles são exemplos
de uso do \abnTeX2 e são distribuídos junto com a instalação padrão. 

\subsection{Compatibilidade entre as versões 1 e 2 do \abnTeX}

As duas versões do \abnTeX\ são compatíveis entre si, ou seja, não há
incompatibilidade mútua entre elas de tal modo que você pode ter as duas versões
do \abnTeX\ instaladas simultaneamente no mesmo computador. Desse modo,
documentos escritos com a versão anterior do \abnTeX\ continuarão a serem
compilados com a classe e os pacotes anteriores. Apenas documentos escritos
conforme este manual utilizarão a nova classe e os novos pacotes de citação e
formatação de referências bibliográficas.

Com base neste manual você provavelmente não terá dificuldades em converter os
trabalhos escritos em \abnTeX1 para o novo \abnTeX2.

\subsection{Licença de uso e customizações para universidades e outras
instituições}

Sinta-se convidado a participar do projeto \abnTeX2! Acesse o site do projeto em
\abnTeXSite. Também fique livre para conhecer, estudar, alterar e redistribuir o
trabalho do \abnTeX2, desde que os arquivos modificados tenham seus nomes
alterados, até mesmo no seu computador, e que os créditos sejam dados aos
autores, nos termos da ``The \LaTeX Project Public
License''\footnote{\url{http://www.latex-project.org/lppl.txt}}.

Encorajamos que customizações específicas para universidades sejam realizadas
--- como capas, folha de aprovação, etc. Porém, recomendamos que ao invés de se
alterar diretamente os arquivos do \abnTeX2, distribua-se arquivos com as
respectivas customizações. Isso permite que futuras versões do \abnTeX2 não se
tornem automaticamente incompatíveis com as customizações promovidas.

% ------
\section{Estrutura geral de documentos técnicos e científicos brasileiros}\label{sec-estrutura}
% ------

A seção 4 da ABNT NBR 14724:2011 estabelece que a estrutura de trabalhos
acadêmicos compreende duas partes: a externa e a interna, que aparecem no texto
na ordem que segue. Essa norma pode ser considerada uma norma geral de estrutura
de documentos técnicos e científicos porque compartilha com as demais uma
estrutura comum. Os demais parágrafos desta seção descrevem as semelhanças e
diferenças entre as outras normas.

\begin{description}
  \item[Parte externa]\ \\ 
       Capa (obrigatório) \\
       Lombada (opcional) 
  \item[Parte Interna]\ 
		\begin{description} 
		\item[Elementos pré-textuais]\ \\
		  Folha de rosto (obrigatório) + \\
		  Dados de catalogação-na-publicação\footnote{O documento
		  ``Dados de catalogação-na-publicação'' é chamado apenas como ``Ficha
		  catalográfica'' neste texto.} (opcional)\\
		  Errata (opcional) \\
		  Folha de aprovação (obrigatório) \\
		  Dedicatória (opcional) \\
		  Agradecimentos (opcional) \\
		  Epígrafe (opcional) \\
		  Resumo em língua vernácula (obrigatório) \\
		  Resumo em língua estrangeira (obrigatório) \\
		  Lista de ilustrações (opcional) \\
		  Lista de tabelas (opcional) \\
		  Lista de abreviaturas e siglas (opcional) \\
		  Lista de símbolos (opcional) \\ 
		  Sumário (obrigatório) 
		\item[Elementos textuais]\footnote{A nomenclatura dos títulos dos elementos
		textuais é a critério do autor.}\
		\\
		  Introdução \\
		  Desenvolvimento \\
		  Conclusão 
		\item[Elementos pós-textuais]\ \\
		  Referências (obrigatório) \\
		  Glossário (opcional) \\
		  Apêndice (opcional) \\
		  Anexo (opcional) \\
		  Índice (opcional) 
		\end{description}
\end{description} 

O \abnTeX2 apresenta instrumentos para produzir todas as partes do documento,
exceto a Lombada e o Glossário, que podem ser produzidos por outros pacotes
adicionais. As seções seguintes descrevem como cada seção pode ser produzida.

A norma ABNT NBR 10719:2011 \emph{Informação e documentação - Relatório técnico
e-ou científico - Apresentação} apresenta basicamente a mesma estrutura que a
norma de trabalhos acadêmicos, exceto que nesta norma a Capa é opcional, não há
Folha de aprovação, Dedicatória, Epígrafe nem Resumo em língua estrangeira e
acrescenta-se opcionalmente um formulário de identificação como último elemento
pós-textual. Um modelo desse formulário está incluído no modelo
\cite{abntex2modelo-relatorio}. Todos os demais elementos aparecem na mesma
ordem e são regidos pelas mesmas regras nas duas normas.

A norma ABNT NBR 15287:2011 \emph{Informação e documentação - Projeto de
pesquisa - Apresentação} também apresenta a mesma estrutura básica da ABNT NBR
14724:2011. Porém, na ABNT NBR 15287:2011 a Capa é opcional, não há Errata,
Folha de aprovação, Dedicatória, Agradecimentos, Epígrafe nem Resumos. Todos os
demais elementos aparecem na mesma ordem e, a exemplo da ABNT NBR 10719:2011,
são regidos pelas mesmas regras.

Dessa forma, a classe \textsf{abntex2} pode ser utilizada para gerar os
documentos previstos tanto na ABNT NBR 10719:2011 quanto na ABNT NBR 15287:2011.

Adicionalmente, a classe \textsf{abntex2} também é compatível com a norma ABNT
NBR 6022:2003 \emph{Informação e documentação - Artigo em publicação periódica
científica impressa - Apresentação}. Porém, nesse caso, a estrutura apresentada
nesta seção não é aplicável. O modelo \cite{abntex2modelo-artigo} é um exemplo
de artigo científico elaborado com a classe \textsf{abntex2}. 

% ------
\section{Configurações gerais}\label{sec-configgerais}
% ------

% ---
\subsection{Codificação dos arquivos: UTF8}
\label{sec-utf8}
% ---

A codificação de todos os arquivos do \abnTeX2 é \texttt{UTF8}. É necessário que
você utilize a mesma codificação nos documentos que escrever, inclusive nos
arquivos de base bibliográficas |.bib|.

No preâmbulo do seu documento você geralmente usará os pacotes
\textsf{inputenc}\footnote{O pacote \textsf{inputenc} é usado para que seja
possível escrever textos acentuados em determinado padrão de codificação. No
caso, \abnTeX2 utiliza a codificação UTF8. Consulte detalhes do pacote em
\url{http://www.ctan.org/pkg/inputenc}. } (com a opção |[utf8]{inputenc}|) e
\textsf{fontenc}\footnote{O pacote \textsf{fontenc} controla a codificação das
fontes usadas para impressão do documento. Consulte detalhes do pacote em
\url{http://www.ctan.org/pkg/fontenc}.} (com a opção |[T1]{fontenc}|) sempre que
|xetex| ou |xelatex| não são usados para compilar o documento, ou seja, os
pacotes geralmente devem ser incorporados ao seu documento quando se utiliza
|pdflatex|, por exemplo. Porém, a classe \textsf{abntex2} não incorpora os
pacotes automaticamente.

% A classe \textsf{abntex2.cls} incorpora automaticamente os pacotes
% \textsf{inputenc}\footnote{O pacote \textsf{inputenc} é usado para que seja
% possível escrever textos acentuados em determinado padrão de codificação. No
% caso, \abnTeX2 utiliza a codificação UTF8. Consulte detalhes do pacote em
% \url{http://www.ctan.org/pkg/inputenc}. } (com a opção |[utf8]{inputenc}|) e
% \textsf{fontenc}\footnote{O pacote \textsf{fontenc} controla a codificação das
% fontes usadas para impressão do documento. Consulte detalhes do pacote em
% \url{http://www.ctan.org/pkg/fontenc}.} (com a opção |[T1]{fontenc}|) sempre que
% |xetex| ou |xelatex| não são usados para compilar o documento, ou seja, os
% pacotes são incorporados automaticamente quando se utiliza |pdflatex|, por
% exemplo. Isso significa que você não precisa incorporar os pacotes nos seus
% modelos, uma vez que já são incluídos automaticamente.

Veja detalhes sobre fontes com Xe\LaTeX\ na \autoref{sec-xelatex}. 

% ---
\subsection{A classe \textsf{abntex2} como extensão de \textsf{memoir}}
% ---

\DescribeMacro{\documentclass}
A classe \textsf{abntex2} foi criada como um conjunto de configurações da classe
\textsf{memoir}\footnote{A versão anterior do \abnTeX~era baseada na classe
\textsf{report}.} \cite{memoir}. Desse modo, todas as opções do \textsf{memoir}
estão disponíveis, como por exemplo, |12pt,openright,twoside,a4paper,article|.
Consulte o manual do \textsf{memoir} para outras opções.

As opções mais comuns de inicialização do texto do documento são:

\begin{verbatim}
\documentclass[12pt,openright,twoside,a4paper,brazil]{abntex2}
\end{verbatim}

% ---
\subsubsection{O tamanho do papel}
% ---

O tamanho do papel pode ser alterado modificando a opção |a4paper| para
|a5paper|, por exemplo. Porém, o tamanho definido pela ABNT NBR 14724:2011 é A4.
A lista completa de opções disponíveis pode ser consultada em
\citeonline[p.~1]{memoir}.

% ---
\subsubsection{Impressão em anverso e verso}
% ---

É interessante observar que a ABNT NBR 14724:2011 (seção 5.1) recomenda que os
documentos sejam impressos no anverso e no verso das folhas. Isso é obtido com a
opção |twoside|. 

% ---
\subsubsection{Opção \texttt{article}}
% ---

\DescribeMacro{article}\DescribeMacro{\counterwithout}
A opção |article| é útil para produção de artigos com \abnTeX2.
Nesse caso, a maioria dos elementos pré-textuais descritos na
\autoref{sec-pretextuais} se tornam desnecessários. Quando esta opção for
utilizada, a classe \textsf{abntex2} não forçará quebra de página para os
elementos pré-textuais e definirá a formatação do capítulo de forma idêntica à
formatação das seções. Por padrão, quando a opção |article| estiver presente,
você deve iniciar as divisões do documento com |\section|, e não |\chapter|,
como é usual em trabalhos monográficos. Porém, caso queira iniciar as divisões com
|\chapter| ao invés de |\section|, adicione as linhas abaixo no preâmbulo do
documento para que a numeração dos capítulos, seções, figuras e tabelas sejam
corretamente sequenciados:

\begin{verbatim}
\counterwithout{section}{section}
\counterwithout{figure}{chapter}
\counterwithout{table}{chapter}
\end{verbatim}

A macro |\part| também é permitida em |article|.

% ---
\subsubsection{As opções de tamanho de fonte e o tamanho ``menor e uniforme''}
% ---

\DescribeMacro{\ABNTEXfontereduzida}\DescribeMacro{\footnotesize}\DescribeEnv{12pt}
A seção 5.1 da ABNT NBR 14724:2011 também estabelece que o tamanho fonte seja 12
para todo o documento (obtida com a opção |12pt|), ``inclusive capa,
excetuando-se citações com mais de três linhas, notas de rodapé, paginação,
dados internacionais de catalogação-na-publicação, legendas e fontes das
ilustrações e das tabelas, que devem ser em tamanho menor e uniforme''.  O
tamanho ``menor e uniforme'' é estabelecido pela macro |\ABNTEXfontereduzida| e
o valor padrão utilizado é o mesmo da macro |\footnotesize|. Você pode alterar o
valor de |\ABNTEXfontereduzida| para |\small|, por exemplo, com o seguinte
comando:

\begin{verbatim}
\renewcommand{\footnotesize}{\small}
\end{verbatim}

Caso deseje utilizar outro tamanho de fonte para o documento, substitua a opção
|12pt| pelo tamanho desejado, como por exemplo, |10pt|, ou |14pt|. A lista
completa de opções disponíveis pode ser consultada em
\citeonline[p.~2-3]{memoir}.

% ---
\subsubsection{Opções específicas da classe \textsf{abntex2}: títulos de
divisões em letras maiúsculas}
% ---

\begin{description}
\item {\texttt{chapter=TITLE}}

Altera para caixa alta (letras maiúsculas) os \emph{títulos} dos
capítulos e o títulos de todos os elementos pré e pós textuais escritos com o
mesmo nível que capítulos.

\item {\texttt{section=TITLE}, \texttt{section=TITLE},
\texttt{subsubsection=TITLE}}

Altera para caixa alta (letras maiúsculas) os \emph{títulos} das seções,
subseções, subsubseções, respectivamente.
\end{description}

Essas opções \emph{não controlam o formato dos items} no sumário, lista de
ilustrações, tabelas, etc., \emph{nem controlam o formato dos cabeçalhos} de
páginas que incluem informações de capítulos ou de secionamento.

Exemplo de uso:

\begin{verbatim}
\documentclass[12pt,openright,twoside,a4paper,
	chapter=TITLE,		% títulos de capítulos convertidos em letras maiúsculas
	section=TITLE,		% títulos de seções convertidos em letras maiúsculas
	subsection=TITLE,	% títulos de subseções convertidos em letras maiúsculas
	subsubsection=TITLE, % títulos de subsubseções em letras maiúsculas
	english,french,spanish,brazil]{abntex2}
\end{verbatim}

% ---
\subsection{Espaçamentos entre parágrafos e linhas}
% ---

\subsubsection{Tamanho do parágrafo}

O tamanho do parágrafo, espaço entre a margem e o início da frase do parágrafo,
é definido por:

\begin{verbatim}
\setlength{\parindent}{1.3cm}
\end{verbatim}

\subsubsection{Tamanho do primeiro parágrafo}

Por padrão, não há espaçamento no primeiro parágrafo de cada início de divisão
do documento. Porém, você pode definir que o primeiro
parágrafo também seja indentado. Para isso, apenas inclua o pacote
\textsf{indentfirst} no preâmbulo do documento:

\begin{verbatim}
\usepackage{indentfirst}	% Indenta o primeiro parágrafo de cada seção.
\end{verbatim}

\subsubsection{Espaçamento entre parágrafos}

O espaçamento entre um parágrafo e outro pode ser controlado por meio do
comando:

\begin{verbatim}
\setlength{\parskip}{0.2cm}	% tente também \onelineskip
\end{verbatim}

\subsubsection{Espaçamento entre linhas}

O espaçamento entre linhas padrão é definido como |\OnehalfSpacing|, ou seja, um
espaço e meio, conforme estabelece a ABNT NBR 14724:2011. De todo modo, os
comando |\SingleSpacing|, |\DoubleSpacing| podem ser utilizados para obter
espaçamento simples e espaçamento duplo, respectivamente. Além dessas
macros, estão disponíveis:

\begin{verbatim}
\begin{SingleSpace} ...\end{SingleSpace}
\begin{Spacing}{hfactori} ... \end{Spacing}
\begin{OnehalfSpace} ... \end{OnehalfSpace}
\begin{OnehalfSpace*} ... \end{OnehalfSpace*}
\begin{DoubleSpace} ... \end{DoubleSpace}
\begin{DoubleSpace*} ... \end{DoubleSpace*} 
\end{verbatim}

Observe que a classe \textsf{abntex2} utiliza o sistema de espaçamento padrão
do \textsf{memoir}. Nesse caso, o pacote \textsf{setspace} não é necessário.

Para mais informações, consulte \citeonline[p. 47-52 e 135]{memoir}.

% ---
\subsection{Margens}
% ---

As \emph{margens} são configuradas conforme a NBR 14724:2011, seção 5.1, 
e podem ser alteradas do seguinte modo:

\begin{verbatim}
\setlrmarginsandblock{3cm}{2cm}{*}
\setulmarginsandblock{3cm}{2cm}{*}
\checkandfixthelayout
\end{verbatim}

% ---
\subsection{Numeração contínua de figuras e tabelas}
% ---
  
A numeração de figuras e tabelas deve ser contínua em todo o documento (ABNT
NBR 14724:2011 seções 5.8 e 5.9). Porém, caso deseje alterar esse comportamento
para numeração por capítulos, por exemplo, use:
\begin{verbatim}
\counterwithout{figure}{section}
\counterwithout{table}{section}
\end{verbatim}


% ---
\subsection{Índice do PDF com pacote \textsf{bookmark}}\label{sec-bookmark}
% ---

O índice da estrutura do documento é automaticamente inserido no PDF final do
documento por meio do pacote
\textsf{bookmark}\footnote{\url{http://www.ctan.org/pkg/bookmark}}. Neste
documento este índice será identificado como ``\textsf{bookmark} do PDF''.

Com exceção da Ficha catalográfica (\autoref{sec-fichacatalografica}), todos os
elementos pré-textuais descritos na \autoref{sec-pretextuais} e as divisões dos
documentos, como |\part|, |\chapter|, |\section|, etc., são automaticamente
inseridos tanto no Sumário (\autoref{sec-sumario}) quanto no \textsf{bookmark}
do PDF.

\DescribeMacro{\pdfbookmark}
A versão * dos comandos, como |\part*| e |\chapter*|, por exemplo, não inclui a
divisão no Sumário, nem no \textsf{bookmark} do PDF nem altera o cabeçalho da
página no caso de capítulos. Porém, você pode explicitamente incluir as divisões
no \textsf{bookmark} com o comando |\pdfbookmark|\oarg{posição}\marg{Título no
bookmark}\marg{texto de identificação única, sem espaços}:

\begin{verbatim}
\pdfbookmark[0]{Capítulo fora do Sumário, mas presente no
 bookmark}{texto-qualquer} 
\chapter*{Capítulo fora do Sumário, mas presente no bookmark}
\end{verbatim}

Para alterar o cabeçalho da página automaticamente com o comando
|\chapter*|, consulte a \autoref{sec-formatacaocapitulos-starred}.

Para inserir uma divisão com * no sumário, consulte \autoref{sec-sumario}.

\DescribeMacro{\phantomsection}
A macro |\phantomsection| pode ser útil imediatamente antes de |\pdfbookmark|
quando o texto adicionado ao bookmark não estiver próxima a uma divisão do
documento. Nesse caso, o comando fica assim:

\begin{verbatim}
\phantomsection\pdfbookmark[0]{Capítulo}{texto-qualquer2} 
\chapter*{Capítulo}
\end{verbatim}

Veja a sugestão de uso do \textsf{bookmark} do PDF na
\autoref{sec-listadeilustracoes}, \autoref{sec-listadetabelas},
\autoref{sec-listadeabreviaturas} e na \autoref{sec-sumario}.

\DescribeMacro{\pretextualchapter}
A macro |\pretextualchapter|\marg{título do capítulo} pode ser utilizada para
criar capítulos sem numeração, que não aparecem no Sumário, mas que são
automaticamente adicionados ao \textsf{bookmark} do PDF e alteram o cabeçalho
da página. Consulte a \autoref{sec-formatacaocapitulos} para mais detalhes.

Informações adicionais sobre configuração dos \textsf{bookmarks} podem ser
obtidas em \citeonline{oberdiek2011}.

% ---
\subsection{Fontes de texto}
% ---

A ABNT NBR 14724:2011 não determina o uso de alguma fonte específica. Apenas o
tamanho, que deve ser 12pt, é estabelecido. 

Se nenhum pacote de fonte for encontrado, o \abnTeX2 utiliza a fonte padrão
do \LaTeX, que é \emph{Computer
Moderns}\footnote{\url{http://www.tug.dk/FontCatalogue/cmr/}}. Para os títulos
das divisões, o \abnTeX2 utiliza uma versão sem serifa da fonte. Consulte a
\autoref{sec-formatacaocapitulos} para mais informações sobre a formatação das divisões.

Para  escolher uma fonte compilando o documento com \texttt{pdflatex}, utilize
um dos pacotes de fontes nativos do \LaTeX: por exemplo,
\verb+\usepackage{fourier}+, para Adobe Utopia; ou, como nos
documentos de exemplo do \abnTeX2, use \verb+\usepackage{lmodern}+ para 
a \emph{Latin Modern}, que é uma versão
aprimorada da \emph{Computer Modern}\footnote{Veja mais informações em
\url{http://www.tug.dk/FontCatalogue/lmodern/}.}. Há dezenas de outros pacotes
de fontes. Muitos deles estão disponíveis na maioria das distribuições
\LaTeX\footnote{Você pode encontrá-las em
\url{http://www.tug.dk/FontCatalogue/}. Há muitas fontes interessantes, boas e
legíveis, e todas elas podem ser utilizadas compilando o documento com o
comando: \texttt{pdflatex (nome do documento).tex} (ou direto no seu editor de
texto favorito). Por exemplo, temos as fontes: Charter, Palatino, Utopia,
Century\ldots}.
 
%Para selecionar uma fonte que não é a padrão (Computer Modern), você precisa
% incluir um pacote de fontes nativo ou usar o pacote \texttt{fontspec} e
% compilar com o comando \texttt{xelatex}.
%
%\begin{verbatim}
% \usepackage[T1]{fontenc}
% \usepackage{times}
%\end{verbatim}
% 
% No caso, \texttt{times} é o pacote da fonte (Times, igual à Times New Roman). 

\subsubsection{Fontes com Xe\LaTeX}
\label{sec-xelatex}

Para utilizar as fontes de tipo \textsf{.ttf} ou \textsf{.otf} --- como as
presentes em seu sistema operacional ou utilizadas por outros programas ---, o
arquivo deve estar em codificação \textsf{UTF8} e ser compilado com o comando
\texttt{xelatex (nomedoarquivo.tex)}. Xe\TeX{} é um programa de diagramação
derivado do \TeX{} que utiliza
\href{http://pt.wikipedia.org/wiki/Unicode}{Unicode} e possibilita o emprego de
fontes tipográficas modernas, como OpenType e AAT (Apple Advanced Typography).
A classe \textsf{abntex2} implementa opções para Xe\LaTeX{} por meio dos pacotes
\textsf{fontspec}\footnote{\url{http://ctan.org/pkg/fontspec}} e
\textsf{polyglossia}\footnote{\url{http://ctan.org/pkg/polyglossia}}. 

Quando um documento é compilado com \texttt{xelatex}, os pacotes
\textsf{inputenc} e \textsf{fontenc}, descritos na \autoref{sec-utf8},
geralmente não devem ser incluídos ao preâmbulo do documento. Ao invés desses
pacotes, geralmente \textsf{fontspec} é usado. O seguinte exemplo de preâmbulo
torna flexível a compilação do documento, que pode tanto ser realizada da forma
tradicional com \texttt{pdflatex} quanto com \texttt{xelatex}, uma vez que
inclui seletivamente os pacotes adequados para cada compilador:

\begin{verbatim}
\usepackage{ifxetex}
 \ifxetex
   \usepackage{fontspec}
   \defaultfontfeatures{Ligatures={TeX}}
  \else
   \usepackage[utf8]{inputenc}
   \usepackage[T1]{fontenc}
  \fi
\end{verbatim}

% Sempre que um documento construído com a classe \textsf{abntex2} é compilado com
% \texttt{xelatex}, os pacotes \textsf{inputenc} e \textsf{fontenc} não são
% incluídos automaticamente. Veja a \autoref{sec-utf8} para outras informações a
% respeito da codificação padrão dos documentos.

Para fonte serifada, sem serifa (geralmente usada para títulos) e monoespaçada,
respectivamente:
 
\begin{verbatim}
\setromanfont{Minion Pro}
\setsansfont{Myriad Pro}
\setmonofont[Scale=MatchLowercase]{Consolas}
\end{verbatim} 
 
O usuário deve implementar esses comandos, substituindo o nome das fontes
pelas que preferir. É importante destacar que, caso as fontes não sejam
embutidas no PDF\footnote{As fontes devem ser embutidas do PDF quando, 
por exemplo, deseja-se produzir documentos em conformidade com o padrão PDF/A
normatizado pela série ISO 19005. Para geração de PDF/A em \LaTeX\, 
consulte 
\url{http://support.river-valley.com/wiki/index.php?title=Generating_PDF/A_compliant_PDFs_from_pdftex}.},
 o software que lerá o PDF deverá ter disponível a fonte
utilizada. Com o Adobe Acrobat Reader, por exemplo, as fontes gratuitas
seguintes já estão disponíveis:
Minion Pro e Myriad Pro\footnote{Para Linux, faça o download de:
\url{http://www.adobe.com/support/downloads/detail.jsp?ftpID=4426}, extraia e
instale. Para Windows, vá até a pasta onde instalou o Reader (Arquivos de
Programas), procure as fontes e instale.}. Outra fonte de alta qualidade é
Gentium\footnote{\url{http://scripts.sil.org/cms/scripts/page.php?site\_id=nrsi\&id=Gentium\_download}.}.
 
Para textos em outras línguas, deverão ser utilizados ambientes do pacote
\textsf{polyglossia}. Por exemplo:
 
\begin{verbatim}
\begin{english}
 Text in English...
\end{english} 
\end{verbatim}
 
Para mais informações, consulte \url{http://www.xelatex.org/}.

% ---
\subsection{Internacionalização}
\label{sec-internacional}
% ---

%---
\subsubsection{Hifenização e diferentes idiomas}
\label{sec-internacional-hifen}

Para usar as diferentes hifenizações de cada idioma, inclua nas opções do
documento o nome dos idiomas usados no texto. Por exemplo:

\begin{verbatim}
\documentclass[12pt,openright,twoside,a4paper,english,french,
spanish,brazil]{abntex2}
\end{verbatim}

O idioma português-brasileiro (\texttt{brazil}) é incluído automaticamente pela
classe \textsf{abntex2}. Porém, mesmo assim a opção \texttt{brazil} deve ser
informada como a última opção da classe para que todos os pacotes reconheçam o
idioma. Vale ressaltar que a última opção de idioma é a utilizada por padrão no
documento. Desse modo, caso deseje escrever um texto em inglês que tenha
citações em português e em francês, você deveria usar o preâmbulo como abaixo:

\begin{verbatim}
\documentclass[12pt,openright,twoside,a4paper,french,brazil,english]{abntex2}
\end{verbatim}

A lista completa de idiomas suportados, bem como outras opções de hifenização,
estão disponíveis na documentação do pacote
\textsf{babel}\footnote{\url{http://www.ctan.org/tex-archive/macros/latex/required/babel/}}
\citeonline[p.~5-6]{babel}.

\DescribeMacro{\foreignlanguage}
A macro |\foreignlanguage|\marg{idioma}\marg{texto a ser inserido} insere um
texto com a hifenização do idioma informado em \marg{idioma}.

\DescribeEnv{otherlanguage}\DescribeEnv{otherlanguage*}
O ambiente |otherlanguage*|\marg{idioma} pode ser usado para o mesmo propósito
da macro |\foreignlanguage|. A versão com * indica que apenas o esquema de hifenização do
idioma selecionado deve ser utilizado. Nesse caso, as demais configurações de
idiomas, como nome da Lista de figuras, rótulos das seções, entre outros, não
são alterados. Exemplo de uso:

\begin{verbatim}
\begin{otherlanguage*}{french}
 Texte en français.
\end{otherlanguage*}
\end{verbatim}

\DescribeMacro{\selectlanguage}
A macro |\selectlanguage|\marg{idioma} altera o idioma padrão do documento a
partir do ponto em que é informado.

Consulte a \autoref{sec-interncional-traducao} para obter detalhes sobre o uso
de diferentes idiomas no documento.

A \autoref{sec-citacao} descreve o ambiente |citacao|\oarg{language} que pode
receber como parâmetro um idioma a ser usado na citação.

%---
\subsubsection{Alterações de nomes padronizados por \textsf{babel}}
\label{sec-internacional-babel-alteracoes}

Algumas alterações foram realizadas nos nomes padronizados pelo pacote
\textsf{babel} para atender a requisitos da \citeonline{NBR14724:2011} e da
\citeonline{NBR6024:2012}, conforme \autoref{tab-babel}.

\begin{table}[htb]
\caption{Alterações no pacote \textsf{babel}}
\label{tab-babel}
\centering
\begin{tabular}{ l l l }
   \textbf{Macro} & \textbf{Valor original} & \textbf{Novo valor} \\
    \hline
    |\bibname| & ``Referências Bibliográficas'' &  ``Referências'' \\
    \hline
    |\indexname| & ``Índice Remissivo'' & ``Índice'' \\
    \hline
    |\listfigurename| & ``Lista de Figuras'' & ``Lista de ilustrações'' \\
    \hline
    |\listtablename| & ``Lista de Tabelas'' & ``Lista de tabelas'' \\
    \hline
    |\pageautorefname| & ``Página'' & ``página'' \\
    \hline
    |\sectionautorefname| & ``Seção'' & ``seção'' \\
    \hline
    |\subsectionautorefname| & ``Subseção'' & ``subseção'' \\
    \hline
    |\subsubsectionautorefname| & ``Subsubseção'' & ``subseção'' \\
    \hline
    |\paragraphautorefname| & ``Parágrafo'' & ``parágrafo'' \\
    \hline
    \hline
\end{tabular}
\end{table}

Se desejar outro comportamento, use:
\begin{verbatim}
\addto\captionsbrazil{
  %% ajusta nomes padroes do babel
  \renewcommand{\bibname}{Refer\^encias}
  \renewcommand{\indexname}{\'Indice}
  \renewcommand{\listfigurename}{Lista de ilustra\c{c}\~{o}es}
  \renewcommand{\listtablename}{Lista de tabelas}
  %% ajusta nomes usados com a macro \autoref
  \renewcommand{\pageautorefname}{p\'agina}
  \renewcommand{\sectionautorefname}{se{\c c}\~ao}
  \renewcommand{\subsectionautorefname}{subse{\c c}\~ao}
  \renewcommand{\paragraphautorefname}{par\'agrafo}
  \renewcommand{\subsubsectionautorefname}{subse{\c c}\~ao}
}  
\end{verbatim} 

Veja observações sobre as alterações referentes à página, seção, subseção,
subsubseção e parágrafo na \autoref{sec-autoref}.

% ---
\subsubsection{Documentos em diferentes idiomas}
\label{sec-interncional-traducao}

\DescribeMacro{\selectlanguage}
Conforme descrito na \autoref{sec-internacional-hifen}, o \abnTeX2 já está
adequado para produzir documentos técnicos e científicos em português-brasil
(padrão) e em inglês. Para alterar o idioma padrão do documento use:

\begin{verbatim}
\selectlanguage{english}
\end{verbatim}

A macro |\selectlanguage{english}| altera os valores padrões listados na
\autoref{tab-babel}, bem como o conteúdo de todas as macros de dados listadas na
\autoref{sec-macrosdados} para o idioma inglês.

Para prover a tradução para outros idiomas, faça como no exemplo, sendo que
|captionsenglish| deve ser alterado para o idioma em questão, conforme descrito
no mamual do pacote \textsf{babel} \cite{babel}:

\begin{verbatim}
\addto\captionsenglish{% ingles
  %% adjusts names from abnTeX2
  \renewcommand{\folhaderostoname}{Title page}
  \renewcommand{\epigraphname}{Epigraph}
  \renewcommand{\dedicatorianame}{Dedication}
  \renewcommand{\errataname}{Errata sheet}
  \renewcommand{\agradecimentosname}{Acknowledgements}
  \renewcommand{\anexoname}{ANNEX}
  \renewcommand{\anexosname}{Annex}
  \renewcommand{\apendicename}{APPENDIX}
  \renewcommand{\apendicesname}{Appendix}
  \renewcommand{\orientadorname}{Supervisor:}
  \renewcommand{\coorientadorname}{Co-supervisor:}
  \renewcommand{\folhadeaprovacaoname}{Approval}
  \renewcommand{\resumoname}{Abstract} 
  \renewcommand{\listadesiglasname}{List of abbreviations and acronyms}
  \renewcommand{\listadesimbolosname}{List of symbols}
   %% adjusts names used by \autoref
  \renewcommand{\pageautorefname}{page}
  \renewcommand{\sectionautorefname}{section}
  \renewcommand{\subsectionautorefname}{subsection}
  \renewcommand{\subsubsectionautorefname}{subsubsection}
  \renewcommand{\paragraphautorefname}{paragraph}
}
\end{verbatim}


% ---
\subsection{Macros de dados do documento}
\label{sec-macrosdados}
% ---

As macros descritas nas próximas subseções são utilizadas para armazenar dados
do documento. Esses dados são utilizados na Capa (\autoref{sec-capa}), Folha de
rosto (\autoref{sec-folhaderosto}), Folha de aprovação
(\autoref{sec-folhadeaprovacao}) e Ficha catalográfica
(\autoref{sec-fichacatalografica}). Recomenda-se que esses dados sejam
preenchidos ainda no preâmbulo do documento \LaTeX, de modo que possam ser úteis
para configurações do PDF final com a classe
\textsf{hyperref}\footnote{\url{http://www.ctan.org/tex-archive/macros/latex/contrib/hyperref/}},
por exemplo. A \autoref{sec-dados_hyperref} ilustra um exemplo de configuração
do pacote \textsf{hyperref}.

As macros |\titulo|, |\autor|, |\data| e seus pares |\imprimirtitulo|,
|\imprimirautor| e |\imprimirdata| (\autoref{sec-macro-titulo},
\autoref{sec-macro-autor}, \autoref{sec-macro-data}) são meras traduções das
macros padrões |\title|, |\author|, |\date|, |\thetitle|, |\theauthor| e
|\thedate|, respectivamente. As traduções servem para abstrair o uso do
\textsf{memoir} e para simplificar futuras adaptações para outras classes bases.

A \autoref{sec-interncional-traducao} descreve como prover traduções para
outros idiomas dos nomes descritos nas subseções seguintes.

% ---
\subsubsection{Título}\label{sec-macro-titulo}
% ---

\DescribeMacro{\titulo}\DescribeMacro{\imprimirtitulo} 
A macro |\titulo|\marg{texto do título} é utilizada para armazenar o título do
documento. O conteúdo armazenado é impresso por meio da macro |\imprimirtitulo|.
Esta macro também executa a macro padrão |\title| com o mesmo conteúdo informado
em |\titulo|.

% ---
\subsubsection{Autor}\label{sec-macro-autor}
% ---

\DescribeMacro{\autor}\DescribeMacro{\imprimirautor} 
A macro |\autor|\marg{nome do(s) autor(es)} é utilizada para armazenar os nomes
dos autores do documento. O conteúdo armazenado é impresso por meio da macro
|\imprimirautor|. Esta macro também executa a macro padrão |\author| com o mesmo
conteúdo informado em |\autor|.

\DescribeMacro{\and} \DescribeMacro{\\}
Para inserir múltiplos autores, use |\and| para que uma vírgula separe os
autores no caso de a opção \texttt{article} não estar ativada, ou use |\\| para
que uma quebra de linha seja inserida entre os autores.

No caso de a opção \texttt{article} estar ativada, |\and| imprime um espaço
adicional entre os nomes, de modo que fiquem visualmente em colunas separadas.

% ---
\subsubsection{Data}\label{sec-macro-data}
% ---

\DescribeMacro{\data}\DescribeMacro{\imprimirdata} 
A macro |\data|\marg{mês e ano ou data em português} é utilizada para armazenar
o mês e ano ou a data do documento. O conteúdo armazenado é
impresso por meio da macro |\imprimirdata|. Esta macro também executa a macro
padrão |\date| com o mesmo conteúdo informado em |\data|.

Durante o desenvolvimento do documento, é possível utilizar |\data{\today}| para
que seja armazenada a data atual.

% ---
\subsubsection{Instituição}
% ---

\DescribeMacro{\instituicao}\DescribeMacro{\imprimirinstituicao} 
A macro |\instituicao|\marg{nome da instituição} é utilizada para armazenar
o nome da instituição. O conteúdo armazenado é impresso por meio da macro
|\imprimirinstituicao|.

% ---
\subsubsection{Localidade}
% ---

\DescribeMacro{\local}\DescribeMacro{\imprimirlocal} 
A macro |\local|\marg{localidade de apresentação do documento} é utilizada para
armazenar a localidade de apresentação do documento, geralmente o nome da cidade
e a unidade federativa. O conteúdo armazenado é impresso por meio da macro
|\imprimirlocal|.

% ---
\subsubsection{Preâmbulo}
% ---

\DescribeMacro{\preambulo}\DescribeMacro{\imprimirpreambulo} 
A macro |\preambulo|\marg{preâmbulo do documento} é utilizada para
armazenar o preâmbulo do documento. O preâmbulo é o texto impresso na Folha de
rosto e na Folha de aprovação. Ele deve conter o tipo do documento, o objetivo, o
nome da instituição e a área de concentração. O conteúdo armazenado é impresso
por meio da macro |\imprimirpreambulo|.

% ---
\subsubsection{Tipo de trabalho}
% ---

\DescribeMacro{\tipotrabalho}\DescribeMacro{\imprimirtipotrabalho} 
A macro |\tipotrabalho|\marg{tipo do trabalho} é utilizada para
armazenar o tipo de trabalho. Geralmente os textos ``Tese (doutorado)'' ou
``Dissertação (mestrado)'' são utilizados. O tipo de trabalho é utilizado na
Ficha catalográfica (\autoref{sec-fichacatalografica}). O conteúdo armazenado é
impresso por meio da macro |\imprimirtipotrabalho|.

% ---
\subsubsection{Orientador}
% ---

\DescribeMacro{\orientador}\DescribeMacro{\imprimirorientador}\DescribeMacro{\imprimirorientadorRotulo}
 A macro |\orientador|\oarg{rótulo}\marg{nome do(s) orientador(es)} é
utilizada para armazenar o nome do(s) orientador(es). O parâmetro opcional
indica o rótulo a ser utilizado. O valor padrão do rótulo é ``Orientador:''. O
conteúdo armazenado é impresso por meio da macro |\imprimirorientador| e o rótulo pode
ser impresso com o comando |\imprimirorientadorRotulo|.

% ---
\subsubsection{Coorientador}
% ---

\DescribeMacro{\coorientador}\DescribeMacro{\imprimircoorientador}\DescribeMacro{\imprimircoorientadorRotulo}
A macro |\coorientador|\oarg{rótulo}\marg{nome do(s) coorientador(es)} é
utilizada para armazenar o nome do(s) coorientador(es). O parâmetro opcional
indica o rótulo a ser utilizado. O valor padrão do rótulo é ``Coorientador:''. O
conteúdo armazenado é impresso por meio da macro |\imprimircoorientador| e o
rótulo pode ser impresso com o comando |\imprimircoorientadorRotulo|.

% ---
\subsubsection{Exemplo de uso \textsf{hyperref} com dados do documento}
\label{sec-dados_hyperref}
% ---

O \textsf{hyperref}\footnote{\url{http://www.ctan.org/pkg/hyperref}.} é um
pacote usado para construir remissões internas e hyper documento.

\DescribeMacro{\hypersetup}
O \textsf{hyperref} pode inserir informações dos dados do documento nos
metadados do PDF final, conforme o exemplo, que também altera informações de cores dos
links internos do documento final:

\begin{verbatim}
\usepackage{hyperref}		% controla a formação do índice

\titulo{Modelo Canônico de\\ Trabalho Acadêmico com \abnTeX}
\autor{Equipe \abnTeX}
\local{Brasil}
\data{2012}
\orientador{Lauro César Araujo}
\coorientador{Equipe \abnTeX}
\instituicao{%
  Universidade do Brasil -- UBr
  \par
  Faculdade de Arquitetura da Informação
  \par
  Programa de Pós-Graduação}
\tipotrabalho{Tese (Doutorado)}
\preambulo{Modelo canônico de trabalho monográfico acadêmico em conformidade
com as normas ABNT apresentado à comunidade de usuários \LaTeX.}

\makeatletter
\hypersetup{
		pdftitle={\@title}, 
		pdfauthor={\@author},
		pdfsubject={\imprimirpreambulo},
		pdfkeywords={PALAVRAS}{CHAVES}{EM}{PORTUGUES},
		pdfcreator={LaTeX with abnTeX2},
		colorlinks=true,
		linkcolor=blue,
		citecolor=blue,
		urlcolor=blue
}
\makeatother
\end{verbatim}

As macros |\imprimirtitulo|, e |\imprimirautor| não devem ser usadas no âmbito
da configuração de |\hypersetup|. Ao invés delas, use as macros internas
|\@title| e |\@author|, conforme o exemplo anterior.

\DescribeMacro{\pdfstringdefDisableCommands}
Na configuração dos metadados do PDF, o \abnTeX2 altera as macros |\and| e |\\|,
geralmente usadas nos campos de título e autor, para |,| e |;|, respectivamente.
Porém, isso pode ser alterado como no exemplo seguinte:

\begin{verbatim}
\pdfstringdefDisableCommands{\def\\{, }\def\and{; }}
\end{verbatim}

% ---
\subsection{Configurações para remissões internas com \texttt{autoref}}
\label{sec-autoref}
% ---

\DescribeMacro{\autoref}
A macro |\autoref|\marg{label} do pacote
\textsf{hyperref}\footnote{\url{http://www.ctan.org/pkg/hyperref}.} é usada para
realizar remissões internas que, além do número do elemento, apresentam também o
rótulo. Por exemplo, se |label| é refere-se a uma figura, |\autoref{label}|
imprime ``Figura X'', sendo ``X'' o número sequencial da figura.

Isso é especialmente útil ao se nomear divisões do documento, como capítulos,
seções, subseções, subsubseções, etc. Ocorre que em português não temos a
palavra ``subsubseção''. Por isso o \abnTeX2 traduz a referência a subsubseções
para ``subseção''. Além disso, o padrão de nomeação em português provida pelo
pacote \textsf{hyperref} desses elementos é colocá-los com letras maiúsculas.
Porém, a \citeonline[p.~4]{NBR6024:2012} apresenta exemplos de remissões
internas entre seções e parágrafos com letras minúsculas. Veja a lista completa
de alterações realizadas na \autoref{sec-internacional-babel-alteracoes}. As remissões internas
a capítulo, parte, figura, tabela e demais não foram alteradas, e serão
impressas com letras maiúsculas.

% ------
\section{Elementos pré-textuais e Parte externa}\label{sec-pretextuais}
% ------

\DescribeMacro{\pretextual}\DescribeMacro{\frontmatter}
O comando |\pretextual| identifica o início dos elementos pré-textuais. Esses
elementos não possuem numeração de páginas, nem em algarismos romanos, conforme
estabelece a ABNT NBR 14724:2011. Porém, a contagem é iniciada logo após a Capa.
A classe \textsf{abntex2} não faz separação entre ``Parte externa'' e ``Parte
interna'', por isso, a macro |\pretextual| é acionado automaticamente no início
de |\begin{document}| e desse modo você não precisa explicitamente incluí-la em
seu documento. A título de coerência, a macro |\frontmatter|, padrão do
\textsf{memoir}, é reescrita para que tenha o mesmo comportamento que
|\pretextual|.

\DescribeMacro{\pretextualchapter}
A macro |\pretextualchapter|\marg{titulo do capitulo} pode ser utilizado para
adicionar um capítulo com a formatação estabelecida na seção 5.2.3 da ABNT NBR
14724:2011 e na seção 4.1 da ABNT NBR 6024:2012. Consulte a
\autoref{sec-formatacaocapitulos} para obter informações sobre o comportamento
do \textsf{bookmark} do PDF de capítulos pré-textuais e a
\autoref{sec-formatacaocapitulos} para ler detalhes adicionais sobre a
formatação de capítulos.

As subseções seguintes descrevem como cada elemento pré-textual pode ser
construído com \abnTeX2.

% ---
\subsection{Capa (obrigatório)}\label{sec-capa}
% ---

\DescribeMacro{\imprimircapa}\DescribeEnv{capa}
A macro |\imprimircapa| imprime um modelo básico de capa que atende aos
requisito da seção 4.1.1 da ABNT NBR 14724:2011. A capa é o único elemento
``externo'' que \abnTeX2 produz.

A capa não é incluída no \textsf{bookmark} do PDF.

Para criar uma capa diferente, você pode reescrever a macro |\imprimircapa|
com base no ambiente |capa|, conforme o exemplo:

\begin{verbatim}
\renewcommand{\imprimircapa}{%
  \begin{capa}%
    \center
    {\ABNTEXchapterfont\large\imprimirautor}

    \vspace*{\fill}
    {\ABNTEXchapterfont\bfseries\LARGE\imprimirtitulo}
    \vspace*{\fill}
    
    {\large\imprimirlocal}
    \par
    {\large\imprimirdata}
    
    \vspace*{1cm}
  \end{capa}
}
\end{verbatim}


% ---
\subsection{Folha de rosto (obrigatório)}\label{sec-folhaderosto}
% ---

\DescribeMacro{\imprimirfolhaderosto}\DescribeMacro{\folhaderostoname}
A macro |\imprimirfolhaderosto|\oarg{nome da folha de rosto} imprime um modelo
básico de folha de rosto que atende aos requisito da seção 4.2.1.1 da ABNT NBR
14724:2011. A folha de rosto é incluída automaticamente no \textsf{bookmark} do
PDF com o nome dado pelo valor do argumento opcional. Caso ele não seja informado, o
conteúdo de |\folhaderostoname| é utilizado (\autoref{sec-macrosdados}).

\DescribeMacro{\imprimirfolhaderosto*}
A variante |\imprimirfolhaderosto*|\oarg{nome da folha de rosto}
deve ser utilizada quando se estiver utilizando a impressão frente e verso
(|twoside|) e se desejar incluir a ``Ficha catalográfica'' (Dados de
catalogação-na-publicação, \autoref{sec-fichacatalografica}), que deve ser
impressa no verso da Folha de rosto, conforme a seção 4.2.1.1.2 da
mesma norma. Observe que o uso de |\imprimirfolhaderosto*| sem o uso da Ficha
catalográfica poderá trazer comportamento não desejado à numeração das páginas.
Porém, se a opção |\twoside| não estiver sendo utilizada, as duas versões
da macro têm o mesmo comportamento.

Você pode criar uma folha de rosto diferente sem se preocupar com as diferenças
entre os comandos |\imprimirfolhaderosto| e |\imprimirfolhaderosto*|. Para isso,
reescreva a macro |\folhaderostocontent|, conforme o exemplo:

\begin{verbatim}
\makeatletter
\newcommand{\folhaderostocontent}{
  \begin{center}

    {\ABNTEXchapterfont\large\imprimirautor}

    \vspace*{\fill}\vspace*{\fill}
    {\ABNTEXchapterfont\bfseries\Large\imprimirtitulo}
    \vspace*{\fill}

    \abntex@ifnotempty{\imprimirpreambulo}{%
      \hspace{.45\textwidth}
      \begin{minipage}{.5\textwidth}
      	\SingleSpacing
         \imprimirpreambulo
       \end{minipage}%
       \vspace*{\fill}
    }%

  {\abntex@ifnotempty{\imprimirinstituicao}{\imprimirinstituicao
\vspace*{\fill}}}

    {\large\imprimirorientadorRotulo~\imprimirorientador\par}
    \abntex@ifnotempty{\imprimircoorientador}{%
       {\large\imprimircoorientadorRotulo~\imprimircoorientador}%
    }%
    \vspace*{\fill}

    {\large\imprimirlocal}
    \par
    {\large\imprimirdata}
    \vspace*{1cm}

  \end{center}
}
\makeatother
\end{verbatim}

% ---
\subsection{Ficha Catalográfica (Dados de
catalogação-na-publicação) (obrigatório)}\label{sec-fichacatalografica}
% ---

\DescribeEnv{fichacatalografica}
O ambiente |fichacatalografica| deve ser utilizado para impressão da Ficha
catalográfica, ou ``Dados de catalogação-na-publicação'', conforme estabelece a
seção 4.2.1.1.2 da ABNT NBR 14724:2011. 

Caso a impressão frente e verso seja acionada (opção |twoside|), é necessário
que a Folha de rosto (\autoref{sec-folhaderosto}) seja impressa com a versão
estrelada (*) da macro |\imprimirfolhaderosto*|, para que
a Ficha catalográfica seja impressa no verso da Folha de rosto.

Um exemplo de uso do comando é:

\begin{verbatim}
\begin{fichacatalografica}
	\vspace*{15cm}			% Posição vertical
	\hrule				% Linha horizontal
	\begin{center}			% Minipage Centralizado
	\begin{minipage}[c]{12.5cm}	% Largura
	
	\imprimirautor
	
	\hspace{0.5cm} \imprimirtitulo  / \imprimirautor. --
	\imprimirlocal, \imprimirdata-
	
	\hspace{0.5cm} \pageref{LastPage} p. : il.(alguma color.); 30 cm.\\
		
	\hspace{0.5cm} \imprimirorientadorRotulo \imprimirorientador\\
	
\hspace{0.5cm}
\parbox[t]{\textwidth}{\imprimirtipotrabalho~--~\imprimirinstituicao,
\imprimirdata.}\\
	
	\hspace{0.5cm}
		1. Palavra-chave1.
		2. Palavra-chave2.
		I. Orientador.
		II. Universidade xxx.
		III. Faculdade de xxx.
		IV. Título\\ 			
	
	\hspace{8.75cm} CDU 02:141:005.7\\
	
	\end{minipage}
	\end{center}
	\hrule
\end{fichacatalografica}
\end{verbatim}

O exemplo apresentado necessita do pacote |lastpage| para
que ele possa obter o número da última página do documento.
Portanto, para usar o exemplo é preciso adicionar a linha
abaixo ao preâmbulo do documento:

\begin{verbatim}
% usado por abntex2-fichacatalografica.tex
\usepackage{lastpage}
\end{verbatim}

A Ficha catalográfica não é incluída no \textsf{bookmark} do PDF
(\autoref{sec-bookmark}). 

% ---
\subsection{Errata (opcional)}
% ---

\DescribeEnv{errata}\DescribeMacro{\errataname}
A Errata é um elemento opcional da ABNT NBR 14724:2011, seção 4.2.1.2, que pode
ser criada por meio do ambiente |errata|\oarg{nome da errata}. O valor do
parâmetro opcional é utilizado como entrada no \textsf{bookmark} do PDF
(\autoref{sec-bookmark}). Caso o parâmetro não seja informado, o conteúdo de
|\errataname| é utilizado (\autoref{sec-macrosdados}).

Um exemplo de uso do ambiente é:

\begin{verbatim}
\begin{errata}

FERRIGNO, C. R. A. \textbf{Tratamento de neoplasias ósseas apendiculares com
reimplantação de enxerto ósseo autólogo autoclavado associado ao plasma
rico em plaquetas}: estudo crítico na cirurgia de preservação de membro em
cães. 2011. 128 f. Tese (Livre-Docência) - Faculdade de Medicina Veterinária 
e Zootecnia, Universidade de São Paulo, São Paulo, 2011.

\begin{table}[htb]
\center
\footnotesize
\begin{tabular}{|p{1.4cm}|p{1cm}|p{3cm}|p{3cm}|}
  \hline
  \textbf{Folha} & \textbf{Linha} & \textbf{Onde se lê} & \textbf{Leia-se}\\
  \hline
  1 & 10 & auto-conclavo & autoconclavo\\
  \hline
\end{tabular}
\end{table}

\end{errata}
\end{verbatim}

% ---
\subsection{Folha de aprovação (obrigatório)}\label{sec-folhadeaprovacao}
% ---

\DescribeEnv{folhadeaprovacao}\DescribeMacro{\folhadeaprovacaoname}
O ambiente |folhadeaprovacao|\oarg{nome da folha de aprovação} permite a criação
de uma Folha de aprovação, elemento obrigatório da ABNT NBR 14724:2011
descrita na seção 4.2.1.3 da norma. O valor do parâmetro opcional é utilizado
como entrada no \textsf{bookmark} do PDF (\autoref{sec-bookmark}). Caso o parâmetro
não seja informado, o conteúdo de |\folhadeaprovacaoname| é utilizado
(\autoref{sec-macrosdados}). Conforme estabelece a seção 5.2.4 da norma em tela,
a Folha de aprovação não possui título nem indicador numérico.

\DescribeMacro{\includepdf}
Um modelo de Folha de aprovação não é oferecido pelo \abnTeX2 porque
ela varia largamente entre instituições. E, além disso, provavelmente você
incluirá uma versão digitalizada com assinaturas dos membros da banca após a
apresentação do trabalho. Uma página digitalizada pode ser incluída no
documento com o comando:

\begin{verbatim}
\includepdf{folhadeaprovacao_final.pdf}
\end{verbatim}

De todo modo, você pode utilizar o seguinte modelo de Folha de aprovação até a
aprovação final do trabalho:

\begin{verbatim}
\begin{folhadeaprovacao}

  \begin{center}
    {\ABNTEXchapterfont\large\imprimirautor}

    \vspace*{\fill}\vspace*{\fill}
    {\ABNTEXchapterfont\bfseries\Large\imprimirtitulo}
    \vspace*{\fill}
    
    \hspace{.45\textwidth}
    \begin{minipage}{.5\textwidth}
        \imprimirpreambulo
    \end{minipage}%
    \vspace*{\fill}
   \end{center}
    
   Trabalho aprovado. \imprimirlocal, 24 de novembro de 2012:

   \assinatura{\textbf{\imprimirorientador} \\ Orientador} 
   \assinatura{\textbf{Professor} \\ Convidado 1}
   \assinatura{\textbf{Professor} \\ Convidado 2}
   \assinatura{\textbf{Professor} \\ Convidado 3}
   \assinatura{\textbf{Professor} \\ Convidado 4}
      
   \begin{center}
    \vspace*{0.5cm}
    {\large\imprimirlocal}
    \par
    {\large\imprimirdata}
    \vspace*{1cm}
  \end{center}
  
\end{folhadeaprovacao}
\end{verbatim}

\DescribeMacro{\assinatura}
A macro |\assinatura|\marg{texto a ser impresso} é um utilitário para
impressão de assinaturas da Folha de aprovação. Ela imprime o 
|texto a ser impresso| centralizado abaixo de uma linha. A versão |\assinatura*|
imprime a mesma assinatura em uma |box| sem posição atribuída, o que é útil para impressão
de assinaturas lado a lado.

\DescribeMacro{\ABNTEXsignwidth}
O comprimento da linha de assinatura é definido pela métrica |\ABNTEXsignwidth|.
O valor padrão é definido como: |\setlength{\ABNTEXsignwidth}{8cm}|.

\DescribeMacro{\ABNTEXsignthickness}
A largura da linha de assinatura é definida pela métrica |\ABNTEXsignthickness|. O
valor padrão é definido como: |\setlength{\ABNTEXsignthickness}{1pt}|.

\DescribeMacro{\ABNTEXsignskip}
O espaçamento entre um comando |\assinatura| e outro é definido pela métrica
|\ABNTEXsignskip|. O valor padrão é definido como:
|\setlength{\ABNTEXsignskip}{1cm}|.

% ---
\subsection{Dedicatória (opcional)}
% ---

\DescribeEnv{dedicatoria}\DescribeMacro{\dedicatorianame}
A Dedicatória é um elemento opcional da ABNT NBR 14724:2011, seção 4.2.1.4, que
pode ser criada por meio do ambiente |dedicatoria|\oarg{nome da dedicatória}.

O valor do parâmetro opcional é utilizado como entrada no \textsf{bookmark} do
PDF (\autoref{sec-bookmark}) e como título da dedicatória, que é escrito como um
título de capítulo pré-textual, ou seja, não numerado e centralizado. Caso o
parâmetro não seja informado, o conteúdo de |\dedicatorianame| é utilizado apenas
para entrada do \textsf{bookmark} do PDF e a dedicatória é impressa sem título e
sem indicador numérico, conforme estabelece a seção 5.2.4 da norma em tela.

Um exemplo de uso do ambiente é:

\begin{verbatim}
\begin{dedicatoria}

   \vspace*{\fill}
   Este trabalho é dedicado aos que acreditam...
   \vspace*{\fill}

\end{dedicatoria}
\end{verbatim}

% ---
\subsection{Agradecimentos (opcional)}
% ---

\DescribeEnv{agradecimentos}\DescribeMacro{\agradecimentosname}
A seção Agradecimentos é um elemento opcional da ABNT NBR 14724:2011, 4.2.1.5,
que pode ser criada por meio do ambiente |agradecimentos|\oarg{nome dos
agradecimentos}. O valor do parâmetro opcional é utilizado como entrada no
\textsf{bookmark} do PDF (\autoref{sec-bookmark}). Caso o parâmetro não seja
informado, o conteúdo de |\agradecimentosname| é utilizado (\autoref{sec-macrosdados}).

Um exemplo de uso do ambiente é:

\begin{verbatim}
\begin{agradecimentos}

   Os agradecimentos...

\end{agradecimentos}
\end{verbatim}

% ---
\subsection{Epígrafe (opcional)}
% ---

\DescribeEnv{epigrafe}\DescribeMacro{\epigraphname}
A Epígrafe é um elemento opcional da ABNT NBR 14724:2011, seção 4.2.1.6, que
pode ser criada por meio do ambiente |epigrafe|\oarg{nome da epígrafe}.
O valor do parâmetro opcional é utilizado como entrada no \textsf{bookmark} do
PDF (\autoref{sec-bookmark}) e como título da epígrafe, que é escrito como um
título de capítulo pré-textual, ou seja, não numerado e centralizado. Caso o
parâmetro não seja informado, o conteúdo de |\epigraphname| é utilizado apenas
para entrada do \textsf{bookmark} do PDF e a epígrafe é impressa sem título e
sem indicador numérico, conforme estabelece a seção 5.2.4 da norma em tela.

Um exemplo de uso do ambiente é:

\begin{verbatim}
\begin{epigrafe}

   \vspace*{\fill}
   \begin{flushright}
      \textit{``Não vos amoldeis às estruturas deste mundo, \\
      mas transformai-vos pela renovação da mente, \\
      a fim de distinguir qual é a vontade de Deus: \\
      o que é bom, o que Lhe é agradável, o que é perfeito.\\
      (Bíblia Sagrada, Romanos 12, 2)}
    \end{flushright}
	
\end{epigrafe}
\end{verbatim}

% ---
\subsection{Resumos em língua vernácula e estrangeira (obrigatório)}
% ---

\DescribeEnv{resumo}\DescribeEnv{resumo}\DescribeMacro{\resumoname}
Os resumos em língua vernácula e estrangeira são elementos obrigatórios da ABNT
NBR 14724:2011, seção 4.2.1.7 e 4.2.1.8, e devem ser escritos conforme
orientação da ABNT NBR 6028. Os elementos podem ser criados por meio do ambiente
|resumo|\oarg{nome do resumo}.
O valor do parâmetro opcional é utilizado como entrada no \textsf{bookmark} do
PDF (\autoref{sec-bookmark}) e como título do resumo. Caso o parâmetro não seja
informado, o conteúdo de |\resumoname| é utilizado. O parâmetro opcional é útil
para criação de resumos em diversos idiomas estrangeiros. 

Exemplos de uso do ambiente são:

\begin{verbatim}
% --- resumo em português ---
\begin{resumo}
  Resumo em português
  \vspace{\onelineskip}
  \noindent
  \textbf{Palavras-chaves}: latex. abntex. editoração de texto.
\end{resumo}

% --- resumo em francês ---
\begin{resumo}[Résumé]
 \begin{otherlanguage*}{french}
    Il s'agit d'un résumé en français.
    \vspace{\onelineskip}
    \noindent
    \textbf{Mots-clés}: latex. abntex. publication de textes.
 \end{otherlanguage*}
\end{resumo}
\end{verbatim}

\DescribeEnv{resumoumacoluna}\DescribeMacro{\resumoname}\DescribeMacro{\twocolumn}
Em documentos que utilizam a opção |twocolumn| para produzir o texto em duas
colunas --- geralmente utilizado conjuntamente com |article| ---, pode-se
desejar imprimir o resumo em uma única coluna e o restante do documento em duas.
Nesse caso, utilize o ambiente |resumoumacoluna|\oarg{nome do resumo}. 

Embora o parâmetro opcional |nome do resumo| esteja disponível, o ambiente
|resumoumacolumna| deve ser usado no contexto da macro |\twocolumn|, que por sua
vez não permite que nenhum ambiente ou macro possua outros parâmetros. Desse
modo, não é possível alterar o nome do resumo com o parâmetro opcional.
Felizmente, é possível fazê-lo redefinindo a macro |\resumoname|, conforme os
exemplos que seguem. A macro |\twocolumn| é usada para passar parâmetro à opção
|twocolumn| da classe \textsf{abntex2}.

Exemplos de uso do ambiente são:

\begin{verbatim}
\twocolumn[  % indica que inicia-se opção de twocolumn
% --- resumo em português ---
\begin{resumoumacoluna}
  Resumo em português
  \vspace{\onelineskip}
  \noindent
  \textbf{Palavras-chaves}: latex. abntex. editoração de texto.
\end{resumoumacoluna}

% --- resumo em francês ---
\renewcommand{\resumoname}{Résumé}
\begin{resumoumacoluna}
  \begin{otherlanguage*}{french}
    Il s'agit d'un résumé en français.
    \vspace{\onelineskip}
    \noindent
    \textbf{Mots-clés}: latex. abntex. publication de textes.
  \end{otherlanguage*}  
\end{resumoumacoluna}
] fim de opção de twocolumn
\end{verbatim}

A \autoref{sec-internacional-hifen} aborda a macro |otherlanguage*|,
responsável pela hifenização em diferentes idiomas.

% ---
\subsection{Lista de ilustrações (opcional)}\label{sec-listadeilustracoes}
% ---

\DescribeMacro{\listoffigures}
A Lista de ilustrações é um elemento opcional da ABNT NBR 14724:2011, seção
4.2.1.9, que pode ser criada por meio da macro padrão |\listoffigures|.

Nem a classe \textsf{memoir}, nem a classe \textsf{abntex2} incluem
automaticamente o capítulo criado pela macro |\listoffigures| no
\textsf{bookmark} do PDF (\autoref{sec-bookmark}). 

O exemplo seguinte cria a Lista de ilustrações e já a adiciona ao
\textsf{bookmark}:

\begin{verbatim}
\pdfbookmark[0]{\listfigurename}{lof}
\listoffigures*
\cleardoublepage
\end{verbatim}

A seção 4.2.1.9 da norma ABNT NBR 14724:2011 recomenda que, quando necessário,
seja produzido uma lista própria para cada tipo de ilustração, como desenhos,
esquemas, fluxogramas, fotografias, gráficos, mapas, organogramas, plantas,
quadros, retratos, e outros. Como essa necessidade é específica de cada
trabalho, o \abnTeX2 não traz essa implementação automaticamente.
Porém, diferentes tipos de lista podem ser criadas por meio de macros do
\textsf{memoir}. Para isso, consulte o capítulo 9 do manual do \textsf{memoir}
\cite{memoir}.

% ---
\subsection{Lista de tabelas (opcional)}\label{sec-listadetabelas}
% ---

\DescribeMacro{\listoftables}
A Lista de tabelas é um elemento opcional da ABNT NBR 14724:2011, seção
4.2.1.10, que pode ser criada por meio da macro padrão |\listoftables|.

Nem a classe \textsf{memoir}, nem a classe \textsf{abntex2} incluem
automaticamente o capítulo criado pela macro |\listoftables| no
\textsf{bookmark} do PDF (\autoref{sec-bookmark}). 

O exemplo seguinte cria a Lista de tabelas e já a adiciona ao
\textsf{bookmark}:

\begin{verbatim}
\pdfbookmark[0]{\listtablename}{lot}
\listoftables*
\cleardoublepage
\end{verbatim}

% ---
\subsection{Lista de abreviaturas e siglas (opcional)}
\label{sec-listadeabreviaturas}
% ---

\DescribeEnv{siglas}
A Lista de abreviaturas e siglas é um elemento opcional da ABNT NBR 14724:2011,
seção 4.2.1.11 e pode ser criada com o ambiente |siglas|:

\begin{verbatim}
  \begin{siglas}
    \item[Fig.] Area of the $i^{th}$ component
    \item[456] Isto é um número
    \item[123] Isto é outro número
    \item[lauro cesar] este é o meu nome
  \end{siglas}
\end{verbatim} 

\DescribeMacro{\listadesiglasname}
A macro |\listadesiglasname| contém o nome da lista de abreviaturas e siglas. 

\DescribeMacro{\printnomenclature}
Opcionalmente, é possível usar o pacote
\textsf{nomencl}\footnote{\url{http://www.ctan.org/tex-archive/macros/latex/contrib/nomencl/}},
que oferece recursos adicionais na composição de lista de siglas, símbolos e até
glossários. Com esse pacote crie a lista de abreviaturas e siglas
seguindo o exemplo:

No preâmbulo:

\begin{verbatim}
\usepackage{nomencl}
   
\makenomenclature
\end{verbatim}

No corpo do documento:

\begin{verbatim}
\nomenclature{Fig.}{Figura}
\nomenclature{$A_i$}{Area of the $i^{th}$ component} 
\nomenclature{456}{Isto é um número}
\nomenclature{123}{Isto é outro número}
\nomenclature{a}{primeira letra do alfabeto}
\nomenclature{lauro}{este é meu nome} 

\renewcommand{\nomname}{\listadesiglasname}
\pdfbookmark[0]{\nomname}{las}
\printnomenclature
\cleardoublepage
\end{verbatim}

Para usar o pacote \textsf{nomencl}, é necessário compilar o documento \LaTeX\
com \textsf{makeindex}:

\begin{verbatim}
makeindex ARQUIVO_PRINCIPAL.nlo -s nomencl.ist -o ARQUIVO_PRINCIPAL.nls
\end{verbatim}

Após a compilação com \textsf{makeindex}, compile normalmente o documento com
|pdflatex|.

Geralmente os editores de documentos \LaTeX\ possuem formas de automatizar essa
compilação. Porém, caso você opte por usar o ambiente |siglas|, nenhuma
compilação adicional é necessária.

% ---
\subsection{Lista de símbolos (opcional)}\label{sec-listadesimbolos}
% ---

\DescribeEnv{simbolos}
A lista de símbolos é um elemento opcional da ABNT NBR 14724:2011,
seção 4.2.1.12 e pode ser criada com o ambiente |simbolos|:

\begin{verbatim}
\begin{simbolos}
  \item[$ \Gamma $] Letra grega Gama
  \item[$ \Lambda $] Lambda
  \item[$ \zeta $] Letra grega minúscula zeta
  \item[$ \in $] Pertence
\end{simbolos}
\end{verbatim}

\DescribeMacro{\listadesimbolosname}
A macro |\listadesimbolosname| o nome da lista de símbolos.

Opcionalmente, use o pacote \textsf{nomencl} ou \textsf{glossaries}, que tanto
podem construir a Lista de símbolos, como Glossários. Consulte 
\autoref{sec-listadeabreviaturas} e a \autoref{sec-glossarios} para outras
informações.

% ---
\subsection{Sumário (obrigatório)}\label{sec-sumario}
% ---

\DescribeMacro{\tableofcontents}
O Sumário é um elemento obrigatório da ABNT NBR 14724:2011, seção
4.2.1.13, que pode ser criada por meio da macro padrão |\tableofcontents|. O
sumário deve ser construído conforme a ABNT NBR 6027:2012.

Nem a classe \textsf{memoir}, nem a classe \textsf{abntex2} incluem
automaticamente o capítulo criado pela macro |\tableofcontents| no
\textsf{bookmark} do PDF. Caso deseje que o título do capítulo seja
incluído no \textsf{bookmark} (\autoref{sec-bookmark}), utilize o exemplo
abaixo:

\begin{verbatim}
\pdfbookmark[0]{\contentsname}{toc}
\tableofcontents*
\cleardoublepage
\end{verbatim}

\DescribeMacro{\settocdepth}
Você pode customizar o nível de divisões que o sumário pode listar com a
macro |\settocdepth|\marg{nome da subdivisão}, sendo |nome da subdivisão| um
dos valores: |chapter|, |part|, |section|, |subsection|, |subsubsection|.

A configuração padrão do \abnTeX2 é |\settocdepth{subsubsection}|.

\DescribeMacro{\setsecnumdepth}
Também é possível customizar se a numeração das divisões é exibida no
sumário. Para isso use |\setsecnumdepth|\marg{nome da subdivisão}, sendo
|nome da subdivisão| um dos mesmos valores utilizados em |\settocdepth|. 

A configuração padrão do \abnTeX2 é |\setsecnumdepth{subsubsection}|.

\DescribeMacro{\tocheadstart}
É possível customizar a fonte das partes e dos capítulos no Sumário. Para isso,
redefina a macro |\tocheadstart|. O \abnTeX2 a redefine por padrão para que a
fonte utilizada no Sumário seja a mesma defina para o capítulo, da seguinte
maneira:

\begin{verbatim}
\renewcommand{\tocheadstart}{\ABNTEXchapterfont}
\end{verbatim}

\DescribeMacro{\addcontentsline}
O comando |\addcontentsline|\marg{sigla do sumario}\marg{nível da
divisão}\marg{texto no sumário} pode ser usado para incluir uma linha no
Sumário. Use o comando, por exemplo, após a criação de capítulo não numerado: 

\begin{verbatim}
\chapter*{Introdução}
\addcontentsline{toc}{chapter}{Introdução}
\end{verbatim}

É importante destacar que nenhum elemento pré-textual deve estar presente no
Sumário. Veja mais informações na \autoref{sec-elementostextuais}.

Consulte a \autoref{sec-bookmark} para obter informações sobre o
\textsf{bookmark}, índice da estrutura do documento no PDF.

\DescribeMacro{KeepFromToc}
O ambiente |KeepFromToc| pode ser utilizada para que um divisão não seja
incluída no Sumário. Esse ambiente é equivalente à macro |\ProximoForaDoSumario|
utilizada pela versão anterior do \abnTeX, e que não está mais presente nesta
versão.

Use a macro como no exemplo:

\begin{verbatim}
\begin{KeepFromToc}
   \chapter{Este capítulo não aparece no sumário}
   \section{Nem esta seção}
\end{KeepFromToc}
\end{verbatim}

O capítulo 9 do mamual do \textsf{memoir} detalha como o sumário pode ser
customizado \cite{memoir}.

% ------
\section{Elementos textuais}\label{sec-elementostextuais}
% ------

\DescribeMacro{\textual}\DescribeMacro{\mainmatter}
O comando |\textual| identifica o início dos elementos textuais. As páginas
desses elementos são numeradas com algarismos arábicos no lado direito superior
ou direito/esquerdo superior caso a impressão frente e verso (opção |twoside|)
seja acionada, conforme estabelece a ABNT NBR 14724:2011. Geralmente a
``Introdução'' é o primeiro capítulo textual. A título de coerência, a macro
|\mainmatter|, padrão do \textsf{memoir}, é reescrita para que tenha o mesmo
comportamento que |\textual|. Por isso, fique livre em escolher qualquer das
macros. Porém, o uso de uma delas é obrigatória, para que os cabeçalhos sejam
montados corretamente.

Segundo a ABNT NBR 14724:2011, seção 4.2.2, ``o texto é composto de uma parte
introdutória, que apresenta os objetivos do trabalho e as razões de sua
elaboração; o desenvolvimento, que detalha a pesquisa ou estudo realizado; e uma
parte conclusiva.'' Os títulos dos capítulos textuais são à critério do
autor e não há nenhuma normatização a respeito deles. No entanto, geralmente o
capítulo ``Introdução'' e o capítulo ``Conclusão'' (ou ``Considerações
finais'') são, respectivamente, o primeiro e o último capítulo textual
e normalmente não são numerados. 

É importante destacar que a norma em tela e a ABNT NBR 6024:2012 não são
explícitas sobre a possibilidade de não numeração de capítulos
textuais\footnote{Embora a seção 5.2.3 da ABNT NBR 14724:2011 seja clara a
respeito dos capítulos pré e pós-textuais estabelecidos por ela: ``Os títulos,
sem indicativo numérico --- errata, agradecimentos, lista de ilustrações, lista
de abreviaturas e siglas, lista de símbolos, resumos, sumário, referências,
glossário, apêndice(s), anexo(s) e índice(s) –-- devem ser centralizados''.}.
Desse modo, sugere-se que se siga o modo de numeração desses capítulos utilizado
pela instituição que você apresentará o trabalho.

\DescribeMacro{\chapter*}
Caso queira incluir capítulos sem numeração (como Introdução e Conclusão, por
exemplo), utilize a macro |\chapter*|\marg{Introdução}. Porém, capítulos com *
não são incluídos automaticamente no Sumário, nem no \textsf{bookmark} do PDF,
nem alteram o cabeçalho das páginas. 

Consulte a \autoref{sec-sumario} para ver como incluir divisões com * ao
sumário.

Consulte a \autoref{sec-formatacaocapitulos} para obter outras
informações sobre formatação de capítulos e, especialmente, consulte a
\autoref{sec-formatacaocapitulos-starred} para ver como alterar automaticamente
os cabeçalhos das páginas com o comando |\chapter*|.

Consulte a \autoref{sec-bookmark} para ver como controlar o \textsf{bookmark} do
PDF.

\DescribeMacro{\part}
A macro |\part|\marg{nome da parte} pode ser utilizada para que uma
página de divisão do trabalho seja incluída. A parte agrupa capítulos. Um
exemplo é o uso do trabalho dividido em três partes:

\begin{verbatim}
\part{Preparação da Pesquisa}
   (...)
   \chapter{Metodologia}
   (...)
\part{Revisão de Literatura}
   (...)
   \chapter{O trabalho de Charles Darwin}
   (...)
\part{Resultados}
   (...)
\end{verbatim}

% ---
\subsection{Formatação dos títulos: partes, capítulos, seções, subseções e
subsubseções}
\label{sec-formatacaocapitulos}
% ---

\DescribeMacro{\ABNTEXchapterfont}\DescribeMacro{\ABNTEXchapterfontsize}
O \textsf{memoir} possui uma vasta lista de opções de estilos de capítulos
(ver \autoref{sec-formatacaocapitulos-adicionais}).
A classe \textsf{abntex2} adiciona a essa lista um estilo chamado |abnt|, que
atende aos requisitos na ABNT NBR 14724:2011 e da ABNT NBR 6024:2012. O estilo
|abnt| é carregado automaticamente e possui duas configurações adicionais:
|\ABNTEXchapterfont| é a fonte utilizada nos capítulos e
|\ABNTEXchapterfontsize| é o tamanho da fonte. 

Por padrão, uma versão sem serifas da fonte corrente do documento é
utilizada para os títulos das divisões. Você pode customizar a fonte dos
títulos dos capítulos alterando os comandos como no exemplo a seguir, para que
seja utilizada a fonte \emph{Computer Modern} com tamanho maior do que o
utilizado por padrão:

\begin{verbatim}
\renewcommand{\ABNTEXchapterfont}{\fontfamily{cmr}\fontseries{b}\selectfont}
\renewcommand{\ABNTEXchapterfontsize}{\HUGE}
\end{verbatim}

\DescribeMacro{\ABNTEXpartfont}\DescribeMacro{\ABNTEXpartfontsize}
\DescribeMacro{\ABNTEXsectionfont}\DescribeMacro{\ABNTEXsectionfontsize}
\DescribeMacro{\ABNTEXsubsectionfont}\DescribeMacro{\ABNTEXsubsectionfontsize}
\DescribeMacro{\ABNTEXsubsubsectionfont}\DescribeMacro{\ABNTEXsubsubsectionfontsize}

As fontes e o tamanho das fontes obtidas com as divisões |\part|\marg{nome da
parte}, |\chapter|\marg{nome do capítulo}, |\section|\marg{nome da seção},
|\subsection|\marg{nome da subseção} e |\subsubsection|\marg{nome da subsubseção} são definidas por padrão,
respectivamente, conforme a \autoref{tab-divisoes}. Você pode alterá-las com o
comando |\renewcommand|.

\begin{table}[htb]
\caption{Macros de formatação de fonte de divisões do texto}
\label{tab-divisoes}
\centering
\begin{tabular}{ l l l }
   \textbf{Macro} & \textbf{Valor padrão} \\
    \hline
    |\ABNTEXchapterfont| & |\sffamily|\footnote{Versão sem serifas da fonte
    corrente.}
    \\
    \hline
    |\ABNTEXchapterfontsize| & |\Huge| \\
    \hline
    
    |\ABNTEXpartfont| & |\ABNTEXchapterfont| \\
    \hline
    |\ABNTEXpartfontsize| & |\ABNTEXchapterfontsize| \\
    \hline
    
    |\ABNTEXsectionfont| & |\ABNTEXchapterfont| \\
    \hline
    |\ABNTEXsectionfontsize| & |\Large| \\
    \hline

    |\ABNTEXsubsectionfont| & |\ABNTEXsectionfont| \\
    \hline
    |\ABNTEXsubsectionfontsize| & |\large| \\
    \hline

    |\ABNTEXsubsubsectionfont| & |\ABNTEXsubsectionfont| \\
    \hline
    |\ABNTEXsubsubsectionfontsize| & |\normalsize| \\
    \hline
    \hline

\end{tabular}

\end{table}

% ---
\subsubsection{Estilos adicionais de capítulos}
\label{sec-formatacaocapitulos-adicionais}
% ---

\DescribeMacro{\chapterstyle}
O estilo de capítulo |abnt| provido pela classe \abnTeX2 pode ser substituído
por outro estilo já fornecido pelo \textsf{memoir} ou mesmo por outro criado por
você. Isso é útil especialmente se estiver interessado em publicar seu trabalho
como livro ou não se importar em não seguir o padrão normativo. Para isso,
utilize o comando:

\begin{verbatim}
\chapterstyle{nome_do_estilo}
\end{verbatim}

Experimente |lyhne| ou |dash|. Você encontra alguns estilos no manual do
\textsf{memoir} e outros neste documento:
\url{http://www.tex.ac.uk/tex-archive/info/MemoirChapStyles/MemoirChapStyles.pdf}.
Ambos mostram como criar um novo estilo.

% ---
\subsubsection{Cabeçalhos de capítulos com *}
\label{sec-formatacaocapitulos-starred}
% ---

\DescribeMacro{\chapter*}
Conforme descrito na \autoref{sec-bookmark}, os capítulos criados com
|\chapter*| não alteram automaticamente o cabeçalho da página. Ou seja, o
cabeçalho de um capítulo iniciado com o comando |\chapter*| é o mesmo do
capítulo anterior, se houver.

Para alterar esse comportamento, informe o parâmetro opcional:
|\chapter*|\oarg{nome do capítulo do cabeçalho}\marg{nome do capítulo na
página}.

Para mais detalhes, consulte o \citeonline[p. 73]{memoir}.

% ---
\subsubsection{Espaçamento entre os capítulos e o texto}
% ---

Segundo a \citeonline[seção 5.2.2]{NBR14724:2011}, os títulos das seções
primárias (os capítulos) ``devem começar em página ímpar (anverso)\footnote{Esse requisito é 
atendido ao usar a opção \texttt{openright}.}, na parte superior da 
mancha gráfica e ser separados do texto que os sucede por um espaço entre as linhas 
de 1,5''. Porém, propositalmente os modelos do \abnTeX não atendem a essa regra, uma vez
que utiliza-se tamanhos diferentes de fontes para capítulos e, nesse caso,
um espaçamento de 1,5 entre o capítulo e o início do texto não ficaria esteticamente 
elegante. Observe que a norma em questão nem as demais normas estabelece um tamanho de fonte 
para as seções, o que torna livre a opção de um tamanho diferente (e maior) de fonte para
os capítulos do que o tamanho da fonte do corpo do texto.

De toda forma, caso você queira seguir estritamente o requisito da norma referente ao
espaçamento entre o capítulo e o início do texto, faça o seguinte.

Utilize no preâmbulo do documento:
\begin{verbatim}
\setlength\afterchapskip{\lineskip}
\end{verbatim}

Imediatamente após a declaração de um capítulo de apêndice ou de anexo:
\begin{verbatim}
\chapter{Título de um anexo ou apêndice}
\setlength{\afterchapskip}{-\baselineskip} 

O texto do apêndice ou do anexo...
\end{verbatim}

Caso o título do apêndice ou do anexo tenha mais de duas linhas, use o comando
abaixo para ajustar um eventual problema de espaçamento:

\begin{verbatim}
\chapter{Título de um anexo ou apêndice muito longo, com mais de duas linhas, que pode
ocasionar um problema de espaçamento}
\phantom{x}

O texto do apêndice ou do anexo...
\end{verbatim}


% ---
\subsection{Citações diretas com mais de três linhas}
\label{sec-citacao}
% ---

\DescribeEnv{citacao}
A ABNT NBR 10520:2002, seção 5.3, descreve que citações diretas com mais de três
linhas devem ser destacadas com recuo de 4 cm da margem esquerda, com letra
menor que a do texto utilizado e sem as aspas. Para inserir citações longas,
utilize o ambiente |citacao|, conforme o exemplo:

\begin{verbatim}
\begin{citacao}
As citações diretas, no texto, com mais de três linhas, devem ser
destacadas com recuo de 4 cm da margem esquerda, com letra menor que
a do texto utilizado e sem as aspas. No caso de documentos datilografados,
deve-se observar apenas o recuo \cite[5.3]{NBR10520:2002}
\end{citacao}
\end{verbatim}

O ambiente |citacao|\oarg{language} pode receber como parâmetro opcional um nome
de idioma previamente carregado nas opções da classe
(\autoref{sec-internacional-hifen}). Nesse caso, o texto da citação é
automaticamente escrito em itálico e a hifenização é ajustada para o idioma
selecionado na opção do ambiente, conforme o exemplo:

\begin{verbatim}
\begin{citacao}[english]
Text in English language in italic with correct hyphenation.
\end{citacao}
\end{verbatim}

A \autoref{sec-internacional-hifen} descreve o uso de diferentes idiomas no
texto.

O tamanho da fonte utilizada no ambiente |citacao| é determinada pela macro
|\ABNTEXfontereduzida|, descrita na \autoref{sec-configgerais}.

\DescribeMacro{\ABNTEXcitacaorecuo}
O recuo utilizado pelo ambiente |citacao| é definido pela métrica\\ |\ABNTEXcitacaorecuo|, que pode ser alterado com:

\begin{verbatim}
\setlength{\ABNTEXcitacaorecuo}{1.8cm}
\end{verbatim}

Quando um documento é produzido com a opção \texttt{twocolumn}, a classe
\textsf{abntex2} automaticamente altera o recuo padrão definido pela
ABNT NBR 10520:2002 para 1.8cm. 



% ---
\subsection{Alíneas e Subalíneas}
% ---

A ABNT NBR 6024:2012, seção 4.2, descreve o uso das alíneas, que podem ser
compreendias como subdivisões não nomeadas de uma seção. 

As alíneas são numeradas com letras minúsculas do alfabeto com recuo em relação
à margem esquerda do documento. A norma prescreve que o texto que antecede as
alíneas deve finalizar com dois pontos (:); as alíneas devem iniciar com letra
minúscula e serem finalizadas com ponto e vírgula (;), exceto a última alínea,
que deve ser finalizada com ponto final, e exceto as alíneas que precederem uma
subalínea, caso em que devem ser finalizadas com dois pontos (:); a segunda e as
seguintes linhas do texto da alínea começa sob a primeira letra do texto da
própria alínea. 

Os ambientes descritos nesta seção são criados com o pacote \textsf{enumitem}.
Consulte a documentação do pacote em \citeonline{enumitem} para obter detalhes e
opções adicionais de configuração.

\DescribeEnv{alineas}
A classe \textsf{abntex2} fornece o ambiente |alineas|, que cria listas conforme
o padrão estipulado pela norma. Veja o exemplo:

\begin{verbatim}
\begin{alineas}
  \item linha 1;
  \item linha 2;
  \item linha 3.
\end{alineas}   
\end{verbatim}
  
\DescribeEnv{subalineas}\DescribeEnv{incisos}
As alíneas podem ser aninhadas. Nesse caso, a numeração é substituída por um
travessão. Você pode criar uma subalínea de três formas diferentes, todas
equivalentes entre si: com outro ambiente |alineas|, com |subalineas| ou
ainda com o ambiente |incisos|:

\begin{verbatim}
\begin{alineas}

  \item linha 1:
  \begin{alineas}
    \item subalinea 1;
    \item subalinea 2;
  \end{alineas}   

  \item linha 2:
  \begin{subalineas}
    \item subalinea 1;
    \item subalinea 2;
  \end{subalineas}

  \item linha 3:   
  \begin{incisos}
    \item subalinea 1;
    \item subalinea 2;
  \end{incisos}

  \item linha 4.
\end{alineas}   
\end{verbatim}

% ---
\subsection{Rótulos e legendas}
% ---

\DescribeMacro{\caption}\DescribeMacro{\legend}
Rótulos e legendas de ilustrações, tabelas e qualquer outro ambiente do tipo
|listing| podem ser definidos pelos comandos |\caption|\marg{rotulo} e
|\legend|\marg{legenda}, respectivamente.

Conforme a ABNT NBR 14724:2011, seção 5.8, o rótulo é atribuído acima do
elemento e a legenda abaixo, conforme no exemplo:

\begin{verbatim}
\begin{figure}[htb]
	\caption{\label{fig_circulo}A delimitação do espaço}
	\begin{center}
	  \includegraphics[scale=0.75]{myfig.pdf}
	\end{center}
	\legend{Fonte: os autores}
\end{figure}
\end{verbatim}

\DescribeMacro{\includegraphics}
A macro |\includegraphics| pode ser utilizada para inclusão de imagens.
Recomenda-se que imagens vetoriais, como imagens em PDF, sejam preferidas em
oposição a imagens baseadas em mapas de bits, uma vez que desse forma não há
perda de qualidade nas imagens. Porém, formatos como PNG, BMP, JPG e outros são
aceitos pelo \LaTeX.

% ------
\section{Elementos pós-textuais}
% ------

\DescribeMacro{\postextual}\DescribeMacro{\backmatter}
O comando |\postextual| identifica o início dos elementos pós-textuais. Na
prática não há nenhum comportamento específico, uma vez que as
normas não prescrevem nenhum requisito para esses elementos. Porém, mesmo
que para uso futuro, a macro |\postextual| já está criada e recomenda-se que
seja utilizada. Dessa forma, caso deseje atribuir algum comportamento
diferenciado aos elementos pós-textuais, faça-o redefinindo a macro. A título de
coerência, a macro |\backmatter|, padrão do \textsf{memoir}, é reescrita para
que tenha o mesmo comportamento que |\postextual|.

% ---
\subsection{Referências (obrigatório)}\label{sec-referencias}
% ---

\DescribeMacro{\bibliography}
A classe \textsf{abntex2} é responsável pela estruturação e o aspecto geral dos
documentos. Mais precisamente, ela é focada em atender os requisitos
apresentados pela norma ABNT NBR 10719:2011, ABNT NBR 14724:2011, ABNT NBR
15287:2011 e normas correlatas. As referências bibliográficas são normatizadas
pela norma ABNT NBR 10520:2002 e ABNT NBR 6023:2002. Os requisitos impostos
por estas normas são atendidos pelo pacote \textsf{abntex2cite}.

Esta seção apresenta apenas uma introdução ao uso de referências bibliográficas
com \abnTeX2. Para um detalhamento completo do tema, consulte os manuais
\citeonline{abntex2cite} e \citeonline{abntex2cite-alf}.

Para utilizar o padrão de bibliografias brasileiro implementado pelo pacote
\textsf{abntex2cite}, declare no preâmbulo do documento:

\begin{verbatim}
\usepackage[alf]{abntex2cite}	% Citações padrão ABNT
\end{verbatim}

A opção |alf| indica que as referências serão alfanuméricas, no padrão
autor-ano. Ela se opõe à opção |num|, que indica que as referências serão
numéricas. Consulte o manual do pacote \textsf{abntex2cite} para informações
detalhadas.

Para indicar o local de impressão da bibliografia, utilize:

\begin{verbatim}
\bibliography{arquivo-de-referencias-bib}
\end{verbatim}

Você pode usar tanto a classe \textsf{abntex2} quanto o pacote de citações
\textsf{abntex2cite} de forma independente. Usar apenas o pacote de citações é
útil quando se está escrevendo um documento baseado em outra classe fornecida.
Uma instituição de ensino, por exemplo, pode se utilizar deste recurso caso se
deseje manter apenas o padrão de citações. Por outro lado, também se pode
optar por utilizar as customizações da classe \textsf{abntex2} com outro
padrão de referências bibliográficas.

\DescribeEnv{backref}
O \textsf{abntex2cite} é compatível com o pacote
\textsf{backref}\footnote{\url{http://www.ctan.org/pkg/backref}}, que permite
que a bibliografia indique quantas vezes e em quais páginas a citação ocorreu.
Para isso, adicione ao preâmbulo:

\begin{verbatim}
\usepackage[brazilian,hyperpageref]{backref}	 
\end{verbatim}

Ainda no preâmbulo, você pode configurar como o pacote \textsf{backref} deverá
imprimir as referências:

\begin{verbatim}
% Configurações do pacote backref
% Usado sem a opção hyperpageref de backref
\renewcommand{\backrefpagesname}{Citado na(s) página(s):~}

% Texto padrão antes do número das páginas
\renewcommand{\backref}{}

% Define os textos da citação
\renewcommand*{\backrefalt}[4]{
	  \ifcase #1 %
		 Nenhuma citação no texto.%
	  \or
		 Citado na página #2.%
	  \else
		 Citado #1 vezes nas páginas #2.%
	  \fi}%
\end{verbatim}

% ---
\subsection{Glossário (opcional)}\label{sec-glossarios}
% ---

O \abnTeX2 não traz uma implementação própria para o Glossário, elemento
opcional estabelecido pela ABNT NBR 14724:2011. Um dos motivos da não inclusão
desse recurso é a existência de diversos pacotes que o fazem, cada um com uma
característica diferente.

Como sugestão, consulte o pacote
\textsf{glossaries}\footnote{\url{http://www.ctan.org/tex-archive/macros/latex/contrib/glossaries}},
que tanto pode construir Glossários como a Lista de símbolos
(\autoref{sec-listadesimbolos}). 

O portal \emph{\LaTeX Community} (\url{http://www.latex-community.org}) possui
um guia de uso do pacote
\textsf{glossaries}\footnote{\url{http://www.latex-community.org/know-how/latex/55-latex-general/263-glossaries-nomenclature-lists-of-symbols-and-acronyms}q}
e também um excelente
artigo\footnote{\url{http://www.latex-community.org/know-how/456-glossary-without-makeindex}}
que mostra como criar um pacote próprio de gerenciamento de glossários que não
necessita de nenhum utilitário externo, uma vez que \textsf{glossaries} requer
os aplicativos \textsf{makeindex} e \textsf{makeglossaries}, este último escrito
e dependente do Perl\footnote{\url{http://www.perl.org/}}. Já o livro 
``LaTeX''\footnote{\url{http://en.wikibooks.org/wiki/LaTeX/Glossary}} da
WikiBooks possui um capítulo dedicado à construção de glossários com
\textsf{glossaries}.

Como exemplo de uso do pacote \textsf{glossaries}, a partir da versão 1.6 o
\abnTeX2 inclui o documento ``Exemplo de uso de glossário com abnTeX2''
\cite{abntex2modelo-glossario}, que pode ser usado como referência para criação
desse elemento pós textual.

% ---
\subsection{Apêndices (opcional)}
% ---

\DescribeMacro{\apendices}\DescribeEnv{apendicesenv}
O início dos apêndices, elementos opcionais da ABNT NBR 14724:2011, seção
4.2.3.3, deve ser marcado com a macro |\apendices|, ou os apêndices devem estar
contidos no ambiente |apendicesenv|. Os apêndices devem preceder os anexos, caso
esses existam.

\DescribeMacro{\apendicename}
Os apêndices devem ser iniciados com a macro |\chapter|\marg{nome do apêndice},
que imprime o nome do apêndice precedido do conteúdo da macro |\apendicename|,
cujo conteúdo padrão é |AP\^ENDICE|.

\DescribeMacro{\apendicesname}\DescribeMacro{\partpage}\DescribeMacro{\partapendices}
No contexto dos apêndices, a macro |\partpage| e seu sinônimo |\partapendices|
imprimem o conteúdo da macro |\apendicesname| como se fosse uma divisão de
partes obtida com |\part| não numerada. A variante |\partapendices*| não inclui
divisão no Sumário nem no \textsf{bookmark} do PDF.

% ---
\subsection{Anexos (opcional)}
% ---

\DescribeMacro{\anexos}\DescribeEnv{anexosenv}
O início dos anexos, elementos opcionais da ABNT NBR 14724:2011, seção
4.2.3.4, deve ser marcado com a macro |\anexos|, ou os anexos devem estar
contidos no ambiente |anexosenv|. Os anexos devem vir dispostos após os
apêndices, caso esses existam.

\DescribeMacro{\anexoname}
Os anexos devem ser iniciados com a macro |\chapter|\marg{nome do anexo},
que imprime o nome do anexo precedido do conteúdo da macro |\anexoname|, cujo
conteúdo padrão é |ANEXO|.

\DescribeMacro{\anexosname}\DescribeMacro{\partpage}\DescribeMacro{\partanexos}
No contexto dos anexos, a macro |\partpage| e seu sinônimo |\partanexos|
imprimem o conteúdo da macro |\anexosname| como se fosse uma divisão de partes
obtida com |\part| não numerada. A variante |\partanexos*| não inclui a divisão
no Sumário nem no \textsf{bookmark} do PDF.

% ---
\subsection{Índice (opcional)}
% ---

\DescribeMacro{\printindex}\DescribeMacro{\index}
O índice, elemento opcional da ABNT NBR 14724:2011, deve ser elabora conforme a
ABNT NBR 6034 e pode ser produzido por meio da macro |\printindex|, que imprime
as páginas nas quais as macros |\index|\marg{palavra a ser indexada} apareceram.

Para que as macros |\printindex| e |\index| funcionem, é preciso utilizar o
compilador
\textsf{MakeIndex}\footnote{\url{http://www.tex.ac.uk/ctan/indexing/makeindex/}}.

% ------
\section{Mais informações}
% ------

Para mais informações, consulte o site do projeto em \abnTeXSite. Há dezenas de
páginas de wiki que podem esclarecer suas dúvidas. 

Faça parte também do fórum de discussão do \abnTeX2 em \abnTeXForum.


% ------
\addcontentsline{toc}{section}{Referências}
\bibliography{abntex2-doc}
% ------


% ------
\PrintChanges
\PrintIndex
% ------

% ------
\StopEventually{\PrintIndex}
% ------

\StopEventually{\PrintIndex}
\end{document}
