%% abntex2cite-alf.tex, v<VERSION> laurocesar
%% Copyright 2012-2013 by abnTeX2 group at http://code.google.com/p/abntex2/ 
%%
%% This work may be distributed and/or modified under the
%% conditions of the LaTeX Project Public License, either version 1.3
%% of this license or (at your option) any later version.
%% The latest version of this license is in
%%   http://www.latex-project.org/lppl.txt
%% and version 1.3 or later is part of all distributions of LaTeX
%% version 2005/12/01 or later.
%%
%% This work has the LPPL maintenance status `maintained'.
%% 
%% The Current Maintainer of this work is the abnTeX2 team, led
%% by Lauro César Araujo. Further information are available on 
%% http://code.google.com/p/abntex2/
%%
%% Creator and original mantainer: Gerald Weber <gweber@codigolivre.org.br>
%% Copyright 2001-2003 by the abnTeX group at http://codigolivre.org.br/projects/abntex
%%
%%
%% command line for latex2html :
%% latex2html -split 0 -toc_depth 8 -noinfo -noaddress -nonavigation -show_section_numbers -html_version 4.0 abnt-bibtex-alf-doc.tex
%%
%% 2013/01/18 21h21	laurocesar
%%  Include fonte lmodern
%%
%% 2013/01/12 21h21	laurocesar
%%  Revisão do documento
%%
%% 2012/12/20 09h20 	laurocesar
%%  Alteração do nome dos arquivos BIB de referências
%%  Revisão da largura de algumas tabelas
%%
%% 2012/12/19 09h21 	laurocesar
%%  Fork of the version v 1.29 2004/03/23 17:35:47   mvsfrasson
%%  Alteração do nome do documento de abnt-bibtex-alf-doc.tex para
% % abntex2cite-alf.tex
%%  Alteração da classe do documento de abnt para ltxdoc
%%  Revisão dos captions das tabelas
%%  Revisão do documento, inclusão das seções ``Escopo'' e ``Créditos aos
%%  autores originais''
%%  Substituição das referências abntcite para abntex2cite
%%
%% 2012/12/13 10h49 	laurocesar
%%  Remoção da referência a abnt-options em \bibliography{}
%%  Alteração da codificação de \usepackage[latin1]{inputenc} para utf8
%%
%% $Log: abnt-bibtex-alf-doc.tex,v $
%% Revision 1.29  2004/03/23 17:35:47  mvsfrasson
%%
%% Arquivos atualizados para nao necessitarem pacote html
%%
%% Revision 1.28  2003/06/30 14:42:50  gweber
%% Pequenas modificações, início da adaptação para 10522/2002.
%%
%% Revision 1.27  2003/06/19 09:32:18  gweber
%% Inicio da adaptação para 10520/2002.
%%
%% Revision 1.26  2003/04/08 00:40:44  gweber
%% novos comandos \apud e entrada @hidden
%%
%% Revision 1.25  2003/03/28 20:54:20  gweber
%% ajustes nas citações às normas
%%
%% Revision 1.24  2003/03/25 14:17:35  gweber
%% alterações nos arquivos bib
%%
%%

\documentclass[a4paper]{ltxdoc}
\usepackage{lmodern}			% Usa a fonte Latin Modern			
\usepackage[T1]{fontenc}		% seleção de códigos de fonte.
\usepackage[utf8]{inputenc}
\usepackage[brazil]{babel}
\usepackage{hyperref}
\usepackage{parskip}			% espaçamento entre os parágrafos

\IfFileExists{html.sty}
{\usepackage{html}}
{\usepackage{comment}
 \excludecomment{htmlonly}
 \excludecomment{rawhtml}
 \includecomment{latexonly}
 \newcommand{\html}[1]{}
 \newcommand{\latex}[1]{##1}
 \ifx\undefined\hyperref
  \ifx\pdfoutput\undefined \let\pdfunknown\relax
   \let\htmlATnew=\newcommand
  \else
   \ifx\pdfoutput\relax \let\pdfunknown\relax
    \usepackage{hyperref}\let\htmlATnew=\renewcommand
   \else
    \usepackage{hyperref}\let\htmlATnew=\newcommand
   \fi
  \fi
 \else
  \let\htmlATnew=\renewcommand
 \fi
 \ifx\pdfunknown\relax
  \htmlATnew{\htmladdnormallink}[2]{##1}
 \else
  \def\htmladdnormallink##1##2{\href{##2}{##1}}
 \fi
 \long\def\latexhtml##1##2{##1}}

%\usepackage{showtags}
\usepackage[alf,abnt-repeated-title-omit=yes,abnt-show-options=warn,abnt-verbatim-entry=yes]{abntex2cite}
\def\Versao$#1 #2${#2}
\def\Data$#1 #2 #3${#2}

\ifx\pdfoutput\undefined
\else
%from /usr/share/texmf/doc/pdftex/base/pdftexman.pdf
\pdfinfo{
  /Title        (Manual de uso do pacote abntex2cite: tópicos específicos da ABNT
                 NBR 10520:2002 e o estilo bibliográfico alfabético (sistema
                 autor-data))
  /Author       (G. Weber - Grupo abnTeX - Grupo abnTeX2) 
  /Subject      (referencias bibliograficas) 
  /Keywords     (ABNT, bibliografia, 6023:2000, 6023:2002, 10520:1988, 10520:2001, 10520:2002)}
\fi

\newcommand{\VerbL}{0.54\textwidth}
\newcommand{\LatL}{0.45\textwidth}
\begin{document}
%se for usar citeoption não esqueça de comentar o \nocite{*} no final
\nocite{7.2.2-2}
\nocite{7.3.2-3} %para testar format.author.or.organization
\nocite{7.5.1.2-1}
\nocite{8.11.3-1} %para testar reprinted-from

\newcommand{\titulo}{Manual de uso do pacote \textsf{abntex2cite}: tópicos específicos da ABNT
NBR 10520:2002 e o estilo bibliográfico alfabético (sistema autor-data)}

\newcommand{\abnTeX}{abn\TeX}

\title{\titulo}
\author{Equipe \abnTeX2\\Lauro César Araujo}
\date{\today, v<VERSION>}

\maketitle

% \begin{htmlonly}
% \title{\titulo}
% \author{Equipe \abnTeX2}
% \date{\Data$Date: 2004/03/23 17:35:47 $}
% \maketitle
% \end{htmlonly}

\begin{abstract}
Este manual é parte integrante da suíte \abnTeX2 e aborda tópicos específicos
da ABNT NBR 10520:2002 e o estilo bibliográfico alfabético (sistema
autor-data).
\end{abstract}


\tableofcontents

\listoftables

\section{Escopo}

O objetivo deste manual é abordar tópicos específicos da ABNT NBR 10520:2002 e
o estilo bibliográfico alfabético (sistema autor-data). 

Este documento é parte integrante da suíte \abnTeX2 e é um complemento do
documento \citeonline{abntex2cite}.

Para referências à classe \textsf{abntex2}, responsável pela aparência e
estrutura de trabalhos acadêmicos, consulte \citeonline{abntex2classe}.

\section{Créditos aos autores originais}

Este documento é derivado do documento \emph{Estilo bibtex compatível com a
`norma' 6023 da ABNT: Questões específicas da `norma' 10520}, versão $ 1.29
$, escrito por G. Weber e atualizado por Miguel V. S. Frasson em 23 de março de
2004 no âmbito do projeto \abnTeX\ hospedado em
\url{http://codigolivre.org.br/projects/abntex}\footnote{O documento original pode ser lido em
\url{http://code.google.com/p/abntex2/downloads/detail?name=abnt-bibtex-doc.pdf}}.

A equipe do projeto \abnTeX2, observando a \emph{LaTeX Project Public License},
registra que os créditos deste manual são dos autores originais.


\section{Considerações iniciais}

\subsection{Prefira o sistema numérico, se possível}

Se você está usando \LaTeX\, assumimos que está usando um computador. Nesse
caso, a citação no esquema autor-data é uma coisa tão arcaica e desnecessária
quanto uma régua-de-cálculo. A leitura de um texto com chamadas no sistema
autor-data é desajeitada, especialmente quando muitas referências são chamadas
de uma só vez. Você não tem absolutamente nenhum motivo para usar o sistema
autor-data já que a ABNT ``autorizou'' o uso do sistema citação numérico
por um cochilo qualquer.

Por outro lado, há aqueles que se dão ao trabalho de te dar trabalho, de seguir
regras esdrúxulas na hora de escrever teses. Estes podem estar exigindo que você
use o arcaico sistema autor-data. Também pode acontecer que os examinadores da
banca, seja por que não leram a sua tese ou porque não entenderam nada
realmente (ou ambos), acabem se enroscando em picuinhas.

Aí mandam você colocar todas as suas referências numéricas em formato
autor-data. Comentamos isso porque tudo isso já aconteceu e continua acontecendo 
com outras pessoas. Portanto, \emph{esteja preparado}. 

Antes de começar a escrever recomendamos verificar quais são as regras \emph{de
fato} da sua universidade ou instituição.

\subsection{Cuidado: estilo em fase de acabamento!}

O estilo \LaTeX\ {\tt abntex2cite.sty} está em fase final de desenvolvimento.
Embora já apresente alguma maturidade sua implementação pode mudar entre uma
versão e outra e é possível que seu texto necessite de algumas pequenas
adaptações futuramente.

Além disso, o estilo foi desenvolvido com base na ABNT NBR 6023:2000, e a versão
corrente da norma é a ABNT NBR 6023:2002. Desse modo, considere inconsistentes
as referências bibliográficas produzidas pelo \abnTeX2 que sofreram alterações
entre a versão de 2000 e 2002 da norma.


\subsection{Cuidado: normas nebulosas!}

Nós elaboramos os estilos debruçados diretamente sobre os originais da ABNT
e as seguimos escrupulosamente. Mas não se iluda! O que a sua coordenação
de pós-graduação, ou orientador, ou chefe etc., entendem por `normas ABNT'
pode não ter qualquer vínculo com a realidade. Por isso \emph{não garantimos}
que ao usar os estilos do \abnTeX\ você esteja em conformidade com as normas
da sua instituição ou empresa.

Um exemplo clássico é o sistema de chamada numérico (citação numérica, usada em
\cite{abntex2cite}). Embora \emph{todas} as `normas' da ABNT
\cite{NBR10520:1988,NBR6023:2000,NBR6023:2002,NBR10520:2001,NBR10520:2002}
`autorizem' seu uso a maioria das instituições, orientadores, membros de banca
dirão que isso não é ABNT. Acredite se quiser.

\section{Como usar a citação alfabética (sistema autor-data)}

\subsection{Como chamar os pacotes e estilos}

\DescribeMacro{\usepackage[alf]{abntcite}}
Para usar o estilo de citação alfabético (seção 9.2 em
6023/2000~\cite{NBR6023:2000}) utilize no preâmbulo:

\begin{verbatim}
    \usepackage[alf]{abntcite}
\end{verbatim}

\DescribeMacro{\bibliographystyle}
Você \emph{não precisa} selecionar o comando \verb+\bibliographystyle+, porque
isso é realizado automaticamente. Do mesom modo, a base bibliográfica
\verb+abntex2-options.bib+ também estará automaticamente incluída.

Se você quiser usar apenas o estilo bibliográfico, e não utilizar o esquema de
citações, selecione com o comando:

\begin{verbatim}
    \bibliographystyle{abnt-alf}
\end{verbatim}

mas esteja avisado que nesse caso as chamadas no texto \emph{não estarão}
em conformidade com a `norma'.

Dessa forma, as citações no texto não estarão em conformidade com as normas
brasileiras, porém a bibliografia será formatada corretamente.

\subsection{Como usar a citação no sistema autor-data}

Este sistema de citação é descrito em \citeonline{NBR10520:2002}. Note que
essa `norma' entrou em vigor em agosto de 2002, apenas um ano depois da
\citeonline{NBR10520:2001}.

Esse sistema aceita basicamente duas formatações para a citação, dependendo da
maneira como ocorre a citação no texto.

\subsubsection{Referências explícitas ({\tt citeonline}) }

\DescribeMacro{\citeonline}
É o caso em que você menciona \emph{explicitamente} o autor da referência na sentença. Algo
como ``Fulano (1900)". Para isso use o comando \verb+\citeonline+.

Exemplos apresentadas em \citeonline{NBR10520:2001}:

%\def\Exemplo#1#2{\noindent\begin{minipage}[t]{8cm}\small#1%
%\end{minipage}\begin{minipage}[t]{9cm}\small#2%
%\end{minipage}\vspace{5mm}}

\vspace{3mm}
\noindent\begin{minipage}[t]{\VerbL}\latex{\small}\begin{verbatim}
A ironia será assim uma \ldots\ proposta
por \citeonline{10520:4.1-1}.
\end{verbatim}\end{minipage}\begin{minipage}[t]{\LatL}\latex{\small}
A ironia será assim uma \ldots\ proposta 
por \citeonline{10520:4.1-1}.
\end{minipage}\vspace{5mm}\\

\noindent\begin{minipage}[t]{\VerbL}\latex{\small}\begin{verbatim}
\citeonline[p.~146]{10520:4.2-2}
dizem que \ldots\ 
\end{verbatim}\end{minipage}\begin{minipage}[t]{\LatL}\latex{\small}
\citeonline[p.~146]{10520:4.2-2} dizem que \latexhtml{\ldots\ }{...}
\end{minipage}\vspace{5mm}\\

\noindent\begin{minipage}[t]{\VerbL}\latex{\small}\begin{verbatim}
Segundo \citeonline[p.~27]{10520:4.2-3}
\ldots\
\end{verbatim}\end{minipage}\begin{minipage}[t]{\LatL}\latex{\small}
Segundo \citeonline[p.~27]{10520:4.2-3}  \latexhtml{\ldots\ }{...}
\end{minipage}\vspace{5mm}\\

\noindent\begin{minipage}[t]{\VerbL}\latex{\small}\begin{verbatim}
Aqui vai um texto qualquer só para\\
poder citar alguém
\cite[p.~225]{10520-2002:6.3-1}.
\end{verbatim}\end{minipage}\begin{minipage}[t]{\LatL}\latex{\small}
Aqui vai um texto qualquer só para poder
citar alguém
\cite[p.~225]{10520-2002:6.3-1}.
\end{minipage}\vspace{5mm}\\

\noindent\begin{minipage}[t]{\VerbL}\latex{\small}\begin{verbatim}
\citeonline[p.~35]{10520-2002:6.3-2}
apresenta uma série
de coisas incompreensíveis.
\end{verbatim}\end{minipage}\begin{minipage}[t]{\LatL}\latex{\small}
\citeonline[p.~35]{10520-2002:6.3-2}
apresenta uma série
de coisas incompreensíveis.
\end{minipage}\vspace{5mm}\\

\noindent\begin{minipage}[t]{\VerbL}\latex{\small}\begin{verbatim}
Mais um exemplo articifial
mas necessário de citação
\cite[p.~3]{10520-2002:6.3-3}.
\end{verbatim}\end{minipage}\begin{minipage}[t]{\LatL}\latex{\small}
Mais um exemplo articifial
mas necessário de citação
\cite[p.~3]{10520-2002:6.3-3}.
\end{minipage}\vspace{5mm}\\

\noindent\begin{minipage}[t]{\VerbL}\latex{\small}\begin{verbatim}
\citeonline{10520-2002:6.3-4}
são mais um exemplo de citação
bem bacana.
\end{verbatim}\end{minipage}\begin{minipage}[t]{\LatL}\latex{\small}
\citeonline{10520-2002:6.3-4}
são mais um exemplo de citação
bem bacana.
\end{minipage}\vspace{5mm}\\

\noindent\begin{minipage}[t]{\VerbL}\latex{\small}\begin{verbatim}
Exemplo que mostra alguma
coisa também
\cite[p.~34]{10520-2002:6.3-5}.
\end{verbatim}\end{minipage}\begin{minipage}[t]{\LatL}\latex{\small}
Exemplo que mostra alguma
coisa também
\cite[p.~34]{10520-2002:6.3-5}.
\end{minipage}\vspace{5mm}\\

\subsubsection{Referências a {\tt organization}}

\DescribeEnv{organization}
Se a autoria do texto for do tipo {\tt organization}, você pode usar as citações
normalmente. No entanto, é recomendado que, para organizações com descrição
longa, você controle o que aparece no texto com o campo {\tt org-short} do
arquivo {\tt .bib}. No exemplo seguinte temos:

\begin{verbatim}
     organization={Comiss\~ao das Comunidades Europ\'eias},
\end{verbatim}

\noindent\begin{minipage}[t]{\VerbL}\latex{\small}\begin{verbatim}
Exemplo que mostra alguma
coisa também
\cite[p.~34]{10520-2002:6.3-5}.
\end{verbatim}\end{minipage}\begin{minipage}[t]{\LatL}\latex{\small}
Exemplo que mostra alguma
coisa também
\cite[p.~34]{10520-2002:6.3-5}.
\end{minipage}\vspace{5mm}\\

Já neste exemplo:
\begin{verbatim}
 organization={Brasil. {Ministério da Administração Federal
 e da Reforma do Estado}},
 org-short   ={Brasil},
\end{verbatim}

\noindent\begin{minipage}[t]{\VerbL}\latex{\small}\begin{verbatim}
E a pátria não poderia faltar
\cite{10520-2002:6.3-6}.
\end{verbatim}\end{minipage}\begin{minipage}[t]{\LatL}\latex{\small}
E a pátria não poderia faltar
\cite{10520-2002:6.3-6}.
\end{minipage}\vspace{5mm}\\

\subsubsection{Citações com autoria dada pelo título}

\noindent\begin{minipage}[t]{\VerbL}\latex{\small}\begin{verbatim}
Uma lei anônima
\cite[p.~55]{10520-2002:6.3-7}.
\end{verbatim}\end{minipage}\begin{minipage}[t]{\LatL}\latex{\small}
Uma lei anônima
\cite[p.~55]{10520-2002:6.3-7}.
\end{minipage}\vspace{5mm}\\

\noindent\begin{minipage}[t]{\VerbL}\latex{\small}\begin{verbatim}
Uma artigo anônimo
\cite[p.~4]{10520-2002:6.3-8}.
\end{verbatim}\end{minipage}\begin{minipage}[t]{\LatL}\latex{\small}
Uma artigo anônimo
\cite[p.~4]{10520-2002:6.3-8}.
\end{minipage}\vspace{5mm}\\

\noindent\begin{minipage}[t]{\VerbL}\latex{\small}\begin{verbatim}
Outro artigo anônimo
\cite[p.~12]{10520-2002:6.3-9}.
\end{verbatim}\end{minipage}\begin{minipage}[t]{\LatL}\latex{\small}
Outro artigo anônimo
\cite[p.~12]{10520-2002:6.3-9}.
\end{minipage}\vspace{5mm}\\

\subsubsection{Referências implícitas ({\tt cite})}

\DescribeMacro{\cite}
São aquelas referências que não fazem parte do texto. Use com \verb+\cite+.

Exemplos:

\vspace{5mm}
\noindent\begin{minipage}[t]{\VerbL}\latex{\small}\begin{verbatim}
para que \ldots\  à sociedade
\cite[p.~46, grifo nosso]{10520:4.8}  
\ldots\
\end{verbatim}\end{minipage}\begin{minipage}[t]{\LatL}\latex{\small}
para que \ldots\ à sociedade 
\cite[p.~46, grifo nosso]{10520:4.8}
\ldots\
\end{minipage}\vspace{5mm}\\

\noindent\begin{minipage}[t]{\VerbL}\latex{\small}\begin{verbatim}
b) desejo de \ldots\  passado colonial
\cite[v.~2, p.~12, grifo do autor]%
{10520:4.8.1} \ldots\
\end{verbatim}\end{minipage}\begin{minipage}[t]{\LatL}\latex{\small}
b) desejo de \ldots\  passado colonial
\cite[v.~2, p.~12, grifo do autor]%
{10520:4.8.1} \ldots\
\end{minipage}\vspace{5mm}\\

\noindent\begin{minipage}[t]{\VerbL}\latex{\small}\begin{verbatim}
``Apesar das \ldots\  da filosofia''
\cite[p.~293]{10520:4.1-2}.
\end{verbatim}\end{minipage}\begin{minipage}[t]{\LatL}\latex{\small}
``Apesar das \latexhtml{\ldots\ }{...} da filosofia'' \cite[p.~293]{10520:4.1-2}.
\end{minipage}\vspace{5mm}\\

\noindent\begin{minipage}[t]{\VerbL}\latex{\small}\begin{verbatim}
Depois, \ldots\  que prefiro
\cite[p.~101--114]{10520:4.1-3}.
\end{verbatim}\end{minipage}\begin{minipage}[t]{\LatL}\latex{\small}
Depois, \latexhtml{\ldots\ }{...} que prefiro \cite[p.~101--114]{10520:4.1-3}.
\end{minipage}\vspace{5mm}\\

\noindent\begin{minipage}[t]{\VerbL}\latex{\small}\begin{verbatim}
A produção de \ldots\  em 1928
\cite[p.~513]{10520:4.2-1} .
\end{verbatim}\end{minipage}\begin{minipage}[t]{\LatL}\latex{\small}
A produção de \latexhtml{\ldots\ }{...} em 1928 \cite[p.~513]{10520:4.2-1} .
\end{minipage}\vspace{5mm}\\


\subsubsection{Referências citadas por outra referência ({\tt apud})}

\DescribeMacro{\apud}\DescribeMacro{\apudeonline}
Para citar uma referência que é citada dentro de outra referência, pode-se
usar as macros \verb+\apud+ e \verb+\apudonline+, como nos exemplos abaixo.

\noindent\begin{minipage}[t]{\VerbL}\latex{\small}\begin{verbatim}
\apud[p.~2--3]{Sage}{Evans}
\end{verbatim}\end{minipage}\begin{minipage}[t]{\LatL}\latex{\small}
\apud[p.~2--3]{Sage}{Evans}
\end{minipage}

\noindent\begin{minipage}[t]{\VerbL}\latex{\small}\begin{verbatim}
Segundo \apudonline[p.~2]{Silva}{Abreu}
\end{verbatim}\end{minipage}\begin{minipage}[t]{\LatL}\latex{\small}
Segundo \apudonline[p.~2]{Silva}{Abreu}
\end{minipage}

Nesses casos, todas as referências aparecerão na lista de referências, exceto
aquelas definidas como entrada {\tt @hidden}. Ou seja, se você não quiser
que uma entrada apareça na lista de referências, você deve defini-la como
{\tt @hidden} na sua base bibliográfica.

\subsubsection{Uso de idem, ibidem, opus citatum e outros}

Implementamos comandos específicos para \emph{idem} (mesmo autor),
\emph{ibidem} (mesma obra), \emph{opus citatum} (obra citada), \emph{passim}
(aqui e alí), \emph{loco citato} (no lugar citado), \emph{confira e et
sequentia} (e sequência). Veja uso desses tipos de citações nos exemplos abaixo.

\DescribeMacro{\Idem}
\noindent\begin{minipage}[t]{\VerbL}\latex{\small}\begin{verbatim}
\Idem[p.~2]{NBR6023:2000}
\end{verbatim}\end{minipage}\begin{minipage}[t]{\LatL}\latex{\small}
\Idem[p.~2]{NBR6023:2000}
\end{minipage}

\DescribeMacro{\Ibidem}
\noindent\begin{minipage}[t]{\VerbL}\latex{\small}\begin{verbatim}
\Ibidem[p.~2]{NBR6023:2000}
\end{verbatim}\end{minipage}\begin{minipage}[t]{\LatL}\latex{\small}
\Ibidem[p.~2]{NBR6023:2000}
\end{minipage}

\DescribeMacro{\opcit}
\noindent\begin{minipage}[t]{\VerbL}\latex{\small}\begin{verbatim}
\opcit[p.~2]{NBR6023:2000}
\end{verbatim}\end{minipage}\begin{minipage}[t]{\LatL}\latex{\small}
\opcit[p.~2]{NBR6023:2000}
\end{minipage}

\DescribeMacro{\passim}
\noindent\begin{minipage}[t]{\VerbL}\latex{\small}\begin{verbatim}
\passim{NBR6023:2000}
\end{verbatim}\end{minipage}\begin{minipage}[t]{\LatL}\latex{\small}
\passim{NBR6023:2000}
\end{minipage}

\DescribeMacro{\loccit}
\noindent\begin{minipage}[t]{\VerbL}\latex{\small}\begin{verbatim}
\loccit{NBR6023:2000}
\end{verbatim}\end{minipage}\begin{minipage}[t]{\LatL}\latex{\small}
\loccit{NBR6023:2000}
\end{minipage}

\DescribeMacro{\cfcite}
\noindent\begin{minipage}[t]{\VerbL}\latex{\small}\begin{verbatim}
\cfcite[p.~2]{NBR6023:2000}
\end{verbatim}\end{minipage}\begin{minipage}[t]{\LatL}\latex{\small}
\cfcite[p.~2]{NBR6023:2000}
\end{minipage}

\DescribeMacro{\etseq}
\noindent\begin{minipage}[t]{\VerbL}\latex{\small}\begin{verbatim}
\etseq[p.~2]{NBR6023:2000}
\end{verbatim}\end{minipage}\begin{minipage}[t]{\LatL}\latex{\small}
\etseq[p.~2]{NBR6023:2000}
\end{minipage}

Note que esses comandos só fazem sentido para citações a uma única referência
por vez. Segundo o ítem 6.1.3 da `norma' \cite{NBR10520:2001}, essas expressões
só devem ser usadas em notas, não no texto. Apenas a expressão \emph{apud}
poderia ser usada no texto.

\subsubsection{Referências múltiplas}
\label{mult-ref}

\DescribeMacro{\cite}\DescribeMacro{\citeonline}
Frequentemente queremos citar diversas referências de uma vez. Para isso, use
|\cite|\marg{key1,key2,\ldots,keyN} ou
|\citeonline|\marg{key1,key2,\ldots,keyN}, separando cada chave por uma vírgula
sem espaços e sem mudança de linha.

\vspace{5mm}
\noindent\begin{minipage}[t]{\VerbL}\latex{\small}\begin{verbatim}
\cite{10520:5.1.4-98,10520:5.1.4-99,%
10520:5.1.4-00}
\end{verbatim}\end{minipage}\begin{minipage}[t]{\LatL}\latex{\small}
\cite{10520:5.1.4-98,10520:5.1.4-99,10520:5.1.4-00}
\end{minipage}\vspace{5mm}\\

\noindent\begin{minipage}[t]{\VerbL}\latex{\small}\begin{verbatim}
\citeonline{10520:5.1.4-98,%
10520:5.1.4-99,10520:5.1.4-00}
\end{verbatim}\end{minipage}\begin{minipage}[t]{\LatL}\latex{\small}
\citeonline{10520:5.1.4-98,10520:5.1.4-99,10520:5.1.4-00}
\end{minipage}\vspace{5mm}\\

\noindent\begin{minipage}[t]{\VerbL}\latex{\small}\begin{verbatim}
\cite{NBR10520:1988,NBR6023:2000,%
NBR10520:2001}
\end{verbatim}\end{minipage}\begin{minipage}[t]{\LatL}\latex{\small}
\cite{NBR10520:1988,NBR6023:2000,%
NBR10520:2001}
\end{minipage}\vspace{5mm}\\

\noindent\begin{minipage}[t]{\VerbL}\latex{\small}\begin{verbatim}
\citeonline{NBR10520:1988,NBR6023:2000,%
NBR10520:2001}
\end{verbatim}\end{minipage}\begin{minipage}[t]{\LatL}\latex{\small}
\citeonline{NBR10520:1988,NBR6023:2000,NBR10520:2001}
\end{minipage}\vspace{5mm}\\

\subsubsection{Múltiplas referências com mesmo autor-data}
\label{mult-abc}
Quando há várias referências com o mesmo autor-data, {\tt abnt-alf.sty}
gera automaticamente um `a', `b', etc., adicional. Isto atende ao ítem
5.1.3 da `norma' \cite{NBR10520:2001}.

\noindent\begin{minipage}[t]{\VerbL}\latex{\small}\begin{verbatim}
\cite{10520:5.1.3a,10520:5.1.3b}
\end{verbatim}\end{minipage}\begin{minipage}[t]{\LatL}\latex{\small}
\cite{10520:5.1.3a,10520:5.1.3b}
\end{minipage}\vspace{5mm}\\

\noindent\begin{minipage}[t]{\VerbL}\latex{\small}\begin{verbatim}
\citeonline{10520:5.1.3a,10520:5.1.3b}
\end{verbatim}\end{minipage}\begin{minipage}[t]{\LatL}\latex{\small}
\citeonline{10520:5.1.3a,10520:5.1.3b}
\end{minipage}\vspace{5mm}\\

\subsubsection{Mais de três autores e o uso de et al.}

A `norma' \cite{NBR10520:2001} é omissa sobre o caso de chamadas a obras com
mais de três autores. Mesmo entre os 204 exemplos da `norma'
\cite{NBR6023:2000}, ocorre apenas um único caso, usado nos exemplos a seguir.

\noindent\begin{minipage}[t]{\VerbL}\latex{\small}\begin{verbatim}
Segundo \citeonline{8.1.1.2}
entende-se que \ldots\ 
\end{verbatim}\end{minipage}\begin{minipage}[t]{\LatL}\latex{\small}
Segundo \citeonline{8.1.1.2}
entende-se que \latexhtml{\ldots\ }{...}
\end{minipage}\vspace{5mm}\\

\noindent\begin{minipage}[t]{\VerbL}\latex{\small}\begin{verbatim}
Em recente estudo \cite{Deng00}
foram investigados estados em 
fios quânticos.
\end{verbatim}\end{minipage}\begin{minipage}[t]{\LatL}\latex{\small}
Em recente estudo \cite{Deng00}
foram investigados estados em 
fios quânticos.
\end{minipage}\vspace{5mm}\\

Você pode altera esse comportamento usando as opções {\tt abnt-etal-cite},
{\tt abnt-etal-list} e {\tt abnt-etal-text}.
Veja mais informacões na \autoref{tabela-opcoes} e em \cite{abntex2cite}.

\subsubsection{Citações de partes da referência}

Às vezes pode ser necessário citar apenas uma parte da referência, como apenas o
autor ou apenas o ano. Para isso, foram implementados os comandos
|\citeauthoronline|, |\citeauthor| e |\citeyear|, exemplificados do seguinte
modo:

% subsubsubsection
%\subsubsection{Apenas o autor ({\tt citeauthoronline}), forma explícita}
\DescribeMacro{\citeauthoronline}
A macro |\citeauthoronline| deve ser utilizada para citar
apenas o autor de forma explícita:

\noindent\begin{minipage}[t]{\VerbL}\latex{\small}\begin{verbatim}
A produção de \ldots\ em 1928
mencionada por
\citeauthoronline{10520:4.2-1}.
\end{verbatim}\end{minipage}\begin{minipage}[t]{\LatL}\latex{\small}
A produção de \ldots\ em 1928  
mencionada por
\citeauthoronline{10520:4.2-1}.
\end{minipage}\vspace{5mm}\\

% subsubsubsection
%\subsubsection{Apenas o autor ({\tt citeauthor}), forma implícita}
\DescribeMacro{\citeauthor}
A macro |\citeauthor| deve ser utilizada para citar
apenas o autor de forma implícita:

\noindent\begin{minipage}[t]{\VerbL}\latex{\small}\begin{verbatim}
A produção de \ldots\  em 1928
(\citeauthor{10520:4.2-1} 1928).
\end{verbatim}\end{minipage}\begin{minipage}[t]{\LatL}\latex{\small}
A produção de \latexhtml{\ldots\ }{...} em 1928
(\citeauthor{10520:4.2-1} 1928).
\end{minipage}\vspace{5mm}\\

% subsubsubsection
%\subsubsection{Apenas o ano ({\tt citeyear})}

\DescribeMacro{\citeyear}
A macro |\citeyear| deve ser utilizada para citar
apenas o ano da publicação:

%\vspace{3mm}
\noindent\begin{minipage}[t]{\VerbL}\latex{\small}\begin{verbatim}
Em \citeyear{10520:4.1-1} a ironia
será assim uma \ldots\  proposta por.
\end{verbatim}\end{minipage}\begin{minipage}[t]{\LatL}\latex{\small}
Em \citeyear{10520:4.1-1} a ironia será assim uma \latexhtml{\ldots\ }{...} proposta por.
\end{minipage}\vspace{5mm}\\

\section{Alteração do estilo alfabético {\tt citeoption}}

\DescribeMacro{\citeoption}
Com o comando \verb+\citeoption+ ou \verb+\usepackage+ você pode alterar alguns comportamentos
do estilo bibliográfico. Na \autoref{tabela-opcoes}, listamos as opções que são
específicas do estilo {\tt abnt-alf}.

Por exemplo, para desativar a substituição dos autores por `et al.', você
incluir o pacote com a seguinte opção:

\begin{verbatim}
    \usepackage[alf,abnt-etal-cite=0]{abntcite}
\end{verbatim}

ou, em qualquer ponto do texto,

\begin{verbatim}
    \citeoption{abnt-etal-cite=0}
\end{verbatim}

Para ver as demais opções e o modo de uso veja o
documento \cite{abntex2cite}.

\begin{table}[htbp]
\caption[Opções de alteração dos estilos bibliográficos: composição]{
Opções de alteração da composição dos estilos bibliográficos.}
\label{tabela-opcoes}

\begin{center}
\begin{tabular}{lrp{7cm}}\hline\hline
campo & opções & descrição \\ \hline
\emph{abnt-and-type} & & determina de que maneira é formatado o \emph{and}.\\
{\tt abnt-and-type=e} & \underline{\tt e}& Usa `e' como em `Fonseca e Paiva'.\\
{\tt abnt-and-type=\&} & {\tt \&} & Usa `\&' como em `Fonseca \& Paiva'.
\\ \hline
\emph{abnt-etal-cite} &  & controla como e quando os co-autores são
substituídos por \emph{et al.}.  Note que a substituição
por \emph{et al.} continua ocorrendo \emph{sempre} se os co-autores tiverem sido indicados
como {\tt others}.\\
{\tt abnt-etal-cite=0}&{\tt 0}& não abrevia a lista de autores.\\
{\tt abnt-etal-cite=2}& {\tt 2} & abrevia com mais de 2 autores.\\
{\tt abnt-etal-cite=3}& \underline{\tt 3} & abrevia com mais de 2 autores.\\
$\vdots$ & $\vdots$ & \\
{\tt abnt-etal-cite=5}& {\tt 5} & abrevia com mais de 5 autores.
\\ \hline\hline
\end{tabular}
\end{center}
\end{table}


\section{Requisitos ainda não implementados}

\subsection{Autores com mesmo sobrenome e nome diferente}

A `norma' \cite{NBR10520:2001}, no seu ítem 5.1.2, sugere que se coloque o nome
por extenso quando o sobrenome+data forem iguais. Isso em geral não é possível
pois raramente se conhece o primeiro nome de um autor. Em todo caso, veja a
\autoref{mult-abc} para ver como é resolvido o caso de autor-data igual.

\subsection{Referências em notas de rodapé}

Não há intenção, a curto prazo, em implementar referências em notas de rodapé
conforme sugerido no ítem 6 da `norma' 10520.

\section{Problemas com a `norma' 10520}

Como já foi dito, a `norma' 10520 \cite{NBR10520:2001} é incompleta e
inconsistente. Aqui apresentamos uma descrição dos principais problemas
encontrados e como foram resolvidos, quando foi possível.

\subsection{Referências múltiplas}

As referências a vários autores simultaneamente, ou várias obras de um mesmo
autor, são um verdadeiro mistério na ``norma'' \cite{NBR10520:2001}. No exemplo
da seção 5.1.4, diz a ``norma'''

\begin{quote}
As citações de diversos documentos de um mesmo autor, publicados em anos diferentes
e mencionados simultaneamente, têm as suas datas separadas por vírgula.\\
Exemplo: (CRUZ; CORREA; COSTA; 1998, 1999, 2000)\footnote{Transcrição exata. Esta
chamada não foi gerada automaticamente.}
\end{quote}

Podemos então entender que Cruz, Correa e Costa são os três autores e 1998, 1999
e 2000 são os três anos de publicação referindo-se a três documentos separados?

Antes de responder sim, veja o que diz a seção seguinte, 5.1.5:

\begin{quote}
As citações de diversos documentos de vários autores, mencionados simultaneamente,
devem ser separadas por ponto e vírgula.\\
Exemplo: (FONSECA; PAIVA; SILVA, 1997)\footnote{Transcrição exata. Esta
chamada não foi gerada automaticamente.}
\end{quote}

E agora? Trata-se de três autores separados, Fonseca, Paiva, Silva? Ou um
autor Fonseca e outro documento com dois autores, Paiva e Silva? Ou um
documento com dois autores Fonseca e Paiva seguido de um autor Silva?
Quem não lê a descrição da seção 5.1.5, pode-se inclusive achar que trata-se
de um único documento. Ou o exemplo da seção 5.1.5 está errado: faltam datas aos
outros autores, ou foi uma tentativa desastrada de agrupar três documentos com
o mesmo ano de publicação. Já exemplo anterior, da seção 5.1.4, há problemas
também com um ponto-e-vírgula após o COSTA, quando deveria haver apenas uma
vírgula. 

Veja na \autoref{mult-ref} como fica a nossa solução.

Exemplo da seção 5.1.5:\\
\noindent\begin{minipage}[t]{\VerbL}\latex{\small}\begin{verbatim}
\cite{10520:5.1.5a,10520:5.1.5b,%
10520:5.1.5c}
\end{verbatim}\end{minipage}\begin{minipage}[t]{\LatL}\latex{\small}
\cite{10520:5.1.5a,10520:5.1.5b,10520:5.1.5c}
\end{minipage}\vspace{5mm}\\
\noindent\begin{minipage}[t]{\VerbL}\latex{\small}\begin{verbatim}
\citeonline{10520:5.1.5a,10520:5.1.5b,%
10520:5.1.5c}
\end{verbatim}\end{minipage}\begin{minipage}[t]{\LatL}\latex{\small}
\citeonline{10520:5.1.5a,10520:5.1.5b,10520:5.1.5c}
\end{minipage}\vspace{5mm}\\

ou, supondo que Fonseca e Paiva sejam co-autores,\\
\noindent\begin{minipage}[t]{\VerbL}\latex{\small}\begin{verbatim}
\cite{10520:5.1.5ab,10520:5.1.5c}
\end{verbatim}\end{minipage}\begin{minipage}[t]{\LatL}\latex{\small}
\cite{10520:5.1.5ab,10520:5.1.5c}
\end{minipage}\vspace{5mm}\\
\noindent\begin{minipage}[t]{\VerbL}\latex{\small}\begin{verbatim}
\citeonline{10520:5.1.5ab,10520:5.1.5c}
\end{verbatim}\end{minipage}\begin{minipage}[t]{\LatL}\latex{\small}
\citeonline{10520:5.1.5ab,10520:5.1.5c}
\end{minipage}\vspace{5mm}\\

\subsection{et al.}

A `norma' 10520 \cite{NBR10520:2001} nada diz sobre como devem ser tratadas as
chamadas a referências com mais de três autores. Supondo que exista uma relação
lógica entre as `normas' 10520 \cite{NBR10520:2001} e 6023 \cite{NBR6023:2000},
então faz sentido pensar que deve-se usar `et al.' na chamada exatamente nos
mesmos casos em que `et al.' é usado na lista bibliográfica. No caso, a 6023 diz
que se deve usar `et al.' para \emph{mais de} três autores, ou seja, a partir
de quatro autores.\footnote{Muitas pessoas entendem isso errado e pensam
que se usa `et al.' já com três autores}

Parece lógico, não é mesmo? Então veja o seguinte caso. Fomos informados que
nas `normas' da UFLA (que \emph{supostamente} seguem as `normas' da ABNT)
exige-se o uso de `et al.' na chamada já com três autores e não se use `et al.'
na lista bibliográfica. E agora? Cadê a lógica?

Por isso insistimos que você sempre verifique a `lógica' da sua instituição
\emph{bem antes} de entregar a sua tese.

\addcontentsline{toc}{section}{Referências}
\bibliography{abntex2-doc-abnt-10520-2001,abntex2-doc-abnt-10520-2002,abntex2-doc-abnt-6023-2000,abntex2-doc-test,abntex2-doc}
%\bibliography{references/abnt-10520-2001,references/abnt-10520-2002,references/abnt-6023-2000,references/normas,references/abnt-test,references/abntex-doc}

\end{document}