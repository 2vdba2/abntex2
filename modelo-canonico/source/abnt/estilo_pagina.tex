%%%%%%%%%%%%%%%%%%%%%%%%%%%%%%%%%%%%%%%%%%%%%%%%%%%%%%
%    Page styles available: 
%         plainheader, header, ruledheader
%%%%%%%%%%%%%%%%%%%%%%%%%%%%%%%%%%%%%%%%%%%%%%%%%%%%%%

\newcommand{\ABNTmarkboth}[2]{%
 \ifthenelse{\boolean{ABNTNextOutOfTOC}}
     {\markboth{\ABNTnextmark}{\ABNTnextmark}}
     {\markboth{#1}{#2}}%
 }

\newcommand{\ABNTmarkright}[1]{%
 \ifthenelse{\boolean{ABNTNextOutOfTOC}}
     {\markright{\ABNTnextmark}}
     {\markright{#1}}%
 }


%%%%%%%  Defining pagestyle "header"
%
\newcommand{\ps@header}{%
  \renewcommand{\@oddfoot}{}%
  \renewcommand{\@evenfoot}{}%
  \renewcommand{\@oddhead}%
    {{\rightmarkformat\rightmark}\hfill{\thepageformat\thepage}}%
  \renewcommand{\@evenhead}%
       {{\thepageformat\thepage}\hfill{\leftmarkformat\leftmark}}%
% Para \chapter* mostrar o cabecalho
  \let\@mkboth\ABNTmarkboth%
% Definindo a maneira como o comando o \chapter marca o cabecalho
  \renewcommand{\chaptermark}[1]{%
    \markboth%
       {\ifnum \c@secnumdepth >\m@ne%
            \thechapter{}  %
        \fi%
        ##1}%
       {\ifnum \c@secnumdepth >\m@ne%
            \thechapter{}  %
        \fi%
        ##1}%
  }%
  \renewcommand{\sectionmark}[1]{%
    \markright{%
      \ifnum \c@secnumdepth >\z@%
        \thesection\ \ %
      \fi%
      ##1}%
  }%   
}% 
 %%  Defining pagestyle plainheader
%
\newcommand{\ps@plainheader}{%
  \renewcommand{\@oddfoot}{}%
  \renewcommand{\@evenfoot}{}%
  \renewcommand{\@oddhead}{\hfill{\thepageformat\thepage}}%
  \renewcommand{\@evenhead}{{\thepageformat\thepage}\hfill}%
% Para \chapter* mostrar o cabecalho
  \let\@mkboth\ABNTmarkboth%
% Definindo a maneira como o comando o \chapter marca o cabecalho
  \renewcommand{\chaptermark}[1]{%
    \markboth%
       {\ifnum \c@secnumdepth >\m@ne%
            \thechapter\ \ %
        \fi%
        ##1}%
       {\ifnum \c@secnumdepth >\m@ne%
            \thechapter\ \ %
        \fi%
        ##1}%
  }%
  \renewcommand{\sectionmark}[1]{%
    \markright{%
      \ifnum \c@secnumdepth >\z@%
        \thesection\ \ %
      \fi%
      ##1}%
  }%   
}% 
 %%  Defining pagestyle ruledheader
%
\newcommand{\ps@ruledheader}{%
  \renewcommand{\@oddfoot}{}%
  \renewcommand{\@evenfoot}{}%
  \renewcommand{\@oddhead}%
     {\underline{\makebox[\textwidth]{\raisebox{-.5ex}{}%
       {\rightmarkformat\rightmark}\hfill{\thepageformat\thepage}}}}%
  \renewcommand{\@evenhead}%
     {\underline{\makebox[\textwidth]{\raisebox{-.5ex}{}%
       {\thepageformat\thepage}\hfill{\leftmarkformat\leftmark}}}}%
% Para \chapter* mostrar o cabecalho
  \let\@mkboth\ABNTmarkboth%
% Definindo a maneira como o comando o \chapter marca o cabecalho
  \renewcommand{\chaptermark}[1]{%
    \markboth%
       {\ifnum \c@secnumdepth >\m@ne%
            \thechapter\ \ %
        \fi%
        ##1}%
       {\ifnum \c@secnumdepth >\m@ne%
            \thechapter\ \ %
        \fi%
        ##1}%
  }%
  \renewcommand{\sectionmark}[1]{%
    \markright{%
      \ifnum \c@secnumdepth >\z@%
        \thesection\ \ %
      \fi%
      ##1}%
  }%   
}% 