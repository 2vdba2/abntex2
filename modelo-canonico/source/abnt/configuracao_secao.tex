% From \@startsection. The only difference is that it calls \@ssect
% changing the meaning id the first parameter. Now, instead of indentation,
% it gives section level for TOC purposes.
\def\ABNTstartsection#1#2#3#4#5#6{%
  \if@noskipsec \leavevmode \fi
  \par
  \@tempskipa #4\relax
  \@afterindenttrue
  \ifdim \@tempskipa <\z@
    \@tempskipa -\@tempskipa \@afterindentfalse
  \fi
  \if@nobreak
    \everypar{}%
  \else
    \addpenalty\@secpenalty\addvspace\@tempskipa
  \fi
  \@ifstar
    {\ABNTssect{#1}{#4}{#5}{#6}}% #3 replaced by #1 here
    {\@dblarg{\ABNTsect{#1}{#2}{#3}{#4}{#5}{#6}}}}

% I change the meaning of the first paramenter here. Instead of an indent
% skip, it is now the name of the section, for `toc' purposes.
\def\ABNTssect#1#2#3#4#5{%
  \@tempskipa #3\relax
  \ifdim \@tempskipa>\z@
    \begingroup
      #4{%
         \interlinepenalty \@M \centering
         \ifthenelse{\boolean{ABNTcapsec}}
             {\MakeUppercase{#5}}{#5}\@@par}%
    \endgroup
    \@ifundefined{ABNT#1mark}{}{\csname ABNT#1mark\endcsname{#5}}
    \ifthenelse{\boolean{ABNTincludeintoc}}
      {\ABNTaddcontentsline{toc}{#1}{#5}}
     {}
  \else
    \def\@svsechd{#4{#5}%
      \@ifundefined{ABNT#1mark}{}{\csname ABNT#1mark\endcsname{#5}}
      \ifthenelse{\boolean{ABNTincludeintoc}}%
         {\ABNTaddcontentsline{toc}{#1}{#5}}{}
    }%
  \fi
  \@xsect{#3}}

\def\ABNTsect#1#2#3#4#5#6[#7]#8{%
  \ifnum #2>\c@secnumdepth
    \let\@svsec\@empty
  \else
    \refstepcounter{#1}%
    \protected@edef\@svsec{\@seccntformat{#1}\relax}%
  \fi
  \@tempskipa #5\relax
  \ifdim \@tempskipa>\z@
    \begingroup
      #6{%
        \@hangfrom{\hskip #3\relax\@svsec}%
          \interlinepenalty \@M
          \ifthenelse{\boolean{ABNTcapsec}}
             {\MakeUppercase{#8}}{#8}\@@par}%
    \endgroup
    \@ifundefined{ABNT#1mark}{}{\csname ABNT#1mark\endcsname{#7}}
    \ABNTaddcontentsline{toc}{#1}{%
      \ifnum #2>\c@secnumdepth \else
        \protect\numberline{\csname the#1\endcsname}%
      \fi
      #7}%
  \else
    \def\@svsechd{%
      #6{\hskip #3\relax
      \@svsec \ifthenelse{\boolean{ABNTcapsec}}
             {\MakeUppercase{#8}}{#8}}%
      \@ifundefined{ABNT#1mark}{}{\csname ABNT#1mark\endcsname{#7}}
      \ABNTaddcontentsline{toc}{#1}{%
        \ifnum #2>\c@secnumdepth \else
          \protect\numberline{\csname the#1\endcsname}%
        \fi
        #7}}%
  \fi
  \@xsect{#5}}
  
  
 
    


\renewcommand\section{\ABNTstartsection{section}{1}{\z@}%
                           {-3.5ex \@plus -1ex \@minus -.2ex}%
                           {2.3ex \@plus.2ex}%
                           {\espaco{simples}\normalfont%
                            \ABNTsectionfont\ABNTsectionfontsize}}
\renewcommand\subsection{\ABNTstartsection{subsection}{2}{\z@}%
                           {-3.25ex\@plus -1ex \@minus -.2ex}%
                           {1.5ex \@plus .2ex}%
                           {\espaco{simples}\normalfont%
                            \ABNTsubsectionfont\ABNTsubsectionfontsize}}
\renewcommand\subsubsection{\ABNTstartsection{subsubsection}{3}{\z@}%
                           {-3.25ex\@plus -1ex \@minus -.2ex}%
                           {1.5ex \@plus .2ex}%
                           {\espaco{simples}\normalfont%
                            \ABNTsubsubsectionfont\ABNTsubsubsectionfontsize}}
\renewcommand\paragraph{\ABNTstartsection{paragraph}{4}{\z@}%
                           {3.25ex \@plus1ex \@minus.2ex}%
                           {-1em}%
                           {\espaco{simples}\normalfont%
                            \ABNTparagraphfont\ABNTparagraphfontsize}}
\renewcommand\subparagraph{\ABNTstartsection{subparagraph}{5}{\parindent}%
                           {3.25ex \@plus1ex \@minus .2ex}%
                           {-1em}%
                           {\espaco{simples}\normalfont%
                            \ABNTsubparagraphfont\ABNTsubparagraphfontsize}}