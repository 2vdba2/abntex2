\documentclass[a4paper]{ltxdoc}

\usepackage[T1]{fontenc}		% seleção de códigos de fonte.
\usepackage[utf8]{inputenc}		% determina a codificação utiizada (conversão automática dos acentos)
\usepackage[brazil]{babel}		% idiomas
\usepackage{hyperref}  			% controla a formação do índice
\usepackage{parskip}


% COMANDOS PROPRIOS
\newcommand{\abnTeX}{abn\TeX}
\newcommand{\abnTeXForum}{\url{http://groups.google.com/group/abntex2}}
\newcommand{\abnTeXSite}{\url{http://code.google.com/p/abntex2/}}


\title{\textbf{A classe \textsf{abntex2}}: \\\Large{Modelo canônico de trabalhos
acadêmicos brasileiros conforme as normas ABNT}
%   \thanks{Este documento
%   se referete ao \textsf{abntex2} versão \fileversion,
%   de \filedate.}
  }
  
\author{Equipe \abnTeX2\\\abnTeXSite 
\and 
Lauro César Araujo\\\url{laurocesar@laurocesar.com}}

\date{\today}

\hypersetup{
		pdftitle={A classe abntex2},
		pdfauthor={Equipe abnTeX2},
    	pdfsubject={Modelo canônico de trabalhos
        acadêmicos brasileiros conforme as normas ABNT},
		pdfkeywords={ABNT}{abntex}{abntex2}{trabalho acadêmico}{dissertação}{tese},
	    pdfproducer={LaTeX with abnTeX2}, 	% producer of the document
	    pdfcreator={Lauro César Araujo -- laurocesar@laurocesar.com},
    	colorlinks=true,
    	linkcolor=blue,
    	citecolor=blue,
		urlcolor=black
}

\EnableCrossrefs
\CodelineIndex
\RecordChanges

\changes{v1.0}{2013/02/01}{Versão inicial}

\begin{document}

\definecolor{blue}{RGB}{41,5,195}


\maketitle

\begin{abstract}
  \noindent Este é o manual de uso da classe \textsf{abntex2}. Trata-se de um
  conjunto de customizações da classe \textsf{memoir} para elaboração de
  trabalhos acadêmicos condizentes com as normas da Associação Brasileira de
  Normas Técnicas (ABNT). Este manual não abrange o pacote
  \textsf{abntex2-cite} ou demais componentes do \abnTeX2.
\end{abstract}

\tableofcontents

% ------
\section{Escopo}
% ------

O objetivo deste manual é descrever a classe \textsf{abntex2}, responsável pelo
\textit{layout} dos elementos de estruturação de documentos acadêmicos,
especialmente aqueles definidos pela ABNT NBR 14724:2011 e pela ABNT NBR
6024:2012.

O pacote de citações do \abnTeX2, o \textsf{abntex2cite}, é detalhado em manual
próprio.

% ------
\section{Introdução}
% ------

Dentre as características de qualidade de trabalhos acadêmicos (teses,
dissertações e outros), ao lado da pertinência do tema e dos aspectos relativos
ao conteúdo abordado no trabalho, consta também o resultado da editoração final e
as características de forma e de estruturação dos documentos. Desse
modo, a existência de um modelo e de ferramentas que atendam às normas
brasileiras de elaboração de trabalhos acadêmicos propostas pela Associação
Brasileira de Normas Técnicas (ABNT) são recursos básicos para o aprimoramento
da qualidade geral dos trabalhos acadêmicos nacionais.

É com esse intuito que o \abnTeX2 é apresentado à comunidade acadêmica
brasileira: o de ser um instrumento de aperfeiçoamento da qualidade dos
textos produzidos. O \abnTeX2 surge para se somar ao já vasto universo de
ferramentas \LaTeX, porém que é escasso em utilitários específicos para
trabalhos brasileiros. Nesse sentido, \abnTeX2 é proposto como uma evolução do
\abnTeX1 original\footnote{Ver \autoref{sec-historico},
\nameref{sec-historico}.} e como facilitador na elaboração de trabalhos
acadêmicos condizentes com as normas ABNT.

A expressão ``Modelo canônico'' é utilizada para indicar que \abnTeX2 não é 
modelo específico de trabalho acadêmico de nenhuma universidade ou instituição,
mas que implementa tão somente os requisitos das normas da ABNT.

O \abnTeX2 foi desenvolvido com base nos requisitos das seguintes normas ABNT:

\begin{description}
  \item[ABNT NBR 6023:2002] Informação e documentação -
  Referência - Elaboração
  \item[ABNT NBR 6024:2012] Informação e documentação - Numeração
  progressiva das seções de um documento - Apresentação
  \item[ABNT NBR 6027:2003] Informação e documentação - Sumário -
  Apresentação
  \item[ABNT NBR 6028:2003] Informação e documentação - Resumo -
  Apresentação
  \item[ABNT NBR 6034:2004] Informação e documentação - Índice -
  Apresentação
  \item[ABNT NBR 10520:2002] Informação e documentação - Citações
  \item[ABNT NBR 14724:2011] Informação e documentação - Trabalhos
  acadêmicos - Apresentação
\end{description}

Sinta-se convidado a participar do projeto \abnTeX2! Acesse o site do projeto em
\abnTeXSite. Também fique livre para conhecer, estudar, alterar e redistribuir o
trabalho do \abnTeX2, desde que os arquivos modificados tenham seus nomes
alterados, até mesmo no seu computador, e que os créditos sejam dados aos
autores originais, nos termos da ``The LaTeX Project Public
License''\footnote{\url{http://www.latex-project.org/lppl.txt}}.

Encorajamos que sejam realizadas customizações específicas para universidades
--- como capas, folha de aprovação, etc. Porém, recomendamos que ao invés de se
alterar diretamente os arquivos do \abnTeX2, distribua-se aos estudantes
arquivos com as respectivas customizações. Isso permite que futuras versões do
\abnTeX2 não se tornem automaticamente incompatíveis com as customizações
promovidas.

Este manual de uso não foi escrito sob o modelo do \abnTeX2, nem segue os
padrões da ABNT.

Este texto deve ser utilizado como complemento do manual do \textsf{memoir},
disponível em
\url{http://mirrors.ctan.org/macros/latex/contrib/memoir/memman.pdf}.

\subsection{Histórico do projeto}\label{sec-historico}

Entre 2001 e 2002 nascia o projeto \abnTeX~original, cujo objetivo era o de
``prover um conjunto de macros para LaTeX para formação de trabalhos acadêmicos
condizentes com as normas ABNT''. A última versão estável publicada pelos
integrantes originais\footnote{Os integrantes originais do projeto são Miguel
Frasson, Gerald Weber, Leslie H. Watter, Bruno Parente Lima, Flávio de
Vasconcellos Corrêa, Otavio Real Salvador e Renato Machnievscz.} é a versão
0.8.2 e data de 3.11.2004 (hospedada em
\url{http://abntex.codigolivre.org.br/}). Em 2006 uma versão não estável foi
publicada para testes, mas nunca foi evoluída.

Em maio de 2009 Leandro Salvador fez uma chamada no fórum Tex-BR\footnote{A
mensagem pode ser lida neste link:
\url{http://marc.info/?l=tex-br\&m=124110883528962}} clamando por voluntários
para a retomada do projeto. Embora tenha sido criado um novo repositório para o
projeto\footnote{O projeto de Salvador está hospedado em
\url{https://sourceforge.net/projects/abntex/}.}, nenhuma nova contribuição ao
código do \abnTeX~foi realizada desde 2006 até dezembro de 2012. No novo
endereço há discussões um pouco mais recentes sobre o projeto, porém datam de
2009. Há, inclusive, uma sugestão de Gerald Weber, um dos criadores originais:

\begin{quote}
``A sugestão que eu daria seria substituir a classe que o Miguel Frasson
desenvolveu por pacotes padrão do LaTeX. Há pacotes que implementam
praticamente tudo que a ABNT requer, basta mexer nas opções. Assim teria algo
muito mais simples de manter e atualizar no longo prazo.''
(\url{https://sourceforge.net/projects/abntex/forums/forum/947854/topic/3265973})
\end{quote}

A ideia de Gerald e o software já construído do \abnTeX~original servem de ponto
de partida para o surgimento deste novo projeto, o \abnTeX2.
O novo projeto utiliza o menor conjunto possível de classes nativas do \LaTeX~de
modo a implementar as exigências das normas ABNT. Para isso, escolheu-se como
base a classe
\textsf{memoir}\footnote{\url{http://www.ctan.org/tex-archive/macros/latex/contrib/memoir/}},
por ser uma classe flexível, com excelente documentação e que constantemente
recebe atualizações de novas funcionalidades e correção de eventuais problemas.
Este documento descreve o uso desta nova classe.


% ------
\section{Estrutura de trabalhos acadêmicos brasileiros}
% ------

A seção 4 da ABNT NBR 14724:2011 estabelece que a estrutura de trabalhos
acadêmicos compreende duas partes: a externa e a interna, que aparecem no texto
na seguinte ordem:

\begin{description}
  \item[Parte externa]\ \\ 
       Capa (obrigatório) \\
       Lombada (opcional) 
  \item[Parte Interna]\ 
		\begin{description} 
		\item[Elementos pré-textuais]\ \\
		  Folha de rosto (obrigatório) + \\
		  \hspace*{0.2cm}Dados de catalogação-na-publicação\footnote{O documento
		  ``Dados de catalogação-na-publicação'' é chamado apenas como ``Ficha
		  catalográfica'' neste texto.} (opcional)\\
		  Errata (opcional) \\
		  Folha de aprovação (obrigatório) \\
		  Dedicatória (opcional) \\
		  Agradecimentos (opcional) \\
		  Epígrafe (opcional) \\
		  Resumo em língua vernácula (obrigatório) \\
		  Resumo em língua estrangeira (obrigatório) \\
		  Lista de ilustrações (opcional) \\
		  Lista de tabelas (opcional) \\
		  Lista de abreviaturas e siglas (opcional) \\
		  Lista de símbolos (opcional) \\ 
		  Sumário (obrigatório) \\
		\item[Elementos textuais]\footnote{A nomenclatura dos títulos dos elementos
		textuais é a critério do autor.}\
		\\
		  Introdução \\
		  Desenvolvimento \\
		  Conclusão 
		\item[Elementos pós-textuais]\ \\
		  Referências (obrigatório) \\
		  Glossário (opcional) \\
		  Apêndice (opcional) \\
		  Anexo (opcional) \\
		  Índice (opcional) 
		\end{description}
\end{description} 

O \abnTeX2 apresenta instrumentos para produzir todas as partes do documento,
exceto a Lombada, a Lista de Símbolos e o Glossário, que podem ser produzidos
por outros pacotes adicionais. As seções seguintes descrevem como cada seção
pode ser produzida.

% ------
\section{Configurações gerais}\label{sec-configgerais}
% ------

\DescribeMacro{\documentclass}
A classe \textsf{abntex2} foi criada como um conjunto de configurações da classe
\textsf{memoir}\footnote{A versão anterior do \abnTeX~era baseada na
classe \textsf{report}.}. Desse modo, todas as opções do \textsf{memoir} estão
disponíveis, como por exemplo, |12pt,openright,twoside,a4paper,artile|. Consulte
o manual do \textsf{memoir} para outras opções.

As opções mais comuns de inicialização do texto do trabalho são:

\begin{verbatim}
   \documentclass[12pt,openright,twoside,a4paper]{abntex2}
\end{verbatim}

É interessante observar que a ABNT NBR 14724:2011 (seção 5.1) recomenda que os
trabalhos sejam impressos no anverso e no verso das folhas. Isso é obtido com a
opção |twoside|. 

A opção |article| é útil para produção de artigos com \abnTeX2.
Nesse caso, a maioria dos elementos pré-textuais descritos neste documento na
\autoref{sec-pretextuais} se tornam desnecessários.


\DescribeMacro{\ABNTEXfontereduzida}\DescribeMacro{\footnotesize}
A seção 5.1 da ABNT NBR 14724:2011 também estabelece que o tamanho fonte seja 12
para todo o trabalho (obtida com a opção |12pt|), ``inclusive capa,
excetuando-se citações com mais de três linhas, notas de rodapé, paginação,
dados internacionais de catalogação-na-publicação, legendas e fontes das
ilustrações e das tabelas, que devem ser em tamanho menor e uniforme''. O
tamanho ``menor e uniforme'' para citações com mais de três linhas, paginação e
dados internacionais de catalogação-na-pulicação é estabelecido pela macro
|\ABNTEXfontereduzida| e o valor padrão utilizado é o mesmo da macro
|\footnotesize|. Os demais elementos são controlados diretamente pela macro
|\footnotesize|. Você pode alterar o valor de |\ABNTEXfontereduzida| para
|\small|, por exemplo, com o seguinte comando:

\begin{verbatim}
   \renewcommand{\footnotesize}{\small}
\end{verbatim}

% ---
\subsection{Espaçamentos}
% ---

O espaçamento padrão é definido como |\OnehalfSpacing|, ou seja, um espaço e
meio. Os comando |\SingleSpacing|, |\DoubleSpacing| podem ser utilizados para
obter espaçamento simples e espaçamento duplo, respectivamente. Também estão
disponíveis ambientes do tipo |\begin{}\end{}| com os mesmos nomes das macros.
Observe que a classe \textsf{abntex2} utiliza o sistema de espaçamento padrão
do \textsf{memoir}. Nesse caso, o pacote \textsf{setspace} não é necessário.

% ---
\subsection{Margens}
% ---

As \emph{margens} são configuradas conforme a NBR 14724/2011, seção 5.1, 
e podem ser alteradas do seguinte modo:

  \begin{verbatim}
     \setlrmarginsandblock{3cm}{2cm}{*}
     \setulmarginsandblock{3cm}{2cm}{*}
     \checkandfixthelayout
  \end{verbatim}

% ---
\subsection{Alterações no pacote \textsf{babel}}
% ---

Algumas alterações foram realizadas nos nomes padronizados pelo pacote
\textsf{babel}\footnote{\url{http://www.ctan.org/tex-archive/macros/latex/required/babel/}}
para atender a requisitos da ABNT NBR 14724:2011, conforme \autoref{tab-babel}.


\begin{table}[htb]
\caption{Alterações no pacote \textsf{babel}}
\label{tab-babel}
\centering
\begin{tabular}{ l l l }
   \textbf{Macro} & \textbf{Valor original} & \textbf{Novo valor} \\
    \hline
    |\bibname| & ``Referências Bibliográficas'' &  ``Referências'' \\
    \hline
    |\indexname| & ``Índice Remissivo'' & ``Índice'' \\
    \hline
    |\listfigurename| & ``Lista de Figuras'' & ``Lista de ilustrações'' \\
    \hline
    \hline
\end{tabular}
\end{table}

Se desejar outro comportamento, use:
  \begin{verbatim}
	 \addto\captionsbrazil{%
	    \renewcommand{\bibname}{Refer\^encias}
	    \renewcommand{\indexname}{\'Indice}
	    \renewcommand{\listfigurename}{Lista de Ilustra\c{c}\~{o}es} }
  \end{verbatim} 

% ---
\subsection{Numeração contínua de figuras e tabelas}
% ---

  
A numeração de figuras e tabelas deve ser contínua em todo o documento (ABNT
NBR 14724:2011 seções 5.8 e 5.9). Porém, caso deseje alterar esse comportamento
para numeração por capítulos, por exemplo, use:
  \begin{verbatim}
    \counterwithout{figure}{section}
    \counterwithout{table}{section}
  \end{verbatim}


% ---
\subsection{Índice do PDF com pacote \textsf{bookmark}}\label{sec-bookmark}
% ---

O índice da estrutura do documento é automaticamente inserida no PDF final do
documento por meio do pacote
\textsf{bookmark}\footnote{\url{http://www.ctan.org/pkg/bookmark}}. Neste
documento este índice será identificado como ``\textsf{bookmark} do PDF''.

Com excessão da Ficha catalográfica (\autoref{sec-fichacatalografica}), todos os
elementos pré-textuais descritos na \autoref{sec-pretextuais} e as divisões dos
documentos, como |\part|, |\chapter|, |\section|, etc., são automaticamente
inseridos tanto no Sumário (\autoref{sec-sumario}) quanto no \textsf{bookmark}
do PDF.

\DescribeMacro{\pdfbookmark}
A versão * dos comandos, como |\part*| e |\chapter*|, por exemplo, não inclui a
divisão nem no Sumário nem no \textsf{bookmark} do PDF. Porém, você pode
explicitamente incluí-las no \textsf{bookmark} com o comando
|\pdfbookmark|\oarg{posição}\marg{Título no bookmark}\marg{texto de
identificação única, sem espaços}:

\begin{verbatim}
   \pdfbookmark[0]{Capítulo fora do Sumário, mas presente no bookmark}{texto-qualquer} 
   \chapter*{Capítulo fora do Sumário, mas presente no bookmark}
\end{verbatim}

\DescribeMacro{\phantomsection}
A macro |\phantomsection| pode ser útil imediatamente antes de |\pdfbookmark|
quando o texto adicionado ao bookmark não estiver próxima a uma divisão do
documento. Nesse caso, o comando fica assim:

\begin{verbatim}
   \phantomsection\pdfbookmark[0]{Capítulo}{texto-qualquer2} 
   \chapter*{Capítulo}
\end{verbatim}

Veja a sugestão de uso do \textsf{bookmark} do PDF na
\autoref{sec-listadeilustracoes}, \autoref{sec-listadetabelas},
\autoref{sec-listadeabreviaturas} e na \autoref{sec-sumario}.

\DescribeMacro{\pretextualchapter}
A macro |\pretextualchapter|\marg{título do capítulo} pode ser utilizada para
criar capítulos sem numeração, que não aparecem no Sumário, mas que são
automaticamente adicionados ao \textsf{bookmark} do PDF. Consulte a
\autoref{sec-formatacaocapitulos} para mais detalhes.

% ---
\subsection{Macros de dados do trabalho}\label{sec-macrosdados}
% ---

As macros descritas nas próximas subseções são utilizadas para armazenar dados
do trabalho. Esses dados são utilizados na Capa (\autoref{sec-capa}), Folha de
rosto (\autoref{sec-folhaderosto}), Folha de aprovação
(\autoref{sec-folhadeaprovacao}) e Ficha catalográfica
(\autoref{sec-fichacatalografica}). Recomenda-se que esses dados sejam
preenchidos ainda no preâmbulo do documento \LaTeX, de modo que possam ser úteis
para configurações do PDF final com a classe
\textsf{hyperref}\footnote{\url{http://www.ctan.org/tex-archive/macros/latex/contrib/hyperref/}},
por exemplo. A \autoref{sec-dados_hyperref} ilustra um exemplo de configuração
do pacote \textsf{hyperref}.

% ---
\subsubsection{Título}
% ---

\DescribeMacro{\titulo}\DescribeMacro{\imprimirtitulo} 
A macro |\titulo|\marg{texto do título} é utilizada para armazenar o título do
trabalho. O conteúdo armazenado é impresso por meio da macro |\imprimirtitulo|.

% ---
\subsubsection{Autor}
% ---

\DescribeMacro{\autor}\DescribeMacro{\imprimirautor} 
A macro |\autor|\marg{nome do(s) autor(es)} é utilizada para armazenar os nomes
dos autores do trabalho. O conteúdo armazenado é impresso por meio da macro
|\imprimirautor|.

% ---
\subsubsection{Data}
% ---

\DescribeMacro{\data}\DescribeMacro{\imprimirdata} 
A macro |\data|\marg{mês e ano ou data em português} é utilizada para armazenar
o mês e ano ou a data do trabalho. O conteúdo armazenado é
impresso por meio da macro |\imprimirdata|.

Durante o desenvolvimento do trabalho, é possível utilizar |\data{\today}| para
que seja armazenada a data atual.

% ---
\subsubsection{Instituição}
% ---

\DescribeMacro{\instituicao}\DescribeMacro{\imprimirinstituicao} 
A macro |\instituicao|\marg{nome da instituição} é utilizada para armazenar
o nome da instituição. O conteúdo armazenado é impresso por meio da macro
|\imprimirinstituicao|.

% ---
\subsubsection{Localidade}
% ---

\DescribeMacro{\local}\DescribeMacro{\imprimirlocal} 
A macro |\local|\marg{localidade de apresentação trabalho} é utilizada para
armazenar a localidade de apresentação do trabalho, geralmente o nome da cidade
e a unidade federativa. O conteúdo armazenado é impresso por meio da macro
|\imprimirlocal|.

% ---
\subsubsection{Preâmbulo}
% ---

\DescribeMacro{\preambulo}\DescribeMacro{\imprimirpreambulo} 
A macro |\preambulo|\marg{preâmbulo trabalho} é utilizada para
armazenar o preâmbulo do trabalho. O preâmbulo é o texto impresso na Folha de
rosto e na Folha de aprovação. Ele deve conter o tipo do trabalho, o objetivo, o
nome da instituição e a área de concentração. O conteúdo armazenado é impresso
por meio da macro |\imprimirpreambulo|.

% ---
\subsubsection{Tipo de trabalho}
% ---

\DescribeMacro{\tipotrabalho}\DescribeMacro{\imprimirtipotrabalho} 
A macro |\tipotrabalho|\marg{tipo do trabalho} é utilizada para
armazenar o tipo de trabalho. Geralmente os textos ``Tese (doutorado)'' ou
``Dissertação (mestrado)'' são utilizados. O tipo de trabalho é utilizado na
Ficha catalográfica (\autoref{sec-fichacatalografica}). O conteúdo armazenado é
impresso por meio da macro |\imprimirtipotrabalho|.

% ---
\subsubsection{Orientador}
% ---

\DescribeMacro{\orientador}\DescribeMacro{\imprimirorientador}\DescribeMacro{\imprimirorientadorRotulo}
 A macro |\orientador|\oarg{rótulo}\marg{nome do(s) orientador(es)} é
utilizada para armazenar o nome do(s) orientador(es). O parâmetro opcional
indica o rótulo a ser utilizado. O valor padrão do rótulo é ``Orientador:''. O
conteúdo armazenado é impresso por meio da macro |\imprimirorientador| e o rótulo pode
ser impresso com o comando |\imprimirorientadorRotulo|.

% ---
\subsubsection{Coorientador}
% ---

\DescribeMacro{\coorientador}\DescribeMacro{\imprimircoorientador}\DescribeMacro{\imprimircoorientadorRotulo}
A macro |\coorientador|\oarg{rótulo}\marg{nome do(s) coorientador(es)} é
utilizada para armazenar o nome do(s) coorientador(es). O parâmetro opcional
indica o rótulo a ser utilizado. O valor padrão do rótulo é ``Coorientador:''. O
conteúdo armazenado é impresso por meio da macro |\imprimircoorientador| e o
rótulo pode ser impresso com o comando |\imprimircoorientadorRotulo|.

% ---
\subsubsection{Exemplo de uso \textsf{hyperref} com dados do
trabalho}\label{sec-dados_hyperref}
% ---

O \textsf{hyperref} pode inserir informações dos dados do trabalho nos metadados
do PDF final, conforme o exemplo, que também altera informações de cores dos
links internos do documento final:

\begin{verbatim}

\usepackage{hyperref}  			% controla a formação do índice

\titulo{Modelo Canônico de\\ Trabalhos Acadêmicos com \abnTeX}
\autor{Equipe \abnTeX}
\local{Brasil}
\data{2012}
\orientador{Lauro César Araujo}
\coorientador{Equipe \abnTeX}
\instituicao{%
  Universidade do Brasil -- UBr
  \par
  Faculdade de Arquitetura da Informação
  \par
  Programa de Pós-Graduação}
\tipotrabalho{Tese (Doutorado)}
\preambulo{Modelo canônico de trabalho monográfico acadêmico em conformidade com
as normas ABNT apresentado à comunidade de usuários \LaTeX.}

\hypersetup{
		pdftitle={\imprimirtitulo}, 
		pdfauthor={\imprimirautor},
    	pdfsubject={\imprimirpreambulo},
		pdfkeywords={PALAVRAS}{CHAVES}{EM}{PORTUGUES},
	    pdfproducer={LaTeX with abnTeX2}, 	% producer of the document
	    pdfcreator={\imprimirautor},
    	colorlinks=true,
    	linkcolor=blue,
    	citecolor=blue,
		urlcolor=blue
}
\end{verbatim}


% ------
\section{Elementos pré-textuais e Parte externa}\label{sec-pretextuais}
% ------

\DescribeMacro{\pretextual}\DescribeMacro{\frontmatter}
O comando |\pretextual| identifica o início dos elementos pré-textuais. Esses
elementos não possuem numeração de páginas, nem em algarismos romanos, conforme
estabelece a ABNT NBR 14724:2011. Porém, a contagem é iniciada logo após a Capa.
A classe \textsf{abntex2} não faz separação entre ``Parte externa'' e ``Parte
interna'', por isso, a macro |\pretextual| é acionado automaticamente no início
de |\begin{document}| e desse modo você não precisa explicitamente incluí-la em
seu documento. A título de coerência, a macro |\frontmatter|, padrão do
\textsf{memoir}, é reescrita para que tenha o mesmo comportamento que
|\pretextual|.

\DescribeMacro{\pretextualchapter}
A macro |\pretextualchapter|\marg{titulo do capitulo} pode ser utilizado para
adicionar um capítulo com a formatação estabelecida na seção 5.2.3 da ABNT NBR
14724:2011 e na seção 4.1 da ABNT NBR 6024:2012. Consulte a
\autoref{sec-formatacaocapitulos} para obter informações sobre o comportamento
do \textsf{bookmark} do PDF de capítulos pré-textuais e a
\autoref{sec-formatacaocapitulos} para ler detalhes adicionais sobre a
formatação de capítulos.

As subseções seguintes descrevem como cada elemento pré-textual pode ser
construído com \abnTeX2.

% ---
\subsection{Capa (obrigatório)}\label{sec-capa}
% ---

\DescribeMacro{\imprimircapa}\DescribeEnv{capa}
A macro |\imprimircapa| imprime um modelo básico de capa que atende aos
requisito da seção 4.1.1 da ABNT NBR 14724:2011. A capa é o único elemento
``externo'' que \abnTeX2 produz.

A capa não é incluída no \textsf{bookmark} do PDF.

Para criar uma capa diferente, você pode reescrever a macro |\imprimircapa|
com base no ambiente |capa|, conforme o exemplo:

\begin{verbatim}
\renewcommand{\imprimircapa}{%
  \begin{capa}%
    \center
    \vspace*{1cm}
    {\ABNTEXchapterfont\large\imprimirautor}

    \vspace*{\fill}
    {\ABNTEXchapterfont\LARGE\imprimirtitulo}
    \vspace*{\fill}
    
    {\large\imprimirlocal}
    \par
    {\large\imprimirdata}
    
    \vspace*{1cm}
  \end{capa}
}
\end{verbatim}


% ---
\subsection{Folha de rosto (obrigatório)}\label{sec-folhaderosto}
% ---

\DescribeMacro{\imprimirfolhaderosto}\DescribeMacro{\folhaderostoname}
A macro |\imprimirfolhaderosto|\oarg{nome da folha de rosto} imprime um modelo
básico de folha de rosto que atende aos requisito da seção 4.2.1.1 da ABNT NBR
14724:2011. A folha de rosto é incluída automaticamente no \textsf{bookmark} do
PDF com o nome dado pelo valor do argumento opcional. Caso ele não seja informado, o
conteúdo de |\folhaderostoname| é utilizado (\autoref{sec-macrosdados}).

\DescribeMacro{\imprimirfolhaderosto*}
A variante |\imprimirfolhaderosto*|\oarg{nome da folha de rosto}
deve ser utilizada quando se estiver utilizando a impressão frente e verso
(|twoside|) e se desejar incluir a ``Ficha catalográfica'' (Dados de
catalogação-na-publicação, \autoref{sec-fichacatalografica}), que deve ser
impressa no verso da Folha de rosto, conforme a seção 4.2.1.1.2 da
mesma norma. Observe que o uso de |\imprimirfolhaderosto*| sem o uso da Ficha
catalográfica poderá trazer comportamento não desejado à numeração das páginas.
Porém, se a opção |\twoside| não estiver sendo utilizada, as duas versões
da macro têm o mesmo comportamento.

Você pode criar uma folha de rosto diferente sem se preocupar com as diferenças
entre os comandos |\imprimirfolhaderosto| e |\imprimirfolhaderosto*|. Para isso,
reescreva a macro |\folhaderostocontent|, conforme o exemplo:

\begin{verbatim}
\makeatletter
\newcommand{\folhaderostocontent}{
  \begin{center}

    \vspace*{1cm}
    {\ABNTEXchapterfont\large\imprimirautor}

    \vspace*{\fill}\vspace*{\fill}
    {\ABNTEXchapterfont\Large\imprimirtitulo}
    \vspace*{\fill}

    \abntex@ifnotempty{\imprimirpreambulo}{%
      \hspace{.45\textwidth}
      \begin{minipage}{.5\textwidth}
      	\SingleSpacing
         \imprimirpreambulo
       \end{minipage}%
       \vspace*{\fill}
    }%

    {\large\imprimirorientadorRotulo~\imprimirorientador\par}
    \abntex@ifnotempty{\imprimircoorientador}{%
       {\large\imprimircoorientadorRotulo~\imprimircoorientador}%
    }%
    \vspace*{\fill}

    {\abntex@ifnotempty{\imprimirinstituicao}{\imprimirinstituicao\vspace*{\fill}}}

    {\large\imprimirlocal}
    \par
    {\large\imprimirdata}
    \vspace*{1cm}

  \end{center}
}
\makeatother
\end{verbatim}

% ---
\subsection{Ficha Catalográfica (Dados de
catalogação-na-publicação) (obrigatório)}\label{sec-fichacatalografica}
% ---

\DescribeEnv{fichacatalografica}
O ambiente |fichacatalografica| deve ser utilizado para impressão da Ficha
catalográfica, ou ``Dados de catalogação-na-publicação'', conforme estabelece a
seção 4.2.1.1.2 da ABNT NBR 14724:2011. 

Caso a impressão frente e verso seja acionada (opção |twoside|), é necessário
que a Folha de rosto (\autoref{sec-folhaderosto}) seja impressa com a versão
estrelada (*) da macro |\imprimirfolhaderosto*|, para que
a Ficha catalográfica seja impressa no verso da Folha de rosto.

Um exemplo de uso do comando é:

\begin{verbatim}
\begin{fichacatalografica}
	\vspace*{15cm}					% Posição vertical
	\hrule							% Linha horizontal
	\begin{center}					% Minipage Centralizado
	\begin{minipage}[c]{12.5cm}		% Largura
	
	\imprimirautor
	
	\hspace{0.5cm} \imprimirtitulo  / \imprimirautor. --
	\imprimirlocal, \imprimirdata-
	
	\hspace{0.5cm} \pageref{LastPage} p. : il. (algumas color.) ; 30 cm.\\
	
	\hspace{0.5cm} \imprimirorientadorRotulo \imprimirorientador\\
	
	\hspace{0.5cm}
	\parbox[t]{\textwidth}{\imprimirtipotrabalho~--~\imprimirinstituicao,
	\imprimirdata.}\\
	
	\hspace{0.5cm}
		1. Palavra-chave1.
		2. Palavra-chave2.
		I. Orientador.
		II. Universidade xxx.
		III. Faculdade de xxx.
		IV. Título\\ 			
	
	\hspace{8.75cm} CDU 02:141:005.7\\
	
	\end{minipage}
	\end{center}
	\hrule
\end{fichacatalografica}
\end{verbatim}

O exemplo apresentado necessita do pacote |lastpage| para
que ele possa obter o número da última página do documento.
Portanto, para usar o exemplo é preciso adicionar a linha
abaixo ao preâmbulo do documento:

\begin{verbatim}
   % usado por abntex2-fichacatalografica.tex
   \usepackage{lastpage}
\end{verbatim}

A Ficha catalográfica não é incluída no \textsf{bookmark} do PDF
(\autoref{sec-bookmark}). 

% ---
\subsection{Errata (opcional)}
% ---

\DescribeEnv{errata}\DescribeMacro{\errataname}
A Errata é um elemento opcional da ABNT NBR 14724:2011, seção 4.2.1.2, que pode
ser criada por meio do ambiente |errata|\oarg{nome da errata}. O valor do
parâmetro opcional é utilizado como entrada no \textsf{bookmark} do PDF
(\autoref{sec-bookmark}). Caso o parâmetro não seja informado, o conteúdo de
|\errataname| é utilizado (\autoref{sec-macrosdados}).

Um exemplo de uso do ambiente é:

\begin{verbatim}
\begin{errata}

	FERRIGNO, C. R. A. \textbf{Tratamento de neoplasias ósseas apendiculares com
	reimplantação de enxerto ósseo autólogo autoclavado associado ao plasma
	rico em plaquetas}: estudo crítico na cirurgia de preservação de membro em
	cães. 2011. 128 f. Tese (Livre-Docência) - Faculdade de Medicina Veterinária e
	Zootecnia, Universidade de São Paulo, São Paulo, 2011.
	
	\begin{table}[htb]
	\center
	\footnotesize
	\begin{tabular}{|p{1.4cm}|p{1cm}|p{3cm}|p{3cm}|}
	  \hline
	   \textbf{Folha} & \textbf{Linha}  & \textbf{Onde se lê}  & \textbf{Leia-se}  \\
	    \hline
	    1 & 10 & auto-conclavo & autoconclavo\\
	   \hline
	\end{tabular}
	\end{table}
	
\end{errata}
\end{verbatim}

% ---
\subsection{Folha de aprovação (obrigatório)}\label{sec-folhadeaprovacao}
% ---

\DescribeEnv{folhadeaprovacao}\DescribeMacro{\folhadeaprovacaoname}
O ambiente |folhadeaprovacao|\oarg{nome da folha de aprovação} permite a criação
de uma Folha de aprovação, elemento obrigatório da ABNT NBR 14724/2011
descrita na seção 4.2.1.3 da norma. O valor do parâmetro opcional é utilizado
como entrada no \textsf{bookmark} do PDF (\autoref{sec-bookmark}). Caso o parâmetro
não seja informado, o conteúdo de |\folhadeaprovacaoname| é utilizado
(\autoref{sec-macrosdados}). Conforme estabelece a seção 5.2.4 da norma em tela,
a Folha de aprovação não possui título nem indicador numérico.

\DescribeMacro{\includepdf}
Um modelo de Folha de aprovação não é oferecido pelo \abnTeX2 porque
ela varia largamente entre instituições. E, além disso, provavelmente você
incluirá uma versão digitalizada com assinaturas dos membros da banca após a
apresentação do trabalho. Uma página digitalizada pode ser incluída no
documento com o comando:

\begin{verbatim}
   \includepdf{folhadeaprovacao_final.pdf}
\end{verbatim}

De todo modo, você pode utilizar o seguinte modelo de Folha de aprovação até a
aprovação final do trabalho:

\begin{verbatim}
\begin{folhadeaprovacao}

  \begin{center}
    \vspace*{1cm}
    {\ABNTEXchapterfont\large\imprimirautor}

    \vspace*{\fill}\vspace*{\fill}
    {\ABNTEXchapterfont\Large\imprimirtitulo}
    \vspace*{\fill}
    
    \hspace{.45\textwidth}
    \begin{minipage}{.5\textwidth}
        \imprimirpreambulo
    \end{minipage}%
    \vspace*{\fill}
   \end{center}
    
   Trabalho aprovado. \imprimirlocal, 24 de novembro de 2012:

   \assinatura{\textbf{\imprimirorientador} \\ Orientador} 
   \assinatura{\textbf{Professor} \\ Convidado 1}
   \assinatura{\textbf{Professor} \\ Convidado 2}
   \assinatura{\textbf{Professor} \\ Convidado 3}
   \assinatura{\textbf{Professor} \\ Convidado 4}
      
   \begin{center}
    \vspace*{0.5cm}
    {\large\imprimirlocal}
    \par
    {\large\imprimirdata}
    \vspace*{1cm}
  \end{center}
  
\end{folhadeaprovacao}
\end{verbatim}

\DescribeMacro{\assinatura}
A macro |\assinatura|\marg{texto a ser impresso} é um utilitário para
impressão de assinaturas da Folha de aprovação. Ela imprime o 
|texto a ser impresso| centralizado abaixo de uma linha. A versão |\assinatura*|
imprime a mesma assinatura em uma |box| sem posição atribuída, o que é útil para impressão
de assinaturas lado a lado.

\DescribeMacro{\ABNTEXsignwidth}
O comprimento da linha de assinatura é definido pela métrica |\ABNTEXsignwidth|.
O valor padrão é definido como: |\setlength{\ABNTEXsignwidth}{8cm}|.

\DescribeMacro{\ABNTEXsignthickness}
A largura da linha de assinatura é definida pela métrica |\ABNTEXsignthickness|. O
valor padrão é definido como: |\setlength{\ABNTEXsignthickness}{1pt}|.

\DescribeMacro{\ABNTEXsignskip}
O espaçamento entre um comando |\assinatura| e outro é definido pela métrica
|\ABNTEXsignskip|. O valor padrão é definido como:
|\setlength{\ABNTEXsignskip}{1cm}|.

% ---
\subsection{Dedicatória (opcional)}
% ---

\DescribeEnv{dedicatoria}\DescribeMacro{\dedicatorianame}
A Dedicatória é um elemento opcional da ABNT NBR 14724:2011, seção 4.2.1.4, que
pode ser criada por meio do ambiente |dedicatoria|\oarg{nome da dedicatória}.

O valor do parâmetro opcional é utilizado como entrada no \textsf{bookmark} do
PDF (\autoref{sec-bookmark}) e como título da dedicatória, que é escrito como um
título de capítulo pré-textual, ou seja, não numerado e centralizado. Caso o
parâmetro não seja informado, o conteúdo de |\dedicatorianame| é utilizado apenas
para entrada do \textsf{bookmark} do PDF e a dedicatória é impressa sem título e
sem indicador numérico, conforme estabelece a seção 5.2.4 da norma em tela.

Um exemplo de uso do ambiente é:

\begin{verbatim}
\begin{dedicatoria}

   \vspace*{\fill}
   Este trabalho é dedicado aos que acreditam...
   \vspace*{\fill}

\end{dedicatoria}
\end{verbatim}

% ---
\subsection{Agradecimentos (opcional)}
% ---

\DescribeEnv{agradecimentos}\DescribeMacro{\agradecimentosname}
A seção Agradecimentos é um elemento opcional da ABNT NBR 14724:2011, 4.2.1.5,
que pode ser criada por meio do ambiente |agradecimentos|\oarg{nome dos
agradecimentos}. O valor do parâmetro opcional é utilizado como entrada no
\textsf{bookmark} do PDF (\autoref{sec-bookmark}). Caso o parâmetro não seja
informado, o conteúdo de |\agradecimentosname| é utilizado (\autoref{sec-macrosdados}).

Um exemplo de uso do ambiente é:

\begin{verbatim}
\begin{agradecimentos}

   Os agradecimentos...

\end{agradecimentos}
\end{verbatim}

% ---
\subsection{Epígrafe (opcional)}
% ---

\DescribeEnv{epigrafe}\DescribeMacro{\epigraphname}
A Epígrafe é um elemento opcional da ABNT NBR 14724:2011, seção 4.2.1.6, que
pode ser criada por meio do ambiente |epigrafe|\oarg{nome da epígrafe}.
O valor do parâmetro opcional é utilizado como entrada no \textsf{bookmark} do
PDF (\autoref{sec-bookmark}) e como título da epígrafe, que é escrito como um
título de capítulo pré-textual, ou seja, não numerado e centralizado. Caso o
parâmetro não seja informado, o conteúdo de |\epigraphname| é utilizado apenas
para entrada do \textsf{bookmark} do PDF e a epígrafe é impressa sem título e
sem indicador numérico, conforme estabelece a seção 5.2.4 da norma em tela.

Um exemplo de uso do ambiente é:

\begin{verbatim}
\begin{epigrafe}

   \vspace*{\fill}
   \begin{flushright}
      \textit{``Não vos amoldeis às estruturas deste mundo, \\
        mas transformai-vos pela renovação da mente, \\
        a fim de distinguir qual é a vontade de Deus: \\
        o que é bom, o que Lhe é agradável, o que é perfeito.\\
        (Bíblia Sagrada, Romanos 12, 2)}
    \end{flushright}
	
\end{epigrafe}
\end{verbatim}

% ---
\subsection{Resumos em língua vernácula e estrangeira (obrigatório)}
% ---

\DescribeEnv{resumo}\DescribeMacro{\abstractname}
Os resumos em língua vernácula e estrangeira são elementos obrigatórios da ABNT
NBR 14724:2011, seção 4.2.1.7 e 4.2.1.8, e devem ser escritos conforme
orientação da ABNT NBR 6028. Os elementos podem ser criados por meio do ambiente
|resumo|\oarg{nome do resumo}.
O valor do parâmetro opcional é utilizado como entrada no \textsf{bookmark} do
PDF (\autoref{sec-bookmark}) e como título do resumo. Caso o parâmetro não seja
informado, o conteúdo de |\abstractname| do pacote \textsf{babel} é utilizado. 
O parâmetro opcional é útil para criação de resumos em diversos idiomas
estrangeiros.

Exemplos de uso do ambiente são:

\begin{verbatim}
% --- resumo em português ---
\begin{resumo}
  Resumo em português
  \vspace{\onelineskip}
  \noindent
  \textbf{Palavras-chaves}: latex. abntex. editoração de texto.
\end{resumo}

% --- resumo em francês ---
\begin{resumo}[Résumé]
  Il s'agit d'un résumé en français.
  \vspace{\onelineskip}
  \noindent
  \textbf{Mots-clés}: latex. abntex. publication de textes.
\end{resumo}
\end{verbatim}


% ---
\subsection{Lista de ilustrações (opcional)}\label{sec-listadeilustracoes}
% ---

\DescribeMacro{\listoffigures}
A Lista de ilustrações é um elemento opcional da ABNT NBR 14724:2011, seção
4.2.1.9, que pode ser criada por meio da macro padrão |\listoffigures|.

Nem a classe \textsf{memoir}, nem a classe \textsf{abntex2} incluem
automaticamente o capítulo criado pela macro |\listoffigures| no
\textsf{bookmark} do PDF (\autoref{sec-bookmark}). 

O exemplo seguinte cria a Lista de ilustrações e já a adiciona ao
\textsf{bookmark}:

\begin{verbatim}
   \pdfbookmark[0]{\listfigurename}{lof}
   \listoffigures*
   \cleardoublepage
\end{verbatim}

A seção 4.2.1.9 da norma ABNT NBR 14724:2011 recomenda que, quando necessário,
seja produzido uma lista própria para cada tipo de ilustração, como
desenhos, esquemas, fluxogramas, fotografias, gráficos, mapas, organogramas,
plantas, quadros, retratos, e outros. Como essa necessidade é específica de
cada trabalho, o \abnTeX2 não traz essa implementação automaticamente.
Porém, diferentes tipos de lista podem ser criadas por meio de macros do 
\textsf{memoir}. Para isso, consulte o capítulo 9 do manual do \textsf{memoir}.

% ---
\subsection{Lista de tabelas (opcional)}\label{sec-listadetabelas}
% ---

\DescribeMacro{\listoftables}
A Lista de tabelas é um elemento opcional da ABNT NBR 14724:2011, seção
4.2.1.10, que pode ser criada por meio da macro padrão |\listoftables|.

Nem a classe \textsf{memoir}, nem a classe \textsf{abntex2} incluem
automaticamente o capítulo criado pela macro |\listoftables| no
\textsf{bookmark} do PDF (\autoref{sec-bookmark}). 

O exemplo seguinte cria a Lista de tabelas e já a adiciona ao
\textsf{bookmark}:

\begin{verbatim}
   \pdfbookmark[0]{\listtablename}{lot}
   \listoftables*
   \cleardoublepage
\end{verbatim}

% ---
\subsection{Lista de abreviaturas e siglas (opcional)}
\label{sec-listadeabreviaturas}
% ---

\DescribeMacro{\printnomenclature}
O \abnTeX2 não traz uma implementação própria para a Lista de abreviaturas e
siglas, elemento opcional da ABNT NBR 14724:2011, seção 4.2.1.11. Um dos motivos
da não inclusão desse recurso é a existência de diversos pacotes que o fazem,
cada um com uma característica diferente.

Uma sugestão é usar o pacote
\textsf{nomencl}\footnote{\url{http://www.ctan.org/tex-archive/macros/latex/contrib/nomencl/}}.
Com esse pacote, você pode criar a lista de abreviaturas e siglas seguindo o
exemplo abaixo, que define algumas siglas, imprime um capítulo com o título
``Lista de abreviaturas e siglas'' e já o insere no \textsf{bookmark} do PDF
(\autoref{sec-bookmark}):

\begin{verbatim}
   \nomenclature{Fig.}{Figura}
   \nomenclature{$A_i$}{Area of the $i^{th}$ component} 
   \nomenclature{456}{Isto é um número}
   \nomenclature{123}{Isto é outro número}
   \nomenclature{a}{primeira letra do alfabeto}
   \nomenclature{lauro}{este é meu nome} 

   \renewcommand{\nomname}{Lista de abreviaturas e siglas}
   \pdfbookmark[0]{\nomname}{las}
   \printnomenclature
   \cleardoublepage
\end{verbatim}

% ---
\subsection{Lista de símbolos (opcional)}\label{sec-listadesimbolos}
% ---

Assim como o \abnTeX2 não traz uma implementação própria para a Lista de
abreviaturas e siglas, uma implementação para Lísta de símbolos também não é
oferecida. A lista de símbolos é um elemento opcional da ABNT NBR 14724:2011,
seção 4.2.1.12.

Um dos motivos da não inclusão desse recurso é a existência de diversos pacotes
que o fazem, cada um com uma característica diferente.

Como sugestão, consulte o pacote \textsf{glossaries}, que tanto pode construir a
Lista de símbolos, como Glossários. A \autoref{sec-glossarios} aborda esse
tema. 

% ---
\subsection{Sumário (obrigatório)}\label{sec-sumario}
% ---

\DescribeMacro{\tableofcontents}
O Sumário é um elemento obrigatório da ABNT NBR 14724:2011, seção
4.2.1.13, que pode ser criada por meio da macro padrão |\tableofcontents|.

Nem a classe \textsf{memoir}, nem a classe \textsf{abntex2} incluem
automaticamente o capítulo criado pela macro |\tableofcontents| no
\textsf{bookmark} do PDF. Caso deseje que o título do capítulo seja
incluído no \textsf{bookmark} (\autoref{sec-bookmark}), utilize o exemplo
abaixo:

\begin{verbatim}
   \pdfbookmark[0]{\contentsname}{toc}
   \tableofcontents*
   \cleardoublepage
\end{verbatim}

\DescribeMacro{\settocdepth}
Você pode customizar o nível de divisões que o sumário pode listar com a
macro |\settocdepth|\marg{nome da subdivisão}, sendo |nome da subdivisão| um
dos valores: |chapter|, |part|, |section|, |subsection|, |subsubsection|.

A configuração padrão do \abnTeX2 é |\settocdepth{subsubsection}|.

\DescribeMacro{\setsecnumdepth}
Também é possível customizar se a numeração das divisões é exibida no
sumário. Para isso use |\setsecnumdepth|\marg{nome da subdivisão}, sendo
|nome da subdivisão| um dos mesmos valores utilizados em |\settocdepth|. 

A configuração padrão do \abnTeX2 é |\setsecnumdepth{subsubsection}|.

\DescribeMacro{\tocheadstart}
É possível customizar a fonte das partes e dos capítulos no Sumário. Para isso,
redefina a macro |\tocheadstart|. O \abnTeX2 a redefine por padrão para que a
fonte utilizada no Sumário seja a mesma defina para o capítulo, da seguinte
maneira:

\begin{verbatim}
   \renewcommand{\tocheadstart}{\ABNTEXchapterfont}
\end{verbatim}

\DescribeMacro{\addcontentsline}
O comando |\addcontentsline|\marg{sigla do sumario}\marg{nível da
divisão}\marg{texto no sumário} pode ser usado para incluir uma linha no
Sumário. Use o comando, por exemplo, após a criação de capítulo não numerado: 

\begin{verbatim}
   \chapter*{Introdução}
   \addcontentsline{toc}{chapter}{Introdução}
\end{verbatim}

É importante destacar que nenhum elemento pré-textual deve estar presente no
Sumário. Veja mais informações na \autoref{sec-elementostextuais}.

Consulte a \autoref{sec-bookmark} para obter informações sobre o
\textsf{bookmark}, índice da estrutura do documento no PDF.

\DescribeMacro{KeepFromToc}
O ambiente |KeepFromToc| pode ser utilizada para que um divisão não seja
incluída no Sumário. Esse ambiente é equivalente à macro |\ProximoForaDoSumario|
utilizada pela versão anterior do \abnTeX, e que não está mais presente nesta
versão.

Use a macro como no exemplo:

\begin{verbatim}
   \begin{KeepFromToc}
      \chapter{Este capítulo não aparece no sumário}
      \section{Nem esta seção}
   \end{KeepFromToc}
\end{verbatim}

% ------
\section{Elementos textuais}\label{sec-elementostextuais}
% ------

\DescribeMacro{\textual}\DescribeMacro{\mainmatter}
O comando |\textual| identifica o início dos elementos textuais. As páginas
desses elementos são numeradas com algarismos arábicos no lado direito superior
ou direito/esquerdo superior caso a impressão frente e verso (opção |twoside|)
seja acionada, conforme estabelece a ABNT NBR 14724:2011. Geralmente a
``Introdução'' é o primeiro capítulo textual. A título de coerência, a macro
|\mainmatter|, padrão do \textsf{memoir}, é reescrita para que tenha o mesmo
comportamento que |\textual|. Por isso, fique livre em escolher qualquer das
macros. Porém, o uso de uma delas é obrigatória, para que os cabeçalhos sejam
montados corretamente.

Segundo a ABNT NBR 14724:2011, seção 4.2.2, ``o texto é composto de uma parte
introdutória, que apresenta os objetivos do trabalho e as razões de sua
elaboração; o desenvolvimento, que detalha a pesquisa ou estudo realizado; e uma
parte conclusiva.'' Os títulos dos capítulos textuais são à critério do
autor e não há nenhuma normatização a respeito deles. No entanto, geralmente o
capítulo ``Introdução'' e o capítulo ``Conclusão'' (ou ``Considerações
finais'') são, respectivamente, o primeiro e o último capítulo textual
e normalmente não são numerados. 

É importante destacar que a norma em tela e a ABNT NBR 6024:2012 não são
explícitas sobre a possibilidade de não numeração de capítulos
textuais\footnote{Embora a seção 5.2.3 da ABNT NBR 14724:2011 seja clara a
respeito dos capítulos pré-textuais: ``Os títulos, sem indicativo numérico ---
errata, agradecimentos, lista de ilustrações, lista de abreviaturas e siglas,
lista de símbolos, resumos, sumário, referências, glossário, apêndice(s),
anexo(s) e índice(s) – devem ser centralizados''.}. Desse modo, sugere-se que se
siga o modo de numeração desses capítulos utilizado pela instituição que você
apresentará o trabalho.

\DescribeMacro{\chapter*}\DescribeMacro{\addcontentsline}
Caso deseje incluir capítulos sem numeração (como Introdução e Conclusão),
utilize a macro |\chapter*|\marg{Introdução}. Porém, capítulos com * não são
incluídos automaticamente no Sumário nem no \textsf{bookmark} do PDF. Para
incluí-los, utilize o comando:

\begin{verbatim}
   \chapter*{Introdução}
   \addcontentsline{toc}{chapter}{Introdução}
\end{verbatim}

\DescribeMacro{\part}
A macro |\part|\marg{nome da parte} pode ser utilizada para que uma
página de divisão do trabalho seja incluída. A parte agrupa capítulos. Um
exemplo é o uso do trabalho dividido em três partes:

\begin{verbatim}
   \part{Preparação da Pesquisa}
     (...)
     \chapter{Metodologia}
     (...)
   \part{Revisão de Literatura}
     (...)
     \chapter{O trabalho de Charles Darwin}
     (...)
   \part{Resultados}
     (...)
\end{verbatim}

% ---
\subsection{Formatação de partes, capítulos, seções, subseções e
subsubseções}\label{sec-formatacaocapitulos}
% ---

\DescribeMacro{\ABNTEXchapterfont}\DescribeMacro{\ABNTEXchapterfontsize}
O \textsf{memoir} possui uma vasta lista de opções de estilos de capítulos
(ver \autoref{sec-formatacaocapitulos-adicionais}).
A classe \textsf{abntex2} adiciona a essa lista um estilo chamado |abnt|, que
atende aos requisitos na ABNT NBR 14724:2011 e da ABNT NBR 6024:2012. O estilo
|abnt| é carregado automaticamente e possui duas configurações adicionais:
|\ABNTEXchapterfont| é a fonte utilizada nos capítulos e
|\ABNTEXchapterfontsize| é o tamanho da fonte. 

Você pode customizar a fonte alterando os comandos como no exemplo a seguir,
para que seja utilizada a fonte \emph{Computer Modern} com tamanho maior do
que o utilizado por padrão:

\begin{verbatim}
   \renewcommand{\ABNTEXchapterfont}{\fontfamily{cmr}\fontseries{b}\selectfont}
   \renewcommand{\ABNTEXchapterfontsize}{\HUGE}
\end{verbatim}

\DescribeMacro{\ABNTEXpartfont}\DescribeMacro{\ABNTEXpartfontsize}
\DescribeMacro{\ABNTEXsectionfont}\DescribeMacro{\ABNTEXsectionfontsize}
\DescribeMacro{\ABNTEXsubsectionfont}\DescribeMacro{\ABNTEXsubsectionfontsize}
\DescribeMacro{\ABNTEXsubsubsectionfont}\DescribeMacro{\ABNTEXsubsubsectionfontsize}

As fontes e o tamanho das fontes obtidas com as divisões |\part|\marg{nome da
parte}, |\chapter|\marg{nome do capítulo}, |\section|\marg{nome da seção},
|\subsection|\marg{nome da subseção} e |\subsubsection|\marg{nome da subsubseção} são definidas por padrão,
respectivamente, conforme a \autoref{tab-divisoes}. Você pode alterá-las com o
comando |\renewcommand|

\begin{table}[htb]
\caption{Macros de formatação de fonte de divisões do texto}
\label{tab-divisoes}
\centering
\begin{tabular}{ l l l }
   \textbf{Macro} & \textbf{Valor padrão} \\
    \hline
    |\ABNTEXchapterfont| & |\fontfamily{cmss}\fontseries{sbc}\selectfont| \\
    \hline
    |\ABNTEXchapterfontsize| & |\Huge| \\
    \hline
    
    |\ABNTEXpartfont| & |\ABNTEXchapterfont| \\
    \hline
    |\ABNTEXpartfontsize| & |\ABNTEXchapterfontsize| \\
    \hline
    
    |\ABNTEXsectionfont| & |\ABNTEXchapterfont| \\
    \hline
    |\ABNTEXsectionfontsize| & |\Large| \\
    \hline

    |\ABNTEXsubsectionfont| & |\ABNTEXsectionfont| \\
    \hline
    |\ABNTEXsubsectionfontsize| & |\large| \\
    \hline

    |\ABNTEXsubsubsectionfont| & |\ABNTEXsubsectionfont| \\
    \hline
    |\ABNTEXsubsubsectionfontsize| & |\normalsize| \\
    \hline
    \hline

\end{tabular}

\end{table}

% ---
\subsubsection{Estilos adicionais de capítulos}
\label{sec-formatacaocapitulos-adicionais}
% ---

\DescribeMacro{\chapterstyle}
O estilo de capítulo |abnt| provido pela classe \abnTeX2 pode ser substituído
por outro estilo já fornecido pelo \textsf{memoir} ou mesmo por outro criado por
você. Isso é útil especialmente se estiver interessado em publicar seu trabalho
como livro ou não se importar em não seguir o padrão normativo. Para isso,
utilize o comando:

\begin{verbatim}
   \chapterstyle{nome_do_estilo}
\end{verbatim}

Experimente |lyhne| ou |dash|. Você encontra alguns estilos no manual do
\textsf{memoir} e outros neste documento:
\url{http://www.tex.ac.uk/tex-archive/info/MemoirChapStyles/MemoirChapStyles.pdf}.
Ambos mostram como criar um novo estilo.

% ---
\subsection{Citações diretas com mais de três linhas}
% ---

\DescribeEnv{citacao}
A ABNT NBR 10520:2002, seção 5.3, descreve que citações diretas com mais de três
linhas devem ser destacadas com recuo de 4 cm da margem esquerda, com letra
menor que a do texto utilizado e sem as aspas. Para inserir citações longas,
utilize o ambiente |citacao|, conforme o exemplo:

\begin{verbatim}
\begin{citacao}
As citações diretas, no texto, com mais de três linhas, devem ser
destacadas com recuo de 4 cm da margem esquerda, com letra menor que a do texto
utilizado e sem as aspas. No caso de documentos datilografados, deve-se
observar apenas o recuo \cite[5.3]{NBR10520:2002}
\end{citacao}
\end{verbatim}

O tamanho da fonte utilizada no ambiente |citacao| é determinada pela macro
|\ABNTEXfontereduzida|, descrita na \autoref{sec-configgerais}.

% ---
\subsection{Alíneas e Subalíneas}
% ---

A ABNT NBR 6024:2012, seção 4.2, descreve o uso das alíneas, que podem ser
compreendias como subdivisões não nomeadas de uma seção. 

As alíneas são numeradas com letras minúsculas do alfabeto com recuo em relação
à margem esquerda do documento. A norma prescreve que o texto que antecede as
alíneas deve finalizar com dois pontos (:); as alíneas devem iniciar com letra
minúscula e serem finalizadas com ponto e vírgula (;), exceto a última alínea,
que deve ser finalizada com ponto final, e exceto as alíneas que precederem uma
subalínea, caso em que devem ser finalizadas com dois pontos (:); a segunda e as
seguintes linhas do texto da alínea começa sob a primeira letra do texto da
própria alínea. 

\DescribeEnv{alineas}
A classe \textsf{abntex2} fornece o ambiente |alineas|, que cria listas conforme
o padrão estipulado pela norma. Veja o exemplo:

\begin{verbatim}
\begin{alineas}
  \item linha 1;
  \item linha 2;
  \item linha 3.
\end{alineas}   
\end{verbatim}
  
\DescribeEnv{subalineas}\DescribeEnv{incisos}
As alíneas podem ser aninhadas. Nesse caso, a numeração é substituída por um
travessão. Você pode criar uma subalínea de três formas diferentes, todas
equivalentes entre si: com outro ambiente |alineas|, com |subalineas| ou
ainda com o ambiente |incisos|:

\begin{verbatim}
\begin{alineas}

  \item linha 1:
  \begin{alineas}
    \item subalinea 1;
    \item subalinea 2;
  \end{alineas}   

  \item linha 2:
  \begin{subalineas}
    \item subalinea 1;
    \item subalinea 2;
  \end{subalineas}

  \item linha 3:   
  \begin{incisos}
    \item subalinea 1;
    \item subalinea 2;
  \end{incisos}

  \item linha 4.
\end{alineas}   
\end{verbatim}

% ---
\subsection{Rótulos e legendas}
% ---

\DescribeMacro{\caption}\DescribeMacro{\legend}
Rótulos e legendas de ilustrações, tabelas e qualquer outro ambiente do tipo
|listing| podem ser definidos pelos comandos |\caption|\marg{rotulo} e
|\legend|\marg{legenda}, respectivamente.

Conforme a ABNT NBR 14724:2011, seção 5.8, o rótulo é atribuído acima do
elemento e a legenda abaixo, conforme no exemplo:

\begin{verbatim}
\begin{figure}[htb]
	\caption{\label{fig_circulo}A delimitação do espaço}
	\begin{center}
	    \includegraphics[scale=0.75]{myfig.pdf}
	\end{center}
	\legend{Fonte: os autores}
\end{figure}
\end{verbatim}

\DescribeMacro{\includegraphics}
A macro |\includegraphics| pode ser utilizada para inclusão de imagens.
Recomenda-se que imagens vetoriais, como imagens em PDF, sejam preferidas em
oposição a imagens baseadas em mapas de bits, uma vez que desse forma não há
perda de qualidade nas imagens. Porém, formatos como PNG, BMP, JPG e outros são
aceitos pelo \LaTeX.

% ------
\section{Elementos pós-textuais}
% ------

\DescribeMacro{\postextual}\DescribeMacro{\backmatter}
O comando |\postextual| identifica o início dos elementos pós-textuais. Na
prática não há nenhum comportamento específico, uma vez que as
normas não prescrevem nenhum requisito para esses elementos. Porém, mesmo
que para uso futuro, a macro |\postextual| já está criada e recomenda-se que
seja utilizada. Dessa forma, caso deseje atribuir algum comportamento
diferenciado aos elementos pós-textuais, faça-o redefinindo a macro. A título de
coerência, a macro |\backmatter|, padrão do \textsf{memoir}, é reescrita para
que tenha o mesmo comportamento que |\postextual|.

% ---
\subsection{Referências (obrigatório)}
% ---

\DescribeMacro{\bibliography}
A classe \textsf{abntex2} é responsável pela estruturação e o aspecto geral do
trabalho. Mais precisamente, ela é focada em atender os requisitos apresentados
pela norma ABNT NBR 14724:2011 e ABNT NBR 6024:2012. As referências
bibliográficas são normatizadas pela norma ABNT NBR 10520:2002 e os requisitos
impostos pela norma são atendidos pelo pacote \textsf{abntex2cite}, integrante
do \abnTeX2, mas não abordado por este manual.

Para utilizar o padrão de bibliografias brasileiro implementado pelo pacote
\textsf{abntex2cite}, declare no preâmbulo do documento:

\begin{verbatim}
   \usepackage[alf]{abntex2cite}	% Citações padrão ABNT
\end{verbatim}

A opção |alf| indica que as referências serão alfanuméricas, no padrão
autor-ano. Ela se opõe à opção |num|, que indica que as referências serão
numéricas. Consulte o manual do pacote \textsf{abntex2cite} para informações
detalhadas.

Para indicar o local de impressão da bibliografia, utilize:

\begin{verbatim}
   \bibliography{arquivo-de-referencias-bib}
\end{verbatim}

Você pode usar tanto a classe \textsf{abntex2} quanto o pacote de citações
\textsf{abntex2cite} de forma independente. Isso é útil, por exemplo, quando se
está escrevendo um documento baseado em outra classe fornecida por uma
instituição, por exemplo, mas se deseja manter o padrão de citações brasileiro,
ou então, deseja-se utilizar as customizações da classe \textsf{abntex2} com
outro padrão de referências bibliográficas.

\DescribeEnv{backref}
O \textsf{abntex2cite} é compatível com o pacote
\textsf{backref}\footnote{\url{http://www.ctan.org/pkg/backref}}, que permite
que a bibliografia indique quantas vezes e em quais páginas a citação ocorreu.
Para isso, adicione ao preâmbulo:

\begin{verbatim}
   \usepackage[brazilian,hyperpageref]{backref}	 
\end{verbatim}

Ainda no preâmbulo, você pode configurar como o pacote \textsf{backref} deverá
imprimir as referências:

\begin{verbatim}
   % Configurações do pacote backref
   % Usado sem a opção hyperpageref de backref
   \renewcommand{\backrefpagesname}{Citado na(s) página(s):~}
   
   % Texto padrão antes do número das páginas
   \renewcommand{\backref}{}
   
   % Define os textos da citação
   \renewcommand*{\backrefalt}[4]{
	   \ifcase #1 %
		   Nenhuma citação no texto.%
	   \or
		   Citado na página #2.%
	   \else
	   	   Citado #1 vezes nas páginas #2.%
	   \fi}%
\end{verbatim}

% ---
\subsection{Glossário (opcional)}\label{sec-glossarios}
% ---

O \abnTeX2 não traz uma implementação própria para o Glossário,
elemento opcional estabelecido pela ABNT NBR 14724:2011. 

Um dos motivos da não inclusão desse recurso é a existência de diversos pacotes
que o fazem, cada um com uma característica diferente.

Como sugestão, consulte o pacote
\textsf{glossaries}\footnote{\url{http://www.ctan.org/tex-archive/macros/latex/contrib/glossaries}},
que tanto pode construir Glossários como a Lista de símbolos
(\autoref{sec-listadesimbolos}). 

O portal \emph{\LaTeX Community} (\url{http://www.latex-community.org}) possui
um guia de uso do pacote
\textsf{glossaries}\footnote{\url{http://www.latex-community.org/know-how/latex/55-latex-general/263-glossaries-nomenclature-lists-of-symbols-and-acronyms}q}
e também um excelente
artigo\footnote{\url{http://www.latex-community.org/know-how/456-glossary-without-makeindex}}
que mostra como criar um pacote próprio de gerenciamento de glossários que não
necessita de nenhum utilitário externo, uma vez que \textsf{glossaries} requer
os aplicativos \textsf{makeindex} e \textsf{makeglossaries}.

% ---
\subsection{Apêndices (opcional)}
% ---

\DescribeMacro{\apendices}\DescribeEnv{apendicesenv}
O início dos apêndices, elementos opcionais da ABNT NBR 14724:2011, seção
4.2.3.3, deve ser marcado com a macro |\apendices|, ou os apêndices devem estar
contidos no ambiente |apendicesenv|. Os apêndices devem preceder os anexos, caso
esses existam.

\DescribeMacro{\apendicename}
Os apêndices devem ser iniciados com a macro |\chapter|\marg{nome do apêndice},
que imprime o nome do apêndice precedido do conteúdo da macro |\apendicename|,
cujo conteúdo padrão é |AP\^ENDICE|.

\DescribeMacro{\apendicesname}\DescribeMacro{\appendixpage}
No contexto dos apêndices, a macro |\appendixpage| imprime o conteúdo da macro
|\apendicesname| como se fosse uma divisão de partes obtida com |\part| não
numerada. A variante |\appendixpage*| não inclui divisão no Sumário nem no
\textsf{bookmark} do PDF.

% ---
\subsection{Anexos (opcional)}
% ---

\DescribeMacro{\anexos}\DescribeEnv{anexosenv}
O início dos anexos, elementos opcionais da ABNT NBR 14724:2011, seção
4.2.3.4, deve ser marcado com a macro |\anexos|, ou os anexos devem estar
contidos no ambiente |anexosenv|. Os anexos devem vir dispostos após os
apêndices, caso esses existam.

\DescribeMacro{\anexoname}
Os anexos devem ser iniciados com a macro |\chapter|\marg{nome do anexo},
que imprime o nome do anexo precedido do conteúdo da macro |\anexoname|, cujo
conteúdo padrão é |ANEXO|.

\DescribeMacro{\anexosname}\DescribeMacro{\appendixpage}
No contexto dos anexos, a macro |\appendixpage| imprime o conteúdo da macro
|\apendicesname| como se fosse uma divisão de partes obtida com |\part| não
numerada. A variante |\appendixpage*| não inclui a divisão no Sumário nem no
\textsf{bookmark} do PDF.

% ---
\subsection{Índice (opcional)}
% ---

\DescribeMacro{\printindex}\DescribeMacro{\index}
O índice, elemento opcional da ABNT NBR 14724:2011, deve ser elabora conforme a
ABNT NBR 6034 e pode ser produzido por meio da macro |\printindex|, que imprime
as páginas nas quais as macros |\index|\marg{palavra a ser indexada} apareceram.

Para que as macros |\printindex| e |\index| funcionem, é preciso utilizar o
compilador
\textsf{MakeIndex}\footnote{\url{http://www.tex.ac.uk/ctan/indexing/makeindex/}}.

% ------
\section{Mais informações}
% ------

Para mais informações, consulte o site do projeto em \abnTeXSite. Poste suas
dúvidas e comentários no Fórum de discussão do \abnTeX2 em \abnTeXForum.


% ------
\PrintChanges
\PrintIndex
% ------

% ------
\StopEventually{\PrintIndex}
% ------

\StopEventually{\PrintIndex}
\end{document}