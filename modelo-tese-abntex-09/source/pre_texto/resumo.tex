%~~~~~~~~~~~~~~~~~~~~~~~~~~~~~~~~~~~~~~~~~~~~~~~~~~~~~~~~~~~~~~~~~~~~~
%
%    File      : resumo
%    Type      : TeX
%    Date      : terça-feira, março 19, 2012 at 10:06
%
%    Content   : Coloque aqui o resumo em de sua tese/dissertação.
%~~~~~~~~~~~~~~~~~~~~~~~~~~~~~~~~~~~~~~~~~~~~~~~~~~~~~~~~~~~~~~~~~~~~~

%    Conforme as normas NBR 14724:2002 da ABNT, o resumo é ``elemento
%    obrigatório,  constituído de  uma seqüência  de frases  concisas e
%    objetivas  e  não  de  uma  simples  enumeração  de  tópicos,  não
%    ultrapassando  500 palavras,  seguido, logo  abaixo,  das palavras
%    representativas  do conteúdo do  trabalho, isto  é, palavras-chave
%    e/ou descritores (...)''.

%    O resumo deve ser elaborado apresentando os objetivos de
%    maneira sucinta e objetiva (duas linhas no máximo), dando ênfase
%    aos principais resultados obtidos no trabalho. A seguinte
%    estrutura deve ser respeitada: apresentação do problema e
%    objetivos; metodologia; resultados; discussão dos resultados.


\begin{resumo}

    Conforme as normas NBR 14724:2002 da ABNT, o resumo é ``elemento
    obrigatório,  constituído de  uma sequência  de frases  concisas e
    objetivas  e  nÐo  de  uma  simples  enumeração  de  tópicos,  não
    ultrapassando  500 palavras,  seguido, logo  abaixo,  das palavras
    representativas  do conteúdo do  trabalho, isto  é, palavras-chave
    e/ou descritores (...)''.

    O resumo deve ser elaborado apresentando os objetivos de
    maneira sucinta e objetiva (duas linhas no máximo), dando ênfase
    aos principais resultados obtidos no trabalho. A seguinte
    estrutura deve ser respeitada: apresentação do problema e
    objetivos; metodologia; resultados; discussão dos resultados.

\textbf{Palavras-chave:}
    (Você deve escolher suas próprias palavras-chave, de acordo com o
    seu trabalho. Umas cinco palavras serão suficientes)
	\palavrasChavesPortugues
\end{resumo}