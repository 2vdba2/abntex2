%% LyX 2.0.6 created this file.  For more info, see http://www.lyx.org/.
%% Do not edit unless you really know what you are doing.
\documentclass[oneside,english,12pt,openright,twoside,a4paper,english,french,spanish,brazil]{abntex2}
\usepackage[T1]{fontenc}
\usepackage[utf8]{inputenc}
\setcounter{secnumdepth}{3}
\setcounter{tocdepth}{3}
\usepackage{verbatim}
\usepackage{float}
\usepackage{url}
\usepackage{pdfpages}
\usepackage[authoryear]{natbib}

\makeatletter

%%%%%%%%%%%%%%%%%%%%%%%%%%%%%% LyX specific LaTeX commands.
\providecommand{\LyX}{L\kern-.1667em\lower.25em\hbox{Y}\kern-.125emX\@}
%% Because html converters don't know tabularnewline
\providecommand{\tabularnewline}{\\}
%% A simple dot to overcome graphicx limitations
\newcommand{\lyxdot}{.}


%%%%%%%%%%%%%%%%%%%%%%%%%%%%%% User specified LaTeX commands.
% ---
% PACOTES
% ---
\usepackage{lastpage}			% Usado pela Ficha catalográfica
\usepackage{indentfirst}		% Indenta o primeiro parágrafo de cada seção.
\usepackage{color}				% Controle das cores

% ---
% Pacotes adicionais, usados apenas no âmbito do Modelo Canônico do abnteX2
% ---
\usepackage{lipsum}				% para geração de dummy text
% ---

% ---
% Pacotes de citações
% ---
\usepackage[brazilian,hyperpageref]{backref}	 % Paginas com as citações na bibl
\usepackage[alf]{abntex2cite}	% Citações padrão ABNT

% --- 
% CONFIGURAÇÕES DE PACOTES
% --- 

% ---
% Configurações do pacote backref
% Usado sem a opção hyperpageref de backref
\renewcommand{\backrefpagesname}{Citado na(s) página(s):~}
% Texto padrão antes do número das páginas
\renewcommand{\backref}{}
% Define os textos da citação
\renewcommand*{\backrefalt}[4]{
\ifcase #1 %
Nenhuma citação no texto.%
\or
Citado na página #2.%
\else
Citado #1 vezes nas páginas #2.%
\fi}%
% ---

% --- 
% Espaçamentos entre linhas e parágrafos 
% --- 

% O tamanho do parágrafo é dado por:
\setlength{\parindent}{1.3cm}

% Controle do espaçamento entre um parágrafo e outro:
\setlength{\parskip}{0.2cm}  % tente também \onelineskip

% ---
% compila o indice
% ---
\makeindex
% ---

\makeatother

\usepackage{babel}
\begin{document}
\begin{comment}
Os valores colocados nos campos abaixo \textbf{não serão impressos}
ao gerar o documento final. Esses campos somente modificam os valores
para as macros do documento, as quais serão usadas posteriormente
dentro do esqueleto da tese. Esta parte do documento é como se fosse
a ``ficha cadastral'' da tese.
\end{comment}



\titulo{Modelo Canônico de\\
Trabalho Acadêmico com abn\TeX{}2 e \LyX{}}




\local{Universidade do Brasil -- UBr\\
Faculdade de Arquitetura da Informação\\
Programa de Pós-Graduação}




\data{6 de Junho de 2013}




\autor{Paulo Afonso Graner Fessel}


\orientador{Lauro César Araújo}


\coorientador{Sérgio Granja}


\coorientador[Co-Orientadores]{Equipe \abnTeX\rule[0.5ex]{1\columnwidth}{1pt}}


\tipotrabalho{Dissertação de Mestrado}

\begin{comment}
A partir daqui, inicia-se o texto impresso
\end{comment}


\imprimircapa

\imprimirfolhaderosto*
\begin{fichacatalografica}


\begin{minipage}[c][12.5cm]{1\textwidth}%
\rule[0.5ex]{1\columnwidth}{1pt}

\imprimirautor

\hspace{0.5cm}\imprimirtitulo / \imprimirautor. -- \imprimirlocal,
\imprimirdata-

\hspace{0.5cm}\pageref{LastPage} p. il. (algumas color.) ; 30 cm.

\hspace{0.5cm}\imprimirorientadorRotulo~\imprimirorientador\\


\parbox[c][1\totalheight][t]{1\textwidth}{%
\hspace{0.5cm}\imprimirtipotrabalho~--~\imprimirinstituicao,
\imprimirdata.\\


\hspace{0.5cm}1. Palavra-chave1. 2. Palavra-chave2. I. Orientador.
II. Universidade xxx. III. Faculdade de xxx. IV. Título\\


\hspace{8.75cm} CDU 02:141:005.7%
}%
\end{minipage}

\rule[0.5ex]{1\columnwidth}{1pt}\end{fichacatalografica}
\begin{errata}
Elemento opcional da \citeauthor{NBR14724:2001}. Exemplo:\\
\\
FERRIGNO, C. R. A. \textbf{Tratamento de neoplasias ósseas apendiculares
com reimplantação de enxerto ósseo autólogo autoclavado associado
ao plasma rico em plaquetas}: estudo crítico na cirurgia de preservação
de membro em cães. 2011. 128 f. Tese (Livre-Docência) - Faculdade
de Medicina Veterinária e Zootecnia, Universidade de São Paulo, São
Paulo, 2011.\\
\begin{tabular}{|l|l|l|l|}
\hline 
\textbf{Folha} & \textbf{Linha} & \textbf{Onde se lê} & \textbf{Leia-se}\tabularnewline
\hline 
\hline 
1 & 10 & auto-conclavo & autoconclavo\tabularnewline
\hline 
\end{tabular}\end{errata}
\begin{folhadeaprovacao}
\includepdf[pages=1]{FolhaDeAprovacao}\end{folhadeaprovacao}
\begin{dedicatoria}
Este trabalho é dedicado às crianças adultas que,\\
 quando pequenas, sonharam em se tornar cientistas.\end{dedicatoria}
\begin{agradecimentos}
Os agradecimentos principais são direcionados à Gerald Weber, Miguel
Frasson, Leslie H. Watter, Bruno Parente Lima, Flávio de Vasconcellos
Corrêa, Otavio Real Salvador, Renato Machnievscz%
\footnote{Os nomes dos integrantes do primeiro projeto \abnTeX foram extraídos
de \url{http://codigolivre.org.br/projects/abntex/}%
} e todos aqueles que contribuíram para que a produção de trabalhos
acadêmicos conforme as normas ABNT com \LaTeX{} fosse possível.

Agradecimentos especiais são direcionados ao Centro de Pesquisa em
Arquitetura da Informação%
\footnote{\url{http://www.cpai.unb.br/}%
} da Universidade de Brasília (CPAI), ao grupo de usuários \textbf{latex-br}%
\footnote{\url{http://groups.google.com/group/latex-br}%
} e aos novos voluntários do grupo abn\TeX{}2 %
\footnote{\url{http://groups.google.com/group/abntex2} e \url{http://abntex2.googlecode.com/}%
} que contribuíram e que ainda contribuirão para a evolução do abn\TeX{}.\end{agradecimentos}
\begin{epigrafe}
\textit{``Não vos amoldeis às estruturas deste mundo,}\\
\textit{mas transformai-vos pela renovação da mente,}\\
\textit{a fim de distinguir qual é a vontade de Deus:}\\
\textit{o que é bom, o que Lhe é agradável, o que é perfeito}.\\
(Bíblia Sagrada, Romanos 12, 2)
\end{epigrafe}
\bibliographystyle{/usr/share/texlive/texmf-dist/bibtex/bst/abntex2/abntex2-alf}
\bibliography{abntex2-modelo-references}

\end{document}
