%% abtex2-modelo.tex
%% Copyright 2012-2013 by abnTeX2 group at http://code.google.com/p/abntex2/ 
%%
%% This work may be distributed and/or modified under the
%% conditions of the LaTeX Project Public License, either version 1.3
%% of this license or (at your option) any later version.
%% The latest version of this license is in
%%   http://www.latex-project.org/lppl.txt
%% and version 1.3 or later is part of all distributions of LaTeX
%% version 2005/12/01 or later.
%%
%% This work has the LPPL maintenance status `maintained'.
%% 
%% The Current Maintainer of this work is the abnTeX2 team, led
%% by Lauro César Araujo. Further information are available on 
%% http://code.google.com/p/abntex2/
%%
%% This work consists of the files abntex2-modelo.tex,
% abntex2-modelo-errata.tex, abntex2-modelo-fichacatalografica.tex,
% abntex2-modelo-folhadeaprovacao.tex and abntex2-modelo-references.bib

% ------------------------------------------------------------------------
% ------------------------------------------------------------------------
% Modelo de Trabalho Acadêmico utilizando abnTeX2 
% tese de doutorado, dissertação de mestrado e trabalhos monográficos em geral
% ------------------------------------------------------------------------
% ------------------------------------------------------------------------
%
\documentclass[12pt,openright,twoside,a4paper]{abntex2}
%\documentclass[12pt,oneside,a4paper]{abntex2}

% ----------------------------------------------------------
% PACOTES
% ----------------------------------------------------------

% ---
% Pacotes fundamentais 
% ---
\usepackage{cmap}				% Mapear caracteres especiais no PDF
\usepackage[T1]{fontenc}		% seleção de códigos de fonte.
\usepackage[utf8]{inputenc}		% determina a codificação utiizada (conversão automática dos acentos)
\usepackage{makeidx}            % cria o indice
\usepackage{hyperref}  			% controla a formação do índice
\usepackage{lastpage}			% usado por abntex2-fichacatalografica.tex
\usepackage{indentfirst}		% Identa o primeiro parágrafo de cada seção.
\usepackage{nomencl} 			% Lista de simbolos
												
% ---
% PACOTES ADICIONAIS, APENAS PARA TESTE DE FUNCIONALIDADES
% ---
\usepackage{lipsum}				% para geração de dummy text
\usepackage{mychemistry}	    % Desenho de estruturas químicas

% ---
% Pacotes de citações
% ---
\usepackage[brazilian,hyperpageref]{backref}	 % Paginas com as citações na bibl
\usepackage[alf]{abntex2cite}	% Citações padrão ABNT

% ---
% Configurações do pacote backref
% Usado sem a opção hyperpageref de backref
\renewcommand{\backrefpagesname}{Citado na(s) página(s):~}
% Texto padrão antes do número das páginas
\renewcommand{\backref}{}
% Define os textos da citação
\renewcommand*{\backrefalt}[4]{
	\ifcase #1 %
		Nenhuma citação no texto.%
	\or
		Citado na página #2.%
	\else
		Citado #1 vezes nas páginas #2.%
	\fi}%
% ---

% ---
% Informações de dados para CAPA e FOLHA DE ROSTO
% ---
\titulo{Modelo Canônico de\\ Trabalhos Acadêmicos com \abnTeX}
\autor{Equipe \abnTeX}
\local{Brasil}
\data{2012}
\orientador{Lauro César Araujo}
\coorientador{Equipe \abnTeX}
\instituicao{%
  Universidade do Brasil -- UBr
  \par
  Faculdade de Arquitetura da Informação
  \par
  Programa de Pós-Graduação}
\tipotrabalho{Tese (Doutorado)}
% O preambulo deve conter o tipo do trabalho, o objetivo, 
% o nome da instituição e a área de concentração 
\preambulo{Modelo canônico de trabalho monográfico acadêmico em conformidade com
as normas ABNT apresentado à comunidade de usuários \LaTeX.}
% ---

% ---
% Configurações de aparência do PDF final

% alterando o aspecto da cor azul
\definecolor{blue}{RGB}{41,5,195}

% informações do PDF
\hypersetup{
     	%backref=true,
     	%pagebackref=true,
		%bookmarks=true,         		% show bookmarks bar?
		pdftitle={\imprimirtitulo}, 
		pdfauthor={\imprimirautor},
    	pdfsubject={\imprimirpreambulo},
		pdfkeywords={PALAVRAS}{CHAVES}{EM}{PORTUGUES},
	    pdfproducer={LaTeX with abnTeX2}, 	% producer of the document
	    pdfcreator={\imprimirautor},
    	colorlinks=true,       		% false: boxed links; true: colored links
    	linkcolor=blue,          	% color of internal links
    	citecolor=blue,        		% color of links to bibliography
    	filecolor=magenta,      		% color of file links
		urlcolor=blue,
		bookmarksdepth=4
}
% --- 

% ---
% compila o indice
% ---
\makeindex
% ---

% ---
% Compila a lista de abreviaturas e siglas
% ---
\makenomenclature
% ---

% ----
% Início do documento
% ----
\begin{document}

% ----------------------------------------------------------
% ELEMENTOS PRÉ-TEXTUAIS
% ----------------------------------------------------------
% \pretextual

% ---
% Capa
% ---
\imprimircapa
% ---

% ---
% Folha de rosto
% (o * indica que haverá a ficha bibliográfica)
% ---
\imprimirfolhaderosto*
% ---

% ---
% Inserir a ficha bibliografica
% ---
%
% Isto é um exemplo de Ficha Catalográfica, ou ``Dados internacionais de
% catalogação-na-publicação''. Você pode utilizar este modelo como referência. 
% Porém, provavelmente a biblioteca da sua universidade lhe fornecerá um PDF
% com a ficha catalográfica definitiva após a defesa do trabalho. Quando estiver
% com o documento, salve-o como PDF no diretório do seu projeto e substitua todo
% o conteúdo de implementação deste arquivo pelo comando abaixo:
%
% \begin{fichacatalografica}
%     \includepdf{fig_ficha_catalografica.pdf}
% \end{fichacatalografica}
%

\begin{fichacatalografica}
	\vspace*{\fill}					% Posição vertical
	\hrule							% Linha horizontal
	\begin{center}					% Minipage Centralizado
	\begin{minipage}[c]{12.5cm}		% Largura
	
	\imprimirautor
	
	\hspace{0.5cm} \imprimirtitulo  / \imprimirautor. --
	\imprimirlocal, \imprimirdata-
	
	\hspace{0.5cm} \pageref{LastPage} p. : il. (algumas color.) ; 30 cm.\\
	
	\hspace{0.5cm} \imprimirorientadorRotulo \imprimirorientador\\
	
	\hspace{0.5cm}
	\parbox[t]{\textwidth}{\imprimirtipotrabalho~--~\imprimirinstituicao,
	\imprimirdata.}\\
	
	\hspace{0.5cm}
		1. Palavra-chave1.
		2. Palavra-chave2.
		I. Orientador.
		II. Universidade xxx.
		III. Faculdade de xxx.
		IV. Título\\ 			
	
	\hspace{8.75cm} CDU 02:141:005.7\\
	
	\end{minipage}
	\end{center}
	\hrule
\end{fichacatalografica}
% ---

% ---
% Inserir errata
% ---
\begin{errata}
Elemento opcional da \citeonline[4.2.1.2]{NBR14724:2011}. Exemplo:

\vspace*{1cm}

FERRIGNO, C. R. A. \textbf{Tratamento de neoplasias ósseas apendiculares com
reimplantação de enxerto ósseo autólogo autoclavado associado ao plasma
rico em plaquetas}: estudo crítico na cirurgia de preservação de membro em
cães. 2011. 128 f. Tese (Livre-Docência) - Faculdade de Medicina Veterinária e
Zootecnia, Universidade de São Paulo, São Paulo, 2011.

\begin{table}[htb]
\center
\footnotesize
\begin{tabular}{|p{1.4cm}|p{1cm}|p{3cm}|p{3cm}|}
  \hline
   \textbf{Folha} & \textbf{Linha}  & \textbf{Onde se lê}  & \textbf{Leia-se}  \\
    \hline
    1 & 10 & auto-conclavo & autoconclavo\\
   \hline
\end{tabular}
\end{table}

\end{errata}
% ---

% ---
% Inserir folha de aprovação
% ---
%
% Isto é um exemplo de Folha de aprovação, elemento obrigatório da NBR
% 14724/2011 (seção 4.2.1.3). Você pode utilizar este modelo até a aprovação
% do trabalho. Após isso, substitua todo o conteúdo deste arquivo por uma
% imagem da página assinada pela banca com o comando abaixo:
%
% \includepdf{folhadeaprovacao_final.pdf}

\begin{folhadeaprovacao}

  \begin{center}
    \vspace*{1cm}
    {\ABNTEXchapterfont\large\imprimirautor}

    \vspace*{\fill}\vspace*{\fill}
    {\ABNTEXchapterfont\Large\imprimirtitulo}
    \vspace*{\fill}
    
    \hspace{.45\textwidth}
    \begin{minipage}{.5\textwidth}
        \imprimirpreambulo
    \end{minipage}%
    \vspace*{\fill}
   \end{center}
    
   Trabalho aprovado. \imprimirlocal, 24 de novembro de 2012:

   \assinatura{\textbf{\imprimirorientador} \\ Orientador} 
   \assinatura{\textbf{Professor} \\ Convidado 1}
   \assinatura{\textbf{Professor} \\ Convidado 2}
%    \assinatura{\textbf{Professor} \\ Convidado 3}
%    \assinatura{\textbf{Professor} \\ Convidado 4}
      
   \begin{center}
    \vspace*{0.5cm}
    {\large\imprimirlocal}
    \par
    {\large\imprimirdata}
    \vspace*{1cm}
  \end{center}
  
\end{folhadeaprovacao}
% ---

% ---
% Dedicatória
% ---
\begin{dedicatoria}
   \vspace*{\fill}
\noindent\textit{Este trabalho é dedicado à criança de León Werth, por ter sido
o maior amigo do mundo de Antoine de Saint-Exupery, pai de um pequeno príncipe que cativa muitas
imaginações.}
 \vspace*{\fill}
\end{dedicatoria}
% ---

% ---
% Agradecimentos
% ---
\begin{agradecimentos}
Os agradecimentos principais são direcionados à Miguel Frasson, Gerald Weber,
Leslie H. Watter, Bruno Parente Lima, Flávio de Vasconcellos Corrêa, Otavio Real
Salvador, Renato Machnievscz\footnote{Os nomes dos primeiros integrantes do
projeto abn\TeX foram extraídos de
\url{http://codigolivre.org.br/projects/abntex/}} e todos aqueles que
contribuíram para que a produção de trabalhos acadêmicos conforme
as normas ABNT com \LaTeX~ fosse possível.

Agradecimentos especiais são direcionados ao grupo de usuários
\emph{latex-br}\footnote{\url{http://groups.google.com/group/latex-br}} e aos
novos voluntários do grupo \emph{\abnTeX}
\footnote{\url{http://groups.google.com/group/abntex2} e
\url{https://code.google.com/p/abntex2/}}~que contribuíram e que ainda
contribuirão para a evolução do abn\TeX.
\end{agradecimentos}
% ---

% ---
% Epígrafe
% ---
\begin{epigrafe}
    \vspace*{\fill}
	\begin{flushright}
		\textit{``Não vos amoldeis às estruturas deste mundo, \\
		mas transformai-vos pela renovação da mente, \\
		a fim de distinguir qual é a vontade de Deus: \\
		o que é bom, o que Lhe é agradável, o que é perfeito.\\
		(Bíblia Sagrada, Romanos 12, 2)}
	\end{flushright}
\end{epigrafe}
% ---

% ---
% RESUMOS
% ---

% resumo em português
\begin{resumo}
 Segundo a \citeonline[3.1-3.2]{NBR6028:2003}, o resumo deve ressaltar o
 objetivo, o método, os resultados e as conclusões do documento. A ordem e a extensão
 destes itens dependem do tipo de resumo (informativo ou indicativo) e do
 tratamento que cada item recebe no documento original. O resumo deve ser
 precedido da referência do documento, com exceção do resumo inserido no
 próprio documento. (\ldots) As palavras-chave devem figurar logo abaixo do
 resumo, antecedidas da expressão Palavras-chave:, separadas entre si por
 ponto e finalizadas também por ponto.

 \vspace{\onelineskip}
    
 \noindent
 \textbf{Palavras-chaves}: latex. abntex. editoração de texto.
\end{resumo}

% resumo em inglês
\begin{resumo}[Abstract]
 This is the english abstract.

 \vspace{\onelineskip}
 
 \noindent 
 \textbf{Key-words}: latex. abntex. text editoration.
\end{resumo}

% resumo em francês 
\begin{resumo}[Résumé]
  Il s'agit d'un résumé en français.
 
 \vspace{\onelineskip}
 
 \noindent
 \textbf{Mots-clés}: latex. abntex. publication de textes.
\end{resumo}

% resumo em espanhol
\begin{resumo}[Resumen]
  Este es el resumen de español.
  
 \vspace{\onelineskip}
 
 \noindent
 \textbf{Palabras clave}: latex. abntex. publicación de textos.
\end{resumo}
% ---

% ---
% inserir lista de ilustrações
% ---
\pdfbookmark[0]{\listfigurename}{lof}
\listoffigures*
\cleardoublepage
% ---

% ---
% inserir lista de tabelas
% ---
\pdfbookmark[0]{\listtablename}{lot}
\listoftables*
\cleardoublepage
% ---

% ---
% inserir lista de abreviaturas e siglas
% A lista de Abreviaturas e Siglas pode ser facilmente montada com o pacote 
% nomencl. Abaixo segue um exemplo.
% ---
\nomenclature{Fig.}{Figura}
\nomenclature{$A_i$}{Area of the $i^{th}$ component} 
\nomenclature{456}{Isto é um número}
\nomenclature{123}{Isto é outro número}
\nomenclature{a}{primeira letra do alfabeto}
\nomenclature{lauro}{este é meu nome} 

\renewcommand{\nomname}{Lista de abreviaturas e siglas}
\pdfbookmark[0]{\nomname}{las}
\printnomenclature
\cleardoublepage
% ---

% ---
% inserir lista de símbolos
% ---
% O abnTeX2 não provê mecanismo para lista de símbolos.
% ---

% ---
% inserir o sumario
% ---
\pdfbookmark[0]{\contentsname}{toc}
\tableofcontents*
\cleardoublepage
% ---



% ----------------------------------------------------------
% ELEMENTOS TEXTUAIS
% ----------------------------------------------------------
% É possível usar \textual ou \mainmatter, que é a macro padrão do memoir.  
\mainmatter

% ----------------------------------------------------------
% Introdução
% ----------------------------------------------------------
\chapter*{Introdução}
\addcontentsline{toc}{chapter}{Introdução}

Este documento exemplifica o uso da classe \abnTeX. A expressão ``Modelo
canônico'' é utilizada para indicar que \abnTeX~não é modelo específico de
trabalho acadêmico de nenhuma universidade ou instituição, mas que implementa
tão somente os requisitos das normas da ABNT.

Sinta-se convidado à participar do projeto \abnTeX! Acesse o site do projeto em
\url{http://code.google.com/p/abntex2/}. Também fique livre para conhecer,
estudar, alterar e redistribuir o trabalho do \abnTeX, desde que os arquivos
modificados tenham seus nomes alterados e que os créditos sejam dados aos
autores originais, nos termos da ``The LaTeX Project Public
License''\footnote{\url{http://www.latex-project.org/lppl.txt}}.

Encorajamos que sejam realizadas customizações específicas para universidades
--- como capas, folha de aprovação, etc. Porém, recomendamos que ao invés de se
alterar diretamente os arquivos do \abnTeX, distribua-se aos estudantes
arquivos com as respectivas customizações. Isso permite que futuras versões do
\abnTeX~não se tornem automaticamente incompatíveis com as customizações
promovidas.

Este texto deve ser utilizado como complemento do Manual do \abnTeX~e do
\textbf{memoir}, este disponível em
\url{http://mirrors.ctan.org/macros/latex/contrib/memoir/memman.pdf}.


% ----------------------------------------------------------
% PARTE - preparação da pesquisa
% ----------------------------------------------------------
\part{Preparação da pesquisa}

% ----------------------------------------------------------
% Capitulo 1
% ----------------------------------------------------------
\chapter{Lorem ipsum dolor sit amet}

\chapterprecis{Phasellus eu tellus sit amet tortor gravida placerat, lacus
libero, pretium at, lobortis vitae, ultricies et, tellus cum sociis natoque
penatibus et magnis dis parturient montes.}

\section{Aliquam ullamcorper euismod nulla mollis enim vel tortor sodales placerat nunc
tempus rutrum wisi accumsan gravida purus etiam facilisis dui eu sem semper}

Quando for necessário enumerar os diversos assuntos de uma seção que não possua
título, esta deve ser subdividida em alíneas \cite[4.2]{NBR6024:2012}:

\begin{alineas}

  \item os diversos assuntos que não possuam título próprio, dentro de uma mesma
  seção, devem ser subdivididos em alíneas\footnote{As notas devem ser digitadas ou datilografadas
  dentro das margens, ficando separadas do texto por um espaço simples de entre as
  linhas e por filete de 5 cm, a partir da margem esquerda. Devem ser
  alinhadas, a partir da segunda linha da mesma nota, abaixo da primeira letra
  da primeira palavra, de forma a destacar o expoente, sem espaço entre elas e
  com fonte menor. \citeonline[5.2.1]{NBR14724:2011}}; 
  
  \item o texto que antecede as alíneas termina em dois pontos;
  \item as alíneas devem ser indicadas alfabeticamente, em letra minúscula,
  seguida de parêntese. Utilizam-se letras dobradas, quando esgotadas as
  letras do alfabeto;

  \item as letras indicativas das alíneas devem apresentar recuso em relação à
  margem esquerda;

  \item o texto da alínea deve começar por letra minúscula e terminar em
  ponto-e-vírgula, exceto a última alínea que termina em ponto final;

  \item o texto da alínea deve terminar em dois pontos, se houver subalínea;

  \item a segunda e as seguintes linhas do texto da alínea começa sob a
  primeira letra do texto da própria alínea;
  
  \item subalíneas \cite[4.3]{NBR6024:2012} deve ser conforme as alíneas a seguir:

  \begin{alineas}
     \item as subalíneas devem começar por travessão seguido de espaço;

     \item as subalíneas devem apresentar recuo em relação à alínea;

     \item o texto da subalínea deve começar por letra minúscula e terminar em
     ponto-e-vírgula. A última subalínea deve terminar em ponto final, se não
     houver alínea subsequente;

     \item a segunda e as seguintes linhas do texto da subalínea começam sob a
     primeira letra do texto da própria subalínea.
  \end{alineas}
  
  \item estão disponíveis os ambientes ``incisos'' e ``subalineas'', que em sua
  são o mesmo que se criar outro nível de ``alineas'':
  
  \begin{incisos}
    \item \lipsum[52]
  \end{incisos}
  
  \item \lipsum[60]
  
  \begin{subalineas}
    \item \lipsum[53]
  \end{subalineas}
  
  \item \lipsum[54]
  
\end{alineas}

\lipsum[4]

\begin{table}[htb]
\footnotesize
\caption[Níveis de investigação]{\footnotesize{Níveis de investigação.
\cite{van86}}}
\label{tab-nivinv}
\begin{tabular}{p{2.6cm}|p{6.0cm}|p{2.25cm}|p{3.40cm}}
  %\hline
   \textbf{Nível de Investigação} & \textbf{Insumos}  & \textbf{Sistemas de Investigação}  & \textbf{Produtos}  \\
    \hline
    Meta-nível & Filosofia\index{Filosofia} da Ciência  & Epistemologia &
    Paradigma  \\
    \hline
    Nível do objeto & Paradigmas do metanível e evidências do nível inferior &
    Ciência  & Teorias e modelos \\
    \hline
    Nível inferior & Modelos e métodos do nível do objeto e problemas do nível inferior & Prática & Solução de problemas  \\
   % \hline
\end{tabular}
\end{table}

\lipsum[5]

\section{Aliquam consectetuer varius nulla}

\lipsum[6-7]

\subsection{Integer vitae justo}

\lipsum[8]

\begin{citacao}
As citações diretas, no texto, com mais de três linhas, devem ser
destacadas com recuo de 4 cm da margem esquerda, com letra menor que a do texto
utilizado e sem as aspas. No caso de documentos datilografados, deve-se
observar apenas o recuo \cite[5.3]{NBR10520:2002}
\end{citacao}

\lipsum[9]

\subsubsection{Integer vitae extra justo}

\lipsum[9-10]

\subsubsection{Nulla vitae fringilla}

\lipsum[11-12]

\subsection{Mauris nibh leo}

\lipsum[12-13]

\section{Sed ante tellus}

\lipsum[14-15]

% ----------------------------------------------------------
% Parte de revisãod e literatura
% ----------------------------------------------------------
\part{Revisão de Literatura}

% ---
% Capitulo de revisão de literatura
% ---
\chapter{Etiam eget ligula eu lectus lobortis condimentum}

\section{Aliquam vestibulum fringilla lorem}

\lipsum[1]

\begin{figure}[htb]
	\caption{\label{fig_circulo}A delimitação do espaço}
	\begin{center}
	    \setlength{\unitlength}{5cm}
		\begin{picture}(1,1)
		\put(0,0){\line(0,1){1}}
		\put(0,0){\line(1,0){1}}
		\put(0,0){\line(1,1){1}}
		\put(0,0){\line(1,2){.5}}
		\put(0,0){\line(1,3){.3333}}
		\put(0,0){\line(1,4){.25}}
		\put(0,0){\line(1,5){.2}}
		\put(0,0){\line(1,6){.1667}}
		\put(0,0){\line(2,1){1}}
		\put(0,0){\line(2,3){.6667}}
		\put(0,0){\line(2,5){.4}}
		\put(0,0){\line(3,1){1}}
		\put(0,0){\line(3,2){1}}
		\put(0,0){\line(3,4){.75}}
		\put(0,0){\line(3,5){.6}}
		\put(0,0){\line(4,1){1}}
		\put(0,0){\line(4,3){1}}
		\put(0,0){\line(4,5){.8}}
		\put(0,0){\line(5,1){1}}
		\put(0,0){\line(5,2){1}}
		\put(0,0){\line(5,3){1}}
		\put(0,0){\line(5,4){1}}
		\put(0,0){\line(5,6){.8333}}
		\put(0,0){\line(6,1){1}}
		\put(0,0){\line(6,5){1}}
		\end{picture}
	\end{center}
	\legend{Fonte: os autores}
	
\end{figure}

\lipsum[2-3]

% ----------------------------------------------------------
% Resultados
% ----------------------------------------------------------
\part{Resultados}

% ---
% primeiro capitulo de Resultados
% ---
\chapter{Lectus lobortis condimentum}

\section{Vestibulum ante ipsum primis in faucibus orci luctus et ultrices
posuere cubilia Curae}

\lipsum[21-22]

% ---
% segundo capitulo de Resultados
% ---
\chapter{Nam sed tellus sit amet lectus urna ullamcorper tristique interdum
elementum}

\section{Pellentesque sit amet pede ac sem eleifend consectetuer}

\lipsum[24]

% ---
% Finaliza a parte no bookmark do PDF, para que se inicie o bookmark na raiz
% ---
\bookmarksetup{startatroot}% 
% ---

% ----------------------------------------------------------
% ELEMENTOS PÓS-TEXTUAIS
% ----------------------------------------------------------
\postextual

% ---
% Conclusão
% ---
\chapter*{Conclusão}
\addcontentsline{toc}{chapter}{Conclusão}

\lipsum[31-33]

% ----------------------------------------------------------
% Referências bibliográficas
% ----------------------------------------------------------
\bibliography{abntex2-modelo-references}

% ----------------------------------------------------------
% Glossário
% ----------------------------------------------------------
%
% Há diversas soluções prontas para glossário. Não é necessário nos preocuparmos
% com isso agora.
%
%\glossary

% ----------------------------------------------------------
% Apêndices
% ----------------------------------------------------------

% ---
% Inicia os apêndices
% ---
\begin{apendicesenv}

% Imprime uma página indicando o início dos apêndices
\appendixpage

% ----------------------------------------------------------
\chapter{Quisque libero justo}
% ----------------------------------------------------------

\lipsum[50]

% ----------------------------------------------------------
\chapter{Nullam elementum urna vel imperdiet sodales elit ipsum pharetra ligula
ac pretium ante justo a nulla curabitur tristique arcu eu metus}
% ----------------------------------------------------------
\lipsum[55-57]

\end{apendicesenv}
% ---


% ----------------------------------------------------------
% Anexos
% ----------------------------------------------------------
\cftinserthook{toc}{AAA}
% ---
% Inicia os anexos
% ---
%\anexos
\begin{anexosenv}

% Imprime uma página indicando o início dos anexos
\appendixpage

% ---
\chapter{Morbi ultrices rutrum lorem.}
% ---
\lipsum[30]

% ---
\chapter{Cras non urna sed feugiat cum sociis natoque penatibus et magnis dis
parturient montes nascetur ridiculus mus}
% ---

\lipsum[31]

% ---
\chapter{Fusce facilisis lacinia dui}
% ---

\lipsum[32]

\end{anexosenv}

%---------------------------------------------------------------------
% INDICE REMISSIVO
%---------------------------------------------------------------------

% \cleardoublepage
% \phantomsection 
\printindex

\end{document}